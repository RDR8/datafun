\documentclass{article}
\usepackage[margin=0.5in]{geometry}

\usepackage{datafun}

\begin{document}

\section{Core Datafun}


\subsection{Syntax}

\[
\begin{array}{rccl}
  \textsf{types} & A,B,C
  &\bnfeq& A \to B \pipe A \x B \pipe A + B
  \pipe \Disc{A} \pipe \Down{\eq{A}} \pipe \tbool
  \vspace{1em}\\

  %% equality types
  \textsf{equality types} & \eq{A}
  &\bnfeq& \eq{A} \x \eq{B} \pipe \eq{A} + \eq{B}
  \pipe \Disc{\eq{A}} \pipe {\color{red} \Down{\eq{A}}} \pipe \tbool
  \vspace{1em}\\

  %% finite types
  \textsf{finite types} & \fin{A}
  &\bnfeq& {\color{blue} \fin{A} \to \fin{B}}
  \pipe \fin{A} \x \fin{B} \pipe \fin{A} + \fin{B}
  \pipe \Disc{\fin{A}} \pipe \Down{\fin{A}} \pipe \tbool
  \vspace{1em}\\

  %% acc types
  \textsf{acc types} & \acc{A}
  &\bnfeq& {\color{blue} \fin{A} \to \acc{B}}
  \pipe \acc{A} \x \acc{B} \pipe \acc{A} + \acc{B}
  \pipe {\color{blue} \Disc{A}} \pipe \Down{\fin{A}} \pipe \tbool
  \vspace{1em}\\

  %% semilattice types
  \textsf{semilattice types} & \lat{A}
  &\bnfeq& A \to \lat{B} \pipe \lat{A} \x \lat{B} \pipe \Down{A} \pipe \tbool
  \vspace{1em}\\

  %% fixpoint types
  \textsf{fixed-point types} & \under{fix}{A}
  &=& \acc{A} \cap \eq{A} \cap \lat{A}
  \vspace{1em}\\

  %% expressions
  \textsf{expressions} & e,f,g
  &\bnfeq& x \pipe \fn\bind{x} e \pipe e\; e
  \pipe \pair{e}{e} \pipe \pi_i\; e
  \pipe \inj{i}{e} \pipe \case{e}{x}{e_1}{x}{e_2} \\
  &&\pipe& \disc{e} \pipe \letdisc{x}{e}{e} \pipe \splitsum{e}\\
  &&\pipe& \unit \pipe e \vee e \pipe \downintro{e} \pipe \downelim{x}{e}{e}\\
  &&\pipe& \ett \pipe \eff \pipe \when{e}{e} \pipe \ifthen{e}{e}{e}\\
  &&\pipe& \fix{x}{e} \pipe \fixle{x}{e}{e}
\end{array}
\]

I've highlighted the equality type ${\color{red} \Down{\eq{A}}}$ because testing
down-sets for equality is nontrivial to implement. If finite sets $\{A\}$ are
implemented as $\Down{\Disc{A}}$, this may have performance implications. A less
general approach would be to consider only $\Down{\Disc{\eq{A}}}$ to be an
equality type.

I've put some justified but probably unnecessary rules in {\color{blue} blue}.


%% \subsection{Properties of types}

%% Some decidable judgments about types:
%% \begin{quote}
%%   \begin{tabular}{ll}
%%     $A \is{eq}$ & $A$ is an equality type.\\
%%     $A \is{fin}$ & $A$ has finitely many inhabitants.\\
%%     $A \is{acc}$ & $A$ satisfies ACC.\\
%%   \end{tabular}
%% \end{quote}

%% And their rules:
%% \begin{mathpar}
%%   \infer{A \x B \is{eq}}{A \is{eq} & B \is{eq}}

%%   \infer{A + B \is{eq}}{A \is{eq} & B \is{eq}}

%%   \infer{\Disc{A} \is{eq}}{A \is{eq}}

%%   {\color{red} \infer{\Down{A} \is{eq}}{A \is{eq}}}

%%   \infer{\tbool \is{eq}}{}
%%   \\
%%   {\color{blue} \infer{A \to B \is{fin}}{A \is{fin} & B \is{fin}}}

%%   \infer{A \x B \is{fin}}{A \is{fin} & B \is{fin}}

%%   \infer{A + B \is{fin}}{A \is{fin} & B \is{fin}}

%%   \infer{\Disc{A} \is{fin}}{A \is{fin}}

%%   \infer{\Down{A} \is{fin}}{A \is{fin}}

%%   \infer{\tbool \is{fin}}{}
%%   \\
%%   {\color{blue} \infer{A \to B \is{acc}}{A \is{fin} & B \is{acc}}}

%%   \infer{A \x B \is{acc}}{A \is{acc} & B \is{acc}}

%%   \infer{A + B \is{acc}}{A \is{acc} & B \is{acc}}

%%   {\color{blue} \infer{\Disc{A} \is{acc}}{}}

%%   \infer{\Down{A} \is{acc}}{A \is{fin}}

%%   \infer{\tbool \is{acc}}{}
%% \end{mathpar}

%% I've put in {\color{blue} blue} some rules that I think are justified, but
%% probably unnecessary.

%% I've highlighted the rule that says when downsets form equality types in
%% {\color{red} red} because it might take some nontrivial work to implement
%% equality testing for downsets. A less powerful but easier to implement rule
%% would be:
%% \[
%% \infer{\Down{\Disc{A}} \is{eq}}{A \is{eq}}
%% \]

%% 
\subsection{Typing rules}

The typing judgment \[\J{\GP}{\GG}{e}{A}\] says that $e$ has type $A$ using
variables with types given by $\GP \cup \GG$, and moreover uses the variables in
$\GG$ in a monotone fashion.

Where possible without ambiguity, I omit the contexts $\GP;\GG$ from typing
rules.

\begin{mathpar}
  \infer{\J{\GP}{\GG}{x}{A}}{x \of A \in \GP \cup \GG}

  \infer{\J{\GP}{\GG}{\fn\bind{x}{e}}{A \to B}}{
    \J{\GP}{\GG,x \of A}{e}{B}}

  \infer{e \; f : B}{e : A \to B & f : A}
  \\
  %% \infer{\pair{e}{f} : A \x B}{e : A & f : B}
  \infer{\pair{e_1}{e_2} : A_1 \x A_2}{e_i : A_i}

  \infer{\pi_i\; e : A_i}{e : A_1 \x A_2}

  \infer{\inj{i}{e} : A_1 + A_2}{e : A_i}

  \infer{\case{e}{x}{f_1}{x}{f_2} : B}
        {e : A_1 + A_2 & \J{\GP}{\GG, x \of A_i}{f_i}{B}}
  \\
  \infer{\J{\GP}{\GG}{\disc{e}}{\Disc{A}}}{\J{\GP}{-}{e}{A}}

  \infer{\letdisc{x}{e}{f} : B}
        { e : \Disc{A}
        & \J{\GP,x\of A}{\GG}{f}{B} }

  \infer{\splitsum{e} : \Disc{A} + \Disc{B}}
        {e : \Disc{(A + B)}}
  \\
  \infer{\unit : \lat{A}}{}

  \infer{e_1 \vee e_2 : \lat{A}}{e_i : \lat{A}}

  \infer{\downintro{e} : \Down{A}}{e : A}

  \infer{\downelim{x}{e}{f} : \lat{B}}
        { e : \Down{A}
        & \J{\GP}{\GG,x \of A}{f}{\lat{B}}}
  \\
  \infer{\ett : \tbool}{}

  \infer{\eff : \tbool}{}

  \infer{\when{e}{f} : \lat{A}}{e : \tbool & f : \lat{A}}

  \infer{\ifthen{e}{f_1}{f_2} : A}{e : \Disc{\tbool} & f_i : A}
  \\
  \infer{\fix{x}{e} : \under{fix}{A}}{
    \J{\GP}{\GG,x\of \under{fix}{A}}{e}{\under{fix}{A}}}

  \infer{\fixle{x}{e}{f} : \under{eq,lat}{A}}{
    e : \under{eq,lat}{A} &
    \J{\GP}{\GG,x \of \under{eq,lat}{A}}{f}{\under{eq,lat}{A}} &
    {\color{red}\text{TODO: elementwise-ACC condition on $A$?}}}
\end{mathpar}


\section{A Theory of Changes for Core Datafun}

\subsection{Change types}

For each type $A$ we define a change type $\GD A$:
\begin{eqnarray*}
  \GD(A \to B) &=& \Disc{A} \to \Delta A \to \Delta B\\
  \GD(A \x B) &=& \GD A \x \GD B\\
  \GD(A + B) &=& \GD A + \GD B\\
  \GD(\Disc{A}) &=& \Disc{\GD A}\\
  \GD(\Down{A}) &=& \Down{A}\\
  \GD 2 &=& 2
\end{eqnarray*}

\todo{Theorems about change types, like that $\GD \lat{A}$ is a lattice.}


\subsection{Derivatives}

We wish to define an operator $\delta$ on typing derivations (or if possible, on
well-typed terms) such that the following holds:
\[\infer{\J{\GP,\GD\GP,\GG}{\GD\GG}{\delta(e)}{\Delta A}}
        {\J{\GP}{\GG}{e}{A}}
\]

As a matter of convention, prefixing a variable with $d$ refers to the
corresponding variable of its change type. That is, if $x \of A \in \GP \cup
\GG$, then $dx \of \GD A \in \GD\GP \cup \GD\GG$. We also transparently use
weakening and monotone-to-discrete weakening (that is, if
$\J{\GP}{\GG_1,\GG_2}{e}{A}$, then $\J{\GP,\GG_1}{\GG_2}{e}{A}$).

\[
\begin{array}{lcll}
  \delta x &=& dx\\
  \delta(\fn\bind{x} e) &=& \fn\bind{y} \letdisc{x}{y} {\fn\bind{dx} \delta(e)}
  & \text{(for fresh $y$)}\\
  \delta(e\;f) &=& \delta e \; f \; \delta f\\
  \delta \pair{e}{f} &=& \pair{\delta{e}}{\delta{f}}\\
  \delta(\pi_i\; e) &=& \pi_i\; \delta{e}\\
  \delta(\inj{i}{e}) &=& \inj{i}{\delta{e}}\\
  \delta(\case{e}{x}{f}{x}{g})
  &=& \case{e}
        {x}{\case{\delta e}{dx}{\delta f}{\_}{\ms{abort!}}}
        {x}{\case{\delta e}{\_}{\ms{abort!}}{dx}{\delta g}}\\
  \delta(\disc e) &=& \disc{\delta e}\\
  \delta(\letdisc{x}{e}{f}) &=& \text{\TODO}\\
  \delta(\splitsum{x}) &=& \text{\TODO}\\
  \delta(\unit) &=& \unit\\
  \delta(e \vee f) &=& \delta e \vee \delta f\\
  \delta(\downintro{e}) &=& \unit\\
  \delta(\downelim{x}{e}{f}) &=& \text{\TODO}\\
  \delta(\ett) &=& \eff\\
  \delta(\eff) &=& \eff\\
  \delta(\when{e}{f}) &=& \text{\TODO}\\
  \delta(\ifthen{e}{f}{g}) &=& \text{\TODO}\\
  \delta(\fix{x}{e}) &=& \text{\TODO}\\
  \delta(\fixle{x}{e}{f}) &=& \text{\TODO}
\end{array}
\]

\end{document}
