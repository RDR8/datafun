\documentclass{article}
\usepackage[a4paper,margin=0.8in]{geometry}

\usepackage{datafun}

%% Allows using >{blah} and <{blah} in array formats.
\usepackage{array}

%% for \midrule
\usepackage{booktabs}

\newcommand{\longarray}{\renewcommand\arraystretch{1.2}}

\newcommand{\dummy}{\ms{dummy}}
\newcommand{\shape}{\ms{shape}}

\newcommand{\preserve}[1]{{\color{ForestGreen}{#1}}}
\newcommand{\preservedisc}[1]{\preserve{\disc{#1}}}

\newcommand{\comment}[1]{\textit{--- {#1}}}

\begin{document}

\section{Poset terminology}

\begin{definition}[Chain]
  In a poset $A$, a \emph{chain} is a monotone map $f : \N \to A$, which we
  interpret as an infinite nondecreasing sequence of $A$s, $f(0) \le f(1) \le
  f(2) \le ...$.
\end{definition}

\begin{definition}[Strict chain]
  A chain $f : \N \to A$ is \emph{strict} if $i < j \implies f(i) < f(j)$; in
  other words, an infinite strictly increasing sequence of $A$s, $f(0) < f(1) <
  f(2) < ...$.
\end{definition}

\begin{definition}[Stops at]
  A chain $a_0 \le a_1 \le a_2 \le ...$ \emph{stops at $i$} iff $a_i$ is the
  supremum of all $a_j$. Equivalent formulations:
  \[\begin{array}{ccccc}
    a_i = \sup_j a_j
    &\Leftrightarrow& \forall(j \ge i)\ a_j \le a_i
    &\Leftrightarrow& \forall(j \ge i)\ a_i = a_j\\
  \end{array}\]
\end{definition}

\begin{definition}[Stops]
  A chain $a$ \emph{stops} iff it contains its own supremum (iff $\exists
  i\binder a~\text{stops at}~i$).
\end{definition}

\begin{observation} Strict chains don't stop. \end{observation}

\begin{definition}
  For a chain $f$, let the sequence $f^s$ (``the strictification of $f$'') be
  $f$ with all repeated elements stripped out. If $f^s$ is infinite, it is by
  construction a strict chain.
\end{definition}

%% \begin{observation}
%%   If $f$ is a strict chain, $f^s = f$, since strict chains have no repeated
%%   elements.
%% \end{observation}

\begin{lemma}
  Given a chain $a$, $a^s$ is finite iff $a$ stops (and $a^s$ is a strict chain
  iff $a$ does not stop).
\end{lemma}
\begin{proof}
  Note that the ranges of $a$ and $a^s$ are identical by construction.

  First, suppose $a^s$ is finite. Note that the last element of $a^s$ is its
  largest, and therefore the supremum of both sequences, since they have the
  same range. Therefore $a$ contains its own supremum, and stops.

  Second, suppose $a$ stops. Thus it contains the supremum of both sequences.
  Let $i$ be the index such that $a^s_i$ is this supremum. There can be no
  elements in $a^s$ after $a^s_i$. Thus $a^s$ is finite.
\end{proof}

\begin{definition}[ACC]
  A poset satisfies the \emph{ascending chain condition (ACC)} iff all chains in
  it stop, or equivalently, if it has no strict chains.
\end{definition}

``Every chain stops'' implies no strict chains exist, since every strict chain
is a chain and strict chains do not stop. Likewise, ``no strict chains exist''
implies every chain stops, since for every non-stopping chain $f$ there is a
strict chain $f^s$.

\begin{definition}[Elementwise ACC]
  A poset $A$ is \emph{elementwise-ACC} iff for any $x : A$, the induced
  sub-poset $\setfor{y}{y \le x}$, consisting of the elements below $x$, is ACC.
\end{definition}


\section{Core Datafun}

\subsection{Syntax}

\[
\begin{array}{rccl}
  \textsf{types} & A,B,C
  &\bnfeq& A \to B \pipe A \x B \pipe A + B
  \pipe \Disc{A} \pipe \Seteq{A} \pipe \tbool
  \vspace{1em}\\

  %% types where the poset relationship is decidable
  \textsf{decidable types} & \eq{A}
  &\bnfeq& \eq{A} \x \eq{B} \pipe \eq{A} + \eq{B}
  \pipe \Disc{\eq{A}} \pipe \Seteq{A} \pipe \tbool
  \vspace{1em}\\

  %% finite types
  \textsf{finite types} & \fin{A}
  &\bnfeq& \fin{A} \x \fin{B} \pipe \fin{A} + \fin{B}
  \pipe \Disc{\fin{A}} \pipe \Set{\fin{A}} \pipe \tbool
  \vspace{1em}\\

  %% acc types
  \textsf{ACC types} & \acc{A}
  &\bnfeq& \acc{A} \x \acc{B} \pipe \acc{A} + \acc{B}
  \pipe \Disc{A} \pipe \Set{\fin{A}} \pipe \tbool
  \vspace{1em}\\

  %% elementwise-ACC types
  \textsf{elementwise-ACC types} & \eltacc{A}
  &\bnfeq& \eltacc{A} \x \eltacc{B} \pipe \eltacc{A} + \eltacc{B}
  \pipe \Disc{A} \pipe \Seteq{A} \pipe \tbool
  \vspace{1em}\\

  %% semilattice types
  \textsf{semilattice types} & \lat{A}
  &\bnfeq& A \to \lat{B} \pipe \lat{A} \x \lat{B} \pipe \Seteq{A} \pipe \tbool
  \vspace{1em}\\

  %% decidable semilattice types
  \textsf{decidable semilattice types} & \eqlat{A}
  &=& \eq{A} \cap \lat{A}
  \vspace{1em}\\

  %% fixpoint types
  \textsf{fixed-point types} & \fixtype{A}
  &=& \eq{A} \cap \lat{A} \cap \acc{A}
  \vspace{1em}\\

  %% TODO: note that, currently, \eq{A} \cap \eltacc{A} = \eq{A}?
  %% however, this would change if we added lexical sums.
  %% (nat <+ 1) is not elementwise-acc, for example.
  \textsf{bounded fixed-point types} & \fixletype{A}
  &=& \eq{A} \cap \lat{A} \cap \eltacc{A}
  \vspace{1em}\\

  %% expressions
  \textsf{expressions} & e,f,g
  &\bnfeq& \m{x} \pipe \d{x} \pipe \fn\bind{\m{x}} e \pipe e\; e
  \pipe \pair{e}{e} \pipe \pi_i\; e\\
  &&\pipe& \inj{i}{e} \pipe \case{e}{\m{x}}{e_1}{\m{x}}{e_2}\\
  &&\pipe& \disc{e} \pipe \letdisc{\d{x}}{e}{e} \pipe \splitsum{e}\\
  &&\pipe& \unit \pipe e \vee e \pipe \single{e} \pipe \setelim{\m{x}}{e}{e}\\
  &&\pipe& \etrue \pipe \efalse \pipe \when{e}{e} \pipe \ifthen{e}{e}{e}\\
  &&\pipe& \fix{\m{x}}{e} \pipe \fixle{\m{x}}{e}{e}
\end{array}
\]

I use blue italics for monotone variables $\m{x}$, and orange bold for discrete variables $\d{x}$.

The ``typeclass'' rules (for decidable, finite, ACC, elementwise-ACC, semilattice, etc.\! types) are \emph{conservative approximations} of the corresponding semantic conditions.

``Decidable types'' are those whose partial ordering relation is practically decidable; that is, given $x, y : \eq{A}$, there is a reasonable algorithm that determines whether $x \le y$.
%
By antisymmetry, this also makes equality $x = y$ testable at these types. Theoretically, a function type $\under{fin}{A} \to \eq{B}$ whose domain is finite and codomain is decidable is also decidable.
%
We have omitted this, as it complicates our implementation for little practical
benefit.
%
We have similarly omitted the cases in which function types are finite $\fin{A} \to \fin{B}$, ACC $\fin{A} \to \acc{B}$, and elementwise-ACC $\fin{A} \to \eltacc{B}$.


\subsection{Typing rules}

The typing judgment \[\J{\GP}{\GG}{e}{A}\] says that $e$ has type $A$ using variables with types given by $\d{\GP} \cup \m{\GG}$, and moreover uses the variables in $\m{\GG}$ in a monotone fashion.
%
Where possible without ambiguity, I omit the contexts $\dt{\GP};\mt{\GG}$ from typing rules.

\begin{mathpar}
  \infer{\J{\GP}{\GG}{\m{x}}{A}}{\mt{\m{x} \of A \in \GG}}

  \infer{\J{\GP}{\GG}{\d{x}}{A}}{\dt{\d{x} \of A \in \GP}}

  \infer{\fn\bind{\m{x}} e : A \to B}{\J{\GP}{\GG,\m{x} \of A}{e}{B}}

  \infer{e \; f : B}{e : A \to B & f : A}
  \\
  %% \infer{\pair{e}{f} : A \x B}{e : A & f : B}
  \infer{\pair{e_1}{e_2} : A_1 \x A_2}{e_i : A_i}

  \infer{\pi_i\; e : A_i}{e : A_1 \x A_2}

  \infer{\inj{i}{e} : A_1 + A_2}{e : A_i}

  \infer{\case{e}{\m{x}}{f_1}{\m{x}}{f_2} : B}
        {e : A_1 + A_2 & \J{\GP}{\GG, \m{x} \of A_i}{f_i}{B}}
  \\
  \infer{\J{\GP}{\GG}{\disc{e}}{\Disc{A}}}{\J{\GP}{-}{e}{A}}

  \infer{\letdisc{\d{x}}{e}{f} : B}
        { e : \Disc{A}
        & \J{\GP, \d{x}\of A}{\GG}{f}{B}}

  \infer{\splitsum{e} : \Disc{A} + \Disc{B}}
        {e : \Disc{(A + B)}}
  \\
  \infer{\unit : \lat{A}}{}

  \infer{e_1 \vee e_2 : \lat{A}}{e_i : \lat{A}}

  \infer{\J{\GP}{\GG}{\single{e}}{\Seteq{A}}}
        { \J{\GP}{-}{e}{\eq{A}} }

  %% TODO: explain that although we allow only decidable types here, we can
  %% eta-expand function types, so we can pretend we allow any semilattice type.
  \infer{\setelim{\d{x}}{e}{f} : \eqlat{B}}
        { e : \Seteq{A}
        & \J{\GP, \d{x} \of A}{\GG}{f}{\eqlat{B}}}
  \\
  \infer{\etrue : \tbool}{}

  \infer{\efalse : \tbool}{}

  %% TODO: Same explanation about eta-expansion here
  \infer{\when{e}{f} : \eqlat{A}}{e : \tbool & f : \eqlat{A}}

  \infer{\ifthen{e}{f_1}{f_2} : A}{e : \Disc{\tbool} & f_i : A}
  \\
  \infer{\fix{\m{x}}{e} : \fixtype{A}}{
    \J{\GP}{\GG, \m{x}\of\fixtype{A}}{e}{\fixtype{A}}}

  \infer{\fixle{\m{x}}{e}{f} : \fixletype{A}}{
    e : \fixletype{A} &
    \J{\GP}{\GG, \m{x}\of\fixletype{A}}{f}{\fixletype{A}}}
\end{mathpar}


\subsection{Eta-expanding excessively expressive eliminators}
%% \subsection{$\eta$-expanding for expressive eliminators}

\todo{Explain how to lift the decidability constraint on the type of the conclusion in the rules for $\bigvee$ and $\mb{when}$ by $\eta$-expanding.}

We can translate away uses of $\bigvee$ and $\mb{when}$ at non-decidable semilattice types by $\eta$-expansion:

\[\longarray
\begin{array}{lclcl}
  \setelim{\d{x}}{e}{f} &:& A \to \lat{B}
  &=& \fn\bind{\m y} \setelim{\d{x}}{e}{f\; \m y}\\
  \setelim{\d{x}}{e}{f} &:& \lat{A} \x \lat{B}
  &=& \pair{\setelim{\d{x}}{e}{\pi_1\; f}}
          {\setelim{\d{x}}{e}{\pi_2\; f}}\\
  \when{e}{f} &:& A \to \lat{B} &=& \fn\bind{\m x} \when{e}{f\;\m{x}}\\
  \when{e}{f} &:& \lat{A} \x \lat{B}
  &=& \pair{\when{e}{\pi_1\; f}}{\when{e}{\pi_2\; f}}\\
\end{array}\]

%% This turns a \ms{when} at type $A \to \lat{B}$ into one at type $\lat{B}$. By extending this we can rewrite away all uses of \ms{when} at functional types, in the same manner as we can rewrite away uses of $\bigvee$ at functional type.


\section{A Theory of Changes for Core Datafun}

\subsection{Change types and operators}
\label{sec:change-types}

For each type $A$ we define a change type $\Ch A$.
%
\begin{eqnarray*}
  \Ch(A \to B) &=& \Disc{A} \to \Ch A \to \Ch B\\
  \Ch(A \x B) &=& \Ch A \x \Ch B\\
  \Ch(A + B) &=& \Ch A + \Ch B\\
  \Ch \Disc{A} &=& \Disc{\Ch A}\\
  \Ch\thinspace \Seteq{A} &=& \Seteq{A}\\
  \Ch\thinspace \tbool &=& \tbool
\end{eqnarray*}

We also define associated operators $\oplus : A \to \Ch A \to A$, $\ominus : A
\to A \to \Ch A$ and $\zero : A \to \Ch A$. Note, however, the
definitions of these operators \emph{are not well-typed Datafun terms}. \todo{I
  do not believe there exists a way to assign them tones that is compatible with
  Datafun's type system; every way I've tried fails. But I have not proved
  this.}

As for the semantic tonal behavior of these operators, I believe I know this
much:
\begin{itemize}
\item $x \oplus dx$ is monotone in $dx$.
\item $x \ominus y$ is monotone in $x$ and antitone in $y$.
\item $\zero$ is discrete; at function type, it finds the derivative of its
  argument, and a function that is larger pointwise does not necessarily have a
  derivative that is larger pointwise.
\end{itemize}

But is $\oplus$ monotone in its first argument? I'm not sure. It seems wrong to
contemplate changing its first argument without also changing its second ---
really it should have a dependent type.

\begin{center}
\[
\scalebox{0.85}{\(
\arraycolsep=1.5em
\begin{array}{cccc}
  \textbf{Type}
  & {\textbf{Definition of}~\oplus}
  & {\textbf{Definition of}~\ominus}
  & {\textbf{Definition of}~\zero}
  \\ \midrule
  A \to B
  & (f \oplus df)\; x = f\; x \oplus df\; x\; (\zero\; x)
  & (f \ominus g)\; x \;dx = f\; (x \oplus dx) \ominus g \; x
  & \zero\; f\; x\; dx = f\; (x \oplus dx) \ominus f\; x
  \\
  A \x B
  & \pair{x}{y} \oplus \pair{dx}{dy} = \pair{x \oplus dx}{y \oplus dy}
  & \pair{a}{x} \ominus \pair{b}{y} = \pair{a \ominus b}{x \ominus y}
  & \zero \; \pair{x}{y} = \pair{\zero\; x}{\zero\; y}
  \\
  A + B
  & \inj{i}{x} \oplus \inj{i}{dx} = \inj{i}{(x \oplus dx)}
  & \inj{i}{x} \ominus \inj{i}{y} = \inj{i}{x \ominus y}
  & \zero\; (\inj{i}{x}) = \inj{i}{(\zero\;x)}
  \\
  \Disc{A}
  & \disc{x} \oplus \disc{dx} = \disc{x \oplus dx}
  & \disc{x} \ominus \disc{y} = \disc{x \ominus y}
  & \zero\; \disc{x} = \disc{\zero\; x}
  \\
  \Seteq{A}
  & x \oplus y = x \vee y
  & x \ominus y = x
  & \zero\; x = \unit
  \\
  \tbool
  & x \oplus y = x \vee y
  & x \ominus y = x
  & \zero\; x = \unit
\end{array}
\)}
\]
\end{center}

These should satisfy the following laws:
\begin{eqnarray}
  x \le y \quad\implies\quad
  y &=& x \oplus (y \ominus x)\\
  x &=& x \oplus \zero\; x
\end{eqnarray}

\todo{TODO: Prove these laws hold.}

\todo{TODO: Note about ``erasure''/invalid changes. Moreover, at $\Disc{A}$
  type, only valid changes are zero changes.}

\todo{TODO: Note about efficiency of computing $\oplus$, $\ominus$, $\zero$,
  and which ones we actually need to compute in the implementation. Define
  efficiently computable $(- \ominus \unit)$ operator for decidable types, and
  point out that $\zero$ is efficiently computable for decidable types.}

%% Zero on decidable types:
%% \[\arraycolsep=1.5em
%% \begin{array}{c|c}
%%   \eq{A} \x \eq{B}
%%   & \zero\; (x,y) = (\zero\; x, \zero\; y)
%%   \\
%%   \eq{A} + \eq{B}
%%   & \zero\; (\inj{i}{x}) = \inj{i}(\zero\; x)\\
%%   \Disc{\eq{A}}
%%   & \zero\;(\disc{x}) = \disc{(\zero\; x)}\\
%%   \Down{\eq{A}} & \zero\;x = \unit\\
%%   \tbool & \zero\; x = \unit
%% \end{array}
%% \]


%% Some lemmas.
\begin{lemma}[Changes on a semilattice form a semilattice]
  $\Ch \lat{A}$ is a semilattice type.

  \todo{TODO: do we actually need/use this lemma?}
\end{lemma}
\begin{proof}
  By induction on the cases:
  \begin{eqnarray*}
    \Ch(A \to \lat{B}) &=& \Disc{A} \to \Ch A \to \Ch \lat{B}\\
    \Ch(\lat{A} \x \lat{B}) &=& \Ch \lat{A} \x \Ch \lat{B}\\
    \Ch\thinspace \Seteq{A} &=& \Seteq{A}\\
    \Ch\thinspace 2 &=& 2
  \end{eqnarray*}
\end{proof}

\begin{lemma}[Changes on decidable semilattices are boring]\label{lemma:dl-boring}
  For any decidable semilattice type $\eqlat{A}$, $\Ch \eqlat{A} = \eqlat{A}$,
  and moreover the following laws hold:
  \begin{eqnarray*}
    x \oplus y &=& x \vee y\\
    x \ominus y &=& x\\
    \zero\; x &=& \unit
  \end{eqnarray*}
\end{lemma}
\begin{proof}
  By induction on the cases:
  \begin{eqnarray*}
    \Ch(\eqlat{A} \x \eqlat{B}) &=& \Ch \eqlat{A} \x \Ch \eqlat{B}\\
    \Ch\thinspace \Seteq{A} &=& \Seteq{A}\\
    \Ch\thinspace \tbool &=& \tbool
  \end{eqnarray*}

  And, for the operations:
  \[\arraycolsep=0.75em\longarray
  \begin{array}{c|ccc}
    \eqlat{A} \x \eqlat{B}
    & (x,y) \oplus (dx,dy) = (x \oplus dx, y \oplus dy)
    & (a,x) \ominus (b,y) = (a \ominus b, x \ominus y)
    & \zero\; (x,y) = (\zero\; x, \zero\; y)\\
    \Set{\eq{A}}
    & x \oplus y = x \vee y
    & x \ominus y = x
    & \zero\;x = \unit\\
    \tbool
    & x \oplus y = x \vee y
    & x \ominus y = x
    & \zero\;x = \unit\\
  \end{array}
  \]
\end{proof}


\subsection{Derivatives}

We wish to define an operator $\delta$ on well-typed terms such that the following holds:
\[
\infer{\J{\GP,\GG,\Ch\GP}{\Ch\GG}{\delta e}{\Ch A}}
      {\J{\GP}{\GG}{e}{A}}
\]

As a matter of convention, prefixing a variable with $d$ refers to the corresponding variable of its change type.
%
So if $\dt{\d{x} \of A \in \GP}$, then $\dt{\d{dx} \of \Ch A \in \Ch \GP}$, and if $\mt{\m{x} \of A \in \GG}$, then $\mt{\m{dx} \of \Ch A \in \Ch \GG}$.
%
We also make implicit use of weakening and of monotone-to-discrete weakening: that is, if $\J{\GP}{\GG_1,\GG_2}{e}{A}$, then $\J{\GP,\GG_1}{\GG_2}{e}{A}$.
%
In particular, if $\J{\GP}{\GG}{e}{A}$, then $\J{\GP,\GG,\Ch\GP}{-}{e}{A}$.
%
This means $e$ may be used in its own derivative, even in positions which require an empty monotone context, such as $\disc{e}$ or $\single{e}$.
%
We highlight such uses in \preserve{green}.
%
\[
\begin{array}{lcll}
  \delta \m{x} &=& \m{dx}\\
  \delta \d{x} &=& \d{dx}\\
  \delta(\fn\bind{\m{x}} e) &=&
  \fn\bind{\disc{\d{x}} \, \m{dx}} \delta e\\
  \delta(e\;f) &=& \delta e \; \preservedisc{f} \; \delta f\\
  \delta \pair{e}{f} &=& \pair{\delta{e}}{\delta{f}}\\
  \delta(\pi_i\; e) &=& \pi_i\; \delta{e}\\
  \delta(\inj{i}{e}) &=& \inj{i}{\delta{e}}\\

  \delta(\case{e}{\m x}{f}{\m y}{g})
  &=& \mb{case}~ \pair{\splitsum{\preservedisc{e}}}{\delta e} ~\mb{of}\\
   && \quad \pair{\inj{1}{\disc{\d{x}}}}{\inj{1}{\m{dx}}} \cto \delta f\\
   && \quad \pair{\inj{2}{\disc{\d{y}}}}{\inj{2}{\m{dy}}} \cto \delta g\\
   && \quad \comment{The following cases are unreachable.}\\
   && \quad \pair{\inj{1}{\disc{\d{x}}}}{\inj{2}{\m{\pwild}}} \cto
      \letv{\m{dx}}{\dummy\; \d{x}} \delta f\\
   && \quad \pair{\inj{2}{\disc{\d{y}}}}{\inj{1}{\m{\pwild}}} \cto
      \letv{\m{dy}}{\dummy\; \d{y}} \delta g\\

  \delta\disc{e} &=& \disc{\delta e}\\
  \delta(\letdisc{\d{x}}{e} f) &=&
  %% TODO: Wait, is this right? Check with Agda code.
    \letdisc{\d{x}}{e} \letdisc{\d{dx}}{\delta e} \delta f\\

  \delta(\splitsum{e}) &=&
  %% want type: Δ□A + Δ□B = □ΔA + □ΔB
  %% e : □(A + B)
  %% δe : □(ΔA + ΔB)
  %% splitsum δe : □ΔA + □ΔB
  \splitsum{\delta e}\\
  \delta \unit &=& \unit\\
  \delta(e \vee f) &=& \delta e \vee \delta f
  \qquad \comment{This is a crucial over-approximation.}\\
  \delta(\single{e}) &=& \unit\\
  %% \delta(\setelim{\d{x}}{e}{f}) &=& \text{\todo{see below}}\\

  \delta(\setelim{\d{x}}{e}{f})
  &=& \phantom{\vee~} (\setelim{\d{x}}{e}
        {\phantom{\delta} \letv{\m{dx}}{\zero\; \d{x}} \delta f})\\
  && \vee~ (\setelim{\d{x}}{\delta e}
        {\letv{\m{dx}}{\zero\; \d{x}}
          (f \oplus \delta f) \ominus \unit})\\

  \delta(\etrue) &=& \efalse\\
  \delta(\efalse) &=& \efalse\\
  \delta(\when{e}{f})
  &=& \ifthen{\preservedisc{e}}{\delta f}{}\\
  && \when{\delta e}{(f \oplus \delta f) \ominus \unit}
  \\
  \delta(\ifthen{e}{f}{g}) &=& \ifthen{e}{\delta f}{\delta g}\\
  \delta(\fix{\m{x}}{e}) &=&
  \letdisc{\d{x}}
          {\disc{\fix{\m{x}} \preserve{e}}}
          {\fix{\m{dx}} \delta e}\\
  && \todo{\comment{I'm not sure this next definition is right yet.}}\\
  %% the best way to find out is probably to try to prove it, though
  \delta(\fixle{\m{x}}{e}{f}) &=&
    \letdisc{\d x}{\disc{\fixle{\m x}{\preserve{e}}{\preserve{f}}}}
    \fixle{\m{dx}}{(e \oplus \delta e) \ominus \unit}{\delta f}
\end{array}
\]


\subsubsection{Overapproximation}

\todo{Explain overapproximation by example. Explain why we do it (performance).
  Discuss which derivatives can produce approximations - is it only $\delta(e
  \vee f)$, or can $\delta(\setelim{\d{x}}{e}{f})$ also do it?}


\subsubsection{Computing dummy values}

The derivative of $\ms{case}$ has branches we know semantically cannot occur; but to appease the type system, we need to put an expression in those branches of the appropriate type.
%
Instead of introducing an \ms{abort} construct, which would complicate our semantics, we generate an expression of the appropriate type using a type-indexed dummy-value function $\dummy{} : A \to \Ch{A}$.
%
\[
\begin{array}{lcl}
  \dummy\; (\m{x} : \tbool) &=& \efalse\\
  \dummy\; (\m{x} : \Seteq{A}) &=& \unit\\
  \dummy\; \pair{\m{x}}{\m{y}} &=& \pair{\dummy\;\m{x}}{\dummy\;\m{y}}\\
  \dummy\; (\inj{i}{\m{x}}) &=& \inj{i}{(\dummy\;\m{x})}\\
  \dummy\; \disc{\d{x}} &=& \disc{\dummy\; \d{x}}\\
  \dummy\; (\m{f} : A \to B) &=&
    \fn \disc{\d{x}} \, \m{dx} \binder\, \dummy\;(\m{f}\;\d{x})
\end{array}
\]

This is mostly a hack to simplify our metatheory; however, see Section \ref{sec:change-operators-at-decidable-semilattice-types} for another use of $\ms{dummy}$.


\subsubsection{Useless definition time!}

It's \emph{almost} possible to give \dummy{} the type $A \to \Disc{\Ch A}$.
%
It requires an additional (semantically valid) axiom internalizing the fact that if $B$ is a discrete type, then so is $A \to B$:
%
\begin{eqnarray*}
  \ms{axiom} &:& (A \to \Disc{B}) \to \Disc(A \to \Disc{B})\\
  \ms{axiom}\;f &=& f \quad\text{(semantically speaking; does not typecheck)}
\end{eqnarray*}

With this axiom, we can define a ``shape'' function:
%
\begin{eqnarray*}
  \shape &:& A \to \Disc{A}\\
  \shape\; (\m{f} : A \to B) &=&
    \ms{axiom}\; (\fn\bind{\m{x}} \shape\; (\m{f}\;\m{x}))\\
  \shape\; \pair{\m{x}}{\m{y}} &=&
    \letdisc{\d{x'}}{\shape\;\m{x}}
    \letdisc{\d{y'}}{\shape\;\m{y}}
    \disc{\pair{\d{x'}}{\d{y'}}}\\
  \shape\; (\inj{i}{\m{x}}) &=&
    \letdisc{\d{x'}}{\shape\;\m x} \disc{\inj{i}{\d{x'}}}\\
  \shape\; \disc{\d{x}} &=& \disc{\disc{\d{x}}}\\
  \shape\; (\m{x} : \Seteq{A}) &=& \disc{\bot}\\
  \shape\; (\m{x} : 2) &=& \disc{\efalse}
\end{eqnarray*}

If we ignore monotonicity, $\dummy = \ms{shape} \circ \dummy = \dummy \circ \ms{shape}$. (I think; I haven't proved this.)
%
Note also that we only require \ms{axiom} in the function case; a restricted version of \ms{shape} can be defined with the type $\eq{A} \to \Disc{\eq{A}}$.


\subsubsection{Computing change operators at decidable semilattice types}
\label{sec:change-operators-at-decidable-semilattice-types}

The definition of $\delta$ given above makes use of the change operators $\zero$, $\oplus$, and $\ominus$ in the cases for $\bigvee$ and $\mb{when}$.
%
But as mentioned in Section \ref{sec:change-types}, in general these operators do not have definitions which Datafun is capable of type-checking!
%
One solution would be to add these operators to Datafun as primitives.
%
This is undesirable, for several reasons:
\begin{itemize}
\item $\zero$ and $\ominus$ are not efficiently computable in general, since
  they find the derivative of functions by runtime calculation rather than
  static differentiation.

\item $x \ominus y$ is only meaningful if $x \ge y$; Datafun's type system
  can't enforce this constraint.

\item We'd need to define $\delta(e \oplus f)$, $\delta(\zero\;e)$, and
  $\delta(e \ominus f)$. This is probably possible; but if we don't need it, why
  bother?
\end{itemize}

Fortunately, our definition of $\delta$ only uses these operators at \emph{decidable semilattice types}.\footnote{\todo{TODO: put a call-back to the $\eta$-expansion trick here}} Moreover, it only uses $\ominus$ in the restricted idiom $e \ominus \unit$. We can implement these cases in core Datafun as follows:
%
\begin{eqnarray*}
  \zero\; e &=& \dummy\; e\\
  e \oplus f &=& e \vee f\\
  e \ominus \unit &=& e
\end{eqnarray*}

\todo{TODO: Prove these definitions correct for the case of decidable semilattice types.}


\subsubsection{Derivatives of fixed points}

Our rule for the derivative of a fixed point is:
\begin{eqnarray*}
  \delta(\fix{\m{x}} e)
  &=& \letdisc{\d{x}}{\disc{\fix{\m{x}} e}} \fix{\m{dx}} \delta e\\
  &=& \fix{\m{dx}}
      (\fn \disc{\d{x}}\, \m{dx}\binder\, \delta e)
      \; \disc{\fix{\m{x}} e}
      \; \m{dx}\\
  &=& \fix{\m{dx}} \delta(\fn\bind{\m{x}} e)
      \; \disc{\fix{\m{x}} e}
      \; \m{dx}
\end{eqnarray*}

Why is this correct? First, let's rephrase this in terms of an operator
$\ms{fix} : (\fixtype{A} \to \fixtype{A}) \to \fixtype{A}$:
\begin{equation}
  \delta(\ms{fix}\; f) = \ms{fix}\; (\delta f \;\disc{\ms{fix}\; \preserve{f}})
\end{equation}

This is such a beautiful equation it can't be wrong. How did we arrive at this
equation? Well,
\[
\begin{array}{rcl>{\hspace{1em}}l}
  \delta(\ms{fix}\; f)
  &=& \delta(f \;(\ms{fix}\;f))
  & \text{because}~\ms{fix}\; f = f\;(\ms{fix}\;f)\\
  &=& \delta f \;\disc{\ms{fix}\; \preserve{f}} \; \delta(\ms{fix}\; f)
  & \text{rule for}~\delta(e_1\;e_2)
\end{array}
\]

We've now found a recursive equation that describes
${\color{Cyan}\delta(\ms{fix}\;f)}$ in terms of itself:
\begin{equation}\label{eqn:delta-fix}
  {\color{Cyan} \delta(\ms{fix}\;f)}
  = \delta f\;\disc{\ms{fix}\; \preserve{f}}\;{\color{Cyan}\delta(\ms{fix}\;f)}
\end{equation}

Let's apply \ms{fix} to solve the equation!
\begin{equation}
  \delta(\ms{fix}\; f)
  =
  \ms{fix}\,(\fn\bind{\m{\color{Cyan} dx}}
  \delta f \;\disc{\ms{fix}\;\preserve{f}} \;{\m{\color{Cyan} dx}})
\end{equation}

However, this is not a proof. We have merely established it is \emph{true} that equation \ref{eqn:delta-fix} holds, not that it suffices as a \emph{definition} of $\delta(\ms{fix}\;f)$. The correctness criterion for $\delta(\ms{fix}\; f)$ is:

\begin{equation}
  \ms{fix}\;f \oplus \delta(\ms{fix}\;f)
  = (\ms{fix}\;f)\sub{\gamma \oplus \delta\gamma/\gamma}
\end{equation}

That is, a $\ms{fix}$ expression plus its derivative equals the original
expression computed with updated free variables. We give a proof of this for our
definition of $\delta(\ms{fix}\;f)$ in \url{fixderiv.pdf}, which is also
available \href{http://www.rntz.net/files/fixderiv.pdf}{online}. This proof is
quite generic, and not specific to Datafun; as such, it assumes a few lemmas,
which we prove here:

\par \todo{TODO: insert proofs of the lemmas from \texttt{fixderiv} here}

%% \begin{lemma}
%%   $x \le x \oplus dx$
%%   \label{lemma:oplus-increasing}
%% \end{lemma}
%% \begin{proof}
%%   \todo{TODO.}
%% \end{proof}

%% \begin{lemma} For $f : A \to B$, $x : A$, $df : \Ch (A \to B)$, and $x :
%%   \Ch A$, where $df$ and $dx$ are valid changes to $f$,$x$ respectively (see
%%   section on ``Dependent change types'', below), we have:
%%   \[(f \oplus df) \; (x \oplus dx) = f\;x \oplus df\;x\;dx\]
%% \end{lemma}
%% \begin{proof}
%%   \todo{TODO. See figure 4, page 5, in Cai et al.}
%% \end{proof}

%% I don't know whether we need this theorem.

%% \begin{theorem}[Soundness of ordering on change types]
%%   For any $dx, dx' : \Ch A$, if $dx \le dx'$, then for any $x : A$, we have
%%   $x \oplus dx \le x \oplus dx'$.
%% \end{theorem}
%% \begin{proof}
%%   \todo{TODO. This seems like it should be true. Note that we don't \emph{need}
%%     this to be true for all types, just for fixed-point types $\fixtype{A}$, but
%%     it would be surprising if it weren't true generally.}
%% \end{proof}


\section{Finding fixed points faster}
%% \section{Finding fixed points faster with $\delta$}

\[\begin{array}{l}
  \textsf{fast-fix} ~:~
  (\Disc{\fixtype{A}} \to \Ch\fixtype{A} \to \Ch\fixtype{A})
  \to \fixtype{A} \to \Ch\fixtype{A} \to \fixtype{A}\\
  \textsf{fast-fix}\; \mi{df}\; \mi{current}\; \mi{change} =\\
  \quad \textbf{let}~ {\mi{next} = \mi{current} \oplus \mi{change}}
  ~\textbf{in}\\
  \quad \textbf{if}~ \mi{next} \le \mi{current}
  ~\textbf{then}~ \mi{current}\\
  \quad \textbf{else}~ \textsf{fast-fix}\; \mi{df}\; \mi{next}
  \;(\mi{df}\; \mi{current}\; \mi{change})\\
  \\
  \textsf{fast-fix}
  \in ((x : \Disc{L}) \to \Delta(L, x) \to \Delta(L, x)) \to L
  \\
  \textsf{fast-fix} \; df =\\
  \quad\textbf{let}~ \textsf{loop} \in (x : L) \to \Delta(L, x) \to L\\
  \quad\phantom{\textbf{let}~} \textsf{loop} \;x \;dx =\\
  \quad\phantom{\textbf{let}~}\quad
  \textbf{let}~ x' = x \oplus dx ~\textbf{in}\\
  \quad\phantom{\textbf{let}~}\quad
  \textbf{if}~ x' \le x ~\textbf{then}~ x\\
  \quad\phantom{\textbf{let}~}\quad
  \textbf{else}~\textsf{loop} \;x' \;(df \; x \; dx)\\
  \quad\textbf{in}~ \textsf{loop} \; \unit \; (f \; \unit \ominus \unit)
\end{array}\]


\section{Dependent change types}

We will define the following type-indexed sets and operators:

\[\begin{array}{ccccl}
  \Change{A}{a} &\subseteq& |\Den{\Ch{A}}|
  &\text{for}& a \in \Den{A}
  \\
  a \oplus_A da &\in& \Den{A}
  &\text{for}& a \in \Den{A} ~\text{and}~ da \in \Change{A}{a}
  \\
  b \ominus_A a &\in& \Change{A}{a}
  &\text{for}& a, b \in \Den{A} ~\text{and}~ a \le b
  \\
  \zero_A\; a &\in& \Change{A}{a}
  &\text{for}& a \in \Den{A}
\end{array}\]

\zero{} is simply syntax sugar; $\zero_A\; x = x \ominus_A x$. The other three
are defined mutually, by induction on types. First we give the rules for
$\Change{A}{a}$:

\[
\def\arraystretch{1.2}
\begin{array}{rcl}
  \Change{\tbool}{\pwild} &=& \Den{\tbool}
  \\
  \Change{\Seteq{A}}{\pwild} &=& \Den{\Seteq{A}}
  \\
  \Change{A_1 + A_2}{\inj{i}{x}}
  &=& \setfor{\inj{i}{dx}}{dx \in \Change{A_i}{dx}}
  %% \\
  %% \inj{i}{dx} \in \Change{A_1 + A_2}{\inj{i}{x}}
  %% &\iff& dx \in \Change{A_i}{x}
  \\
  \Change{A \x B}{\pair{a}{b}}
  &=& \Change{A}{a} \x \Change{B}{b}
  \\
  \Change{\Disc{A}}{a}
  &=& \setfor{dx \in \Change{A}{a}}{x \oplus_A dx = x}
  \\
  df \in \Change{A \to B}{f}
  &\Leftrightarrow&
  \forall({a,b \in \Den{A}},\, da \in \Change{A}{a},\, db \in \Change{A}{b})\\
  &&\phantom{{}\wedge{}}
  df\; a\; da \in \Change{B}{f\; a}
  \wedge df\;b\;db \in \Change{B}{f\; b}
  %% \\
  %% &&{}\wedge (a \mapsto f\; a \oplus_B df\; a \; (\zero_A\; a))
  %% ~\text{is monotone}
  \\
  &&{}\wedge (a \oplus_A da \le b \oplus_A db \implies
  f\;a \oplus_B df\;a\;da \le f\;b \oplus_B df\;b\;db)
\end{array}
\]

Next we define the operators:
\begin{center}
  \scalebox{1}{\(
  \begin{array}{cccc}
    \textbf{Type} & \oplus & \ominus %% & \zero
    \\ \midrule
    \tbool
    & a \oplus da = a \vee da
    & b \ominus a = b
    %% & \zero\; a = \ms{false}
    \\
    \Seteq{A}
    & a \oplus da = a \cup da
    %% should this be: b \ominus a = b \setminus a?
    & b \ominus a = \todo{b}
    %% & \zero\; a = \emptyset
    \\
    A + B
    & \inj{i}{a} \oplus \inj{i}{da} = \inj{i}{a \oplus da}
    & \inj{i}{b} \ominus \inj{i}{a} = \inj{i}{b \ominus a}
    %% & \zero\; (\inj{i}{a}) = \inj{i}{\zero\; a}
    \\
    A \x B
    & \pair{a}{b} \oplus \pair{da}{db} = \pair{a \oplus da}{b \oplus db}
    & \pair{b}{y} \ominus \pair{a}{x} = \pair{b \ominus a}{y \ominus x}
    %% & \zero\;\pair{a}{b} = \pair{\zero\; a}{\zero\; b}
    \\
    \Disc{A}
    & a \oplus da = a
    & b \ominus_{\Disc{A}} a = \zero_{A}\; a
    %% & \zero_{\Disc{A}}\; a = \zero_{A}\; a
    \\
    A \to B
    & (f \oplus df)\; x = f\; x \oplus df\; x \; (\zero\; x)
    & (g \ominus f)\; x\; dx = g\; (x \oplus dx) \ominus f\; x
    %% & (\zero\; f)\; x\; dx = f\; (x \oplus dx) \ominus f\; x
  \end{array}
  \)}
\end{center}

Although we define $\Change{A}{a}$ as a subset of the elements $\Den{\Ch{A}}$,
we consider it to be a poset with the induced ordering.

The last line in the definition of $df \in \Change{A \to B}{f}$ is equivalent to
saying the map $(a, da) \mapsto (f\;a, df\;a\;da)$ is monotone when we order its
domain and codomain by $(a,da) \le (b,db) \iff a \oplus da \le b \oplus db$.
%
In Cai et al., figure 4, page 5, the corresponding condition is
$(f \oplus_{A \to B} df)\; (a \oplus_A da) = f\;a \oplus_B df\;a\;da$.
%
Our condition is stronger, to account for monotonicity, and implies theirs as follows:
%
\[\begin{array}{crcll}
  & a \oplus da &=& (a \oplus da) \oplus \zero_A \;(a \oplus da)
  & \text{by Lemma \ref{lemma:zero-id}}\\
  \implies& f\;a \oplus df\;a\;da &=&
  f\;(a \oplus da) \oplus df \;(a \oplus da) \;(\zero_A \;(a \oplus da))
  & \text{our condition, reflexivity and antisymmetry}\\
  \implies& f\;a \oplus df\;a\;da &=& (f \oplus df) \; (a \oplus da)
  & \text{definition of $f \oplus_{A \to B} df$}
\end{array}\]

We will show the following:

\begin{enumerate}
\item If $a \le b$ then $a \oplus_A (b \ominus_A a) = b$. From this it
  follows that $a \oplus_A \zero_A\; a = a$.

\item \todo{(do we use this?)} $a \oplus_A da$ is monotone in $da$. That is: for
  any $a \in \Den{A}$ and $da, db \in \Change{A}{a}$, if $da \le db$ in
  $\Den{\Ch{A}}$, then $a \oplus_A da \le a \oplus_A db$ in $\Den{A}$.

\item \todo{(do we use this?)} $b \ominus_A a$ is monotone in $b$.
\end{enumerate}

%% NB. One might think we could also say something like: ``$\zero_A\; a$ is the
%% least element of $\Change{A}{a}$''. This might be true for $\zero$ as defined
%% (although I'm not sure it is), but it is not true of zero changes in general:
%% $\{5\}$ is a zero-change on $\{5\}$, but not the least element of
%% $\Change{\Set{{\N}}}{\{5\}}$; the empty set (also a zero-change) is smaller.


\begin{lemma}
  \label{lemma:oplus-ominus}
  \label{lemma:zero-id}
  If $a \le b$ then $a \oplus_A (b \ominus_A a) = b$.
\end{lemma}

\begin{proof}
  By induction on $A$:
  \begin{description}
    \item[Case $\tbool$]
      \begin{equation*}
      a \oplus_\tbool (b \ominus_\tbool a) = a \vee b = b
      \end{equation*}
      Where $a \vee b = b$ follows from $a \le b$.

    \item[Case $\Seteq{A}$]
      \begin{equation*}
        a \oplus_{\Seteq{A}} (b \ominus_{\Seteq{A}} a)
        = a \cup b
        = b
      \end{equation*}
      Where $a \cup b = b$ follows from $a \le b$, i.e. $a \subseteq b$.

    \item[Case $A_1 + A_2$]
      \begin{equation*}
        \inj{i}{a} \oplus (\inj{i}{b} \ominus \inj{i}{a})
        = \inj{i}{(a \oplus_{A_i} (b \ominus_{A_i} a))}
        = \inj{i}{b}  \quad\text{(by IH)}
      \end{equation*}

      We know the subscript $i$ is identical for $\inj{i}{a}$ and $\inj{i}{b}$
      because $\inj{i}{a} \le \inj{j}{b} \implies i = j$.

    \item[Case $A \x B$]
      \begin{equation*}
        \pair{a}{x} \oplus (\pair{b}{y} \ominus \pair{a}{x})
        = \pair{a \oplus (b \ominus a)}{x \oplus (y \ominus x)}
        = \pair{b}{y} \quad\text{(by IH)}
      \end{equation*}

    \item[Case $\Disc{A}$]
      \begin{equation*}
        a \oplus (b \ominus a) = a = b
      \end{equation*}
      Where $a = b$ follows from $a \le b$ and discreteness.


    \item[Case $A \to B$]
      \[\begin{array}{rcll}
      (f \oplus (g \ominus f))\; x
      &=& f\;x \oplus (g \ominus f)\; x\; (\zero\; x)\\
      &=& f\;x \oplus (g \; (x \oplus \zero\; x) \ominus f \; x)\\
      &=& g\; (x \oplus \zero\; x) & \text{by IH at $B$}\\
      &=& g\; x & \text{by IH at $A$}
      \end{array}\]

  \end{description}
\end{proof}


\section{Metatheory}

\begin{theorem}[Denotation]
  \label{thm:denotation}
  If $\J{\GP}{\GG}{e}{A}$, then $\den{e} \in \Den{\Disc{\GP} \x \GG \to A}$.
\end{theorem}
\begin{proof}
  \todo{Assumed, for now.}
\end{proof}

\newcommand{\dendelta}[1]{\den{#1}_\delta}

If $\J{\GP}{\GG}{e}{A}$, then $\J{\GP, \GG, \Ch\GP}{\Ch\GG}{\delta e}{\Ch A}$.
%
So by Theorem \ref{thm:denotation} we have
$\den{\delta e} \in \Den{\Disc{(\GP \x \GG \x \Ch\GP)} \x \Ch\GG \to \Ch A}$.
%
We define $\dendelta{e}$ to be a reassociated version of
$\den{\delta e}$ with a type that looks like the change type of $\den{e}$:
%
\begin{eqnarray*}
  \dendelta{e} &\in& \Den{\Ch (\Disc{\GP} \x \GG \to A) }\\
  \dendelta{e} &\in& \Disc{(\Den{\GP} \x \Den{\GG})}
    \to \Disc{\Den{\Ch\GP}} \x \Den{\Ch\GG}  \to \Den{A}\\
  \dendelta{e} \; (\psi, \gamma) \; (d\psi, d\gamma)
  &\defeq& \den{\delta e} \; ((\psi, \gamma, d\psi), d\gamma)
\end{eqnarray*}
%
Now we prove that $\dendelta{e}$ is indeed a change to $\den{e}$:

\begin{theorem}[Legitimacy]
  Given $\J{\GP}{\GG}{e}{A}$ we have
  $\dendelta{e} \in \Change{\Disc{\Psi} \x \GG \to A}{\den{e}}$.
\end{theorem}

In fact, $\dendelta{e}$ is a zero change to $\den{e}$, that is, a derivative:

\begin{theorem}[Correctness]
  Given $\J{\GP}{\GG}{e}{A}$, $s \in \Den{\Disc{\GP} \x \GG}$,
  and $ds \in \Change{\Disc{\GP} \x \GG}{s}$, we have
  \begin{equation*}
    \den{e}\; (s \oplus ds)
    =
    \den{e}\; s \oplus \dendelta{e} \; s \; ds
  \end{equation*}

  %% Given $\J{\GP}{\GG}{e}{A}$,
  %% $\psi \in \Den{\Disc{\GP}}$, $\gamma \in \Den{\GG}$,
  %% $\delta\psi \in \Change{\Disc{\Psi}}{\psi}$,
  %% $\delta\gamma \in \Change{\GG}{\gamma}$,
  %% we have
  %% \begin{equation*}
  %%   \den{e}\; (\psi \oplus \delta \psi, \gamma \oplus \delta\gamma)
  %%   =
  %%   \den{e}\; (\psi, \gamma) \oplus
  %%   \den{\delta e}\;(\psi, \gamma, \delta \psi, \delta\gamma)
  %% \end{equation*}

\end{theorem}

Strategy: attack each by induction on $\J{\GP}{\GG}{e}{A}$. Try legitimacy
first.

NB. In order for that last $\oplus$ to be well-defined, we need legitimacy to
hold.


\section{Extending Datafun}

\todo{NB. To add a type to Datafun, we must extend \dummy{} appropriately.}

\todo{We should carefully distinguish which properties are necessary to safely
  extend core Datafun --- and therefore, which proofs are ``open''. For example,
  all that should be necessary to add a type to Datafun and declare it decidable
  is that it should actually \emph{be} decidable. However, for \emph{decidable
    semilattice types}, we require that Lemma \ref{lemma:dl-boring} holds in
  order to extend the language with them.

  This means that when proving something about decidable types, we cannot do it
  by induction (or we violate extensibility), only by invoking the fact that
  they are decidable. But when proving something about decidable semilattice
  types, we can use Lemma \ref{lemma:dl-boring}. The proof of Lemma
  \ref{lemma:dl-boring} gets to use induction, but the hypothesis can't be
  strengthened (or we'd violate extensibility).}


\section{Stuff to put in this document}

\begin{enumerate}
\item Correctness criterion for $\delta$ (see google doc).
\item Lemma: for decidable semilattices, $\den{e} \oplus \den{f} = \den{e \vee
  f}$.
\item Lemma justifying $\delta (\unit : A \to L) = \unit : \Disc{A} \to
  \Delta{A} \to \Delta{L}$ at functional types.
\item Definition of $\Delta_A : \Den{A} \to \ms{Poset}$.

  or $\Delta(A, a) \in \ms{Poset}$ for $a \in \Den{A}$
\item Show $\Delta_A(a)$ is an induced subposet of $\Den{A}$.

\item Define $\oplus_A : (a : \Disc{\Den{A}}) \to \Change{A}{a} \to \Den{A}$
  and show it is monotone in its second argument.
  This also involves defining \zero.

\end{enumerate}

\end{document}
