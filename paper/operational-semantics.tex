\begin{figure*}
  \centering

  %% OPSEM: Additional syntax definitions
  \begin{minipage}{0.5\textwidth}
    \centering
    %% TODO: format this nicer
    \textbf{Additional syntax}
    \begin{displaymath}
      \begin{array}{ccl}
        %% expressions
        e,f,g
        &\bnfeq& ... \pipe \setlit{\vec{v}} \pipe \unit_L \pipe e \vee_L e \pipe \tforin{L}{x \in e} e\\
        \textsf{expressions}
        && \tfix{\fineq{L}}{x}{e} \pipe \tfixle{\eq{L}}{x}{e}{e}\\
        && \iter{\eq{A}}{e}{x}{e} \pipe \iterstep{\eq{A}}{e}{e}{x}{e}\\
        && \iterle{\eq{A}}{e}{e}{x}{e} \pipe \iterlestep{\eq{A}}{e}{e}{e}{x}{e}
        \vspace{0.5em}\\
        %% values
        v,u,w
        &\bnfeq& \fn\bind{x} e \pipe (v, v) \pipe \ms{in}_i\; v
        \pipe \ms{true} \pipe \ms{false} \pipe \setlit{\vec{v}}\\
        \textsf{values}
        \vspace{0.5em}\\
        %% contexts
        E
        &\bnfeq& \hole \pipe E\;e \pipe v\;E \pipe (E, e) \pipe (v, E) \pipe \ms{in}_i\;E
        \pipe \pi_i \; E\\
        \textsf{evaluation}
        && E \vee_L e \pipe v \vee_L E \pipe \tforin{L}{x \in E} e\\
        \textsf{contexts}
        && \ifthen{E}{e}{e}\\
        && \case{E}{x}{e}{x}{e}\\
        && \iter{\eq{A}}{E}{x}{e} \pipe \iterstep{\eq{A}}{v}{E}{x}{e}\\
        && \iterle{\eq{A}}{E}{e}{x}{e} \pipe \iterle{\eq{A}}{v}{E}{x}{e}\\
        && \iterlestep{\eq{A}}{v}{v}{E}{x}{e}
      \end{array}
    \end{displaymath}
  \end{minipage}
  ~
  %% OPSEM: Rules for (v \ale u : A)
  \begin{minipage}{0.45\textwidth}
    \centering
    \textbf{Rules for $v \ale u : \eq{A}$ and $v \aeq u : \eq{A}$}
    \begin{mathpar}
      \infer{\ms{false} \ale \ms{false} : \bool}{}
      \and
      \infer{\ms{false} \ale \ms{true} : \bool}{}
      \and
      \infer{\ms{true} \ale \ms{true} : \bool}{}
      \and
      %% rules for set inequality
      \infer[\rn{\subseteq}]
            { \setlit{\vec{v_i}} \ale \setlit{\vec{u_i}} : \Set{\eq{A}} }
            { \forall{v_i}\,\exists{u_j}\; (v_i \aeq u_j : \eq{A}) }
            \and
            \infer%% [\rn{\ale_{\x}}]
                { (v_1, u_1) \ale (v_2, u_2) : \eq{A} \x \eq{B} }
                { v_1 \ale v_2 : \eq{A} & u_1 \ale u_2 : \eq{B} }
                \and
                \infer%% [\rn{\ale_{+}}]
                    { \ms{in}_i\; v \ale \ms{in}_i\; u : \eq{A}_1 + \eq{A}_2 }
                    { v \ale u : \eq{A}_i }
                    \and
                    \infer[\rn{{\aeq}}]{v \aeq u : \eq{A}}{v \ale u : \eq{A} & u \ale v : \eq{A}}
    \end{mathpar}
  \end{minipage}
  %% OPSEM: Rules for \step
  \vspace{1em}
  \begin{displaymath}
    \begin{array}{ccl}
      \multicolumn{3}{c}{\textbf{$\beta$-reductions}}\\
      (\fn\bind{x}e_1) \; e_2 &\step& \sub{e_2/x} e_1\\
      \pi_i \; (v_1, v_2) &\step& v_i\\
      \rawcase{\ms{in}_i\,v}{\widevec{\ms{in_j}\,x_j \cto e_j}}
      &\step& \sub{v/x_i} e_i\\
      \ifthen{\ms{true}}{e_1}{e_2} &\step& e_1\\
      \ifthen{\ms{false}}{e_1}{e_2} &\step& e_2

      %% rules for unit
      \vspace{1em}\\
      \multicolumn{3}{c}{\textbf{Evaluating }\unit}\\
      \unit_2 &\step& \ms{false}\\
      \unit_{\Set{A}} &\step& \{\}\\
      \unit_{L \x M} &\step& (\unit_L, \unit_M)\\
      \unit_{A \to L} &\step& \fn\bind{x} \unit_L\\
      \unit_{A \mto L} &\step& \fn\bind{x} \unit_L

      %% rules for \vee
      \vspace{1em}\\
      \multicolumn{3}{c}{\textbf{Evaluating }\vee}\\
      \ms{false} \vee_2 v &\step& v\\
      \ms{true} \vee_2 v &\step& \ms{true}\\
      %% the rule we've all been waiting for
      \setlit{\vec{v}} \vee_{\Set{A}} \setlit{\vec{u}} &\step& \setlit{\vec{v}, \vec{u}}\\
      (v_1, v_2) \vee_{L \x M} (u_1, u_2) &\step& (v_1 \vee_L u_1, v_2 \vee_M u_2)\\
      v \vee_{A \to L} u &\step& \fn\bind{x} v\;x \vee_L u\;x\\
      v \vee_{A \mto L} u &\step& \fn\bind{x} v\;x \vee_L u\;x

    \end{array}
    \begin{array}{ccl}

      %% rules for \bigvee
      %% \vspace{1em}\\
      \multicolumn{3}{c}{\textbf{Evaluating }\bigvee}\\
      \tforin{L}{x \in \{\}} e &\step& \unit_L\\
      \tforin{L}{x \in \setlit{v, \vec{u}}} e
      &\step& \sub{v/x} e \vee_L \tforin{L}{x \in \setlit{\vec{u}}} e

      %% rules for \ms{fix}
      \vspace{1em}\\
      \multicolumn{3}{c}{\textbf{Evaluating \ms{fix} and \ms{iter}}}\\
      \tfix{\fineq{L}}{x}{e} &\step& \iter{\fineq{L}}{\unit_{\fineq{L}}}{x}{e}\\
      \iter{\eq{A}}{v}{x}{e} &\step& \iterstep{\eq{A}}{v}{\sub{v/x} e}{x}{e}\\
      \iterstep{\eq{A}}{v_1}{v_2}{x}{e}
      &\step& \begin{cases}
        v_1 & \text{if}~{v_1 \aeq v_2 : \eq{A}}\\
        \iter{\eq{A}}{v_2}{x}{e} & \text{otherwise}
      \end{cases}\\
      %% rules for fixle, iterle
      \tfixle{\eq{L}}{x}{e_\top}{e} &\step& \iterle{\eq{L}}{e_\top}{\unit_{\eq{L}}}{x}{e}\\
      \iterle{\eq{A}}{v_\top}{v}{x}{e}
      &\step& \begin{cases}
        \iterlestep{\eq{A}}{v_\top}{v}{\sub{v/x} e}{x}{e} & \text{if}~{v \ale v_\top : \eq{A}}\\
        v_\top & \text{otherwise}
      \end{cases}\\
      \iterlestep{\eq{A}}{v_\top}{v_1}{v_2}{x}{e}
      &\step& \begin{cases}
        v_1 &\text{if}~{v_1 \aeq v_2 : \eq{A}}\\
        \iterle{\eq{A}}{v_\top}{v_2}{x}{e} & \text{otherwise}
      \end{cases}
    \end{array}
  \end{displaymath}

  \caption{Operational semantics}
  \label{fig:opsem}
\end{figure*}


\section{A na\"ive operational semantics}

We consider the denotational semantics to be primary in Datafun; as with
Datalog, any implementation technique is valid so long as it lines up with these
semantics. As a proof of concept, however, we present a simple call-by-value
structural operational semantics in Figure \ref{fig:opsem} and show that all
well-typed terms terminate.

In our operational semantics we:
\begin{enumerate}
\item Assume an elaboration step which subscripts all semilattice operations
  ($\unit$, $\vee$, $\bigvee$, and $\ms{fix}$) with their type.
\item Drop the distinction between discrete and monotone variables, and write
  $x,y$ for arbitrary variables.
\item Ignore the types $\N$ and $\str$. \todo{TODO: justify?}
\item Add \ms{iter} expressions, which occur as intermediate forms in the
  evaluation of \ms{fix}.
\item Classify some expressions $e$ as values $v$, and add a value-form
  $\setlit{\vec{v}}$ for finite sets.
\end{enumerate}

We use a small-step operational semantics using \emph{evaluation contexts} $E$
after the style of \citet{reduction-contexts} to enforce a call-by-value
evaluation order; an evaluation context $E$ is an expression with a hole in it
(written $\hole{}$) such that whatever is in the hole is next in line to be
evaluated (if it is not a value already). To fill the hole $\hole$ in an
evaluation context $E$ with the expression $e$, we write $E[e]$.

We define a relation $e \step e'$ for expressions $e$ whose outermost structure
is immediately reducible; we extend this relation to all expressions with the
rule
\[\infer{E[e] \step E[e']}{e \step e'}\]

%% FIXME
In our rules for $e \step e'$ we make use of a \emph{decidable inequality test
  on values}, $v \ale u : \eq{A}$.
