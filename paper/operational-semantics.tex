\begin{figure*}
  \centering

  %% OPSEM: Additional syntax definitions
  \begin{minipage}{0.5\textwidth}
    \centering
    %% TODO: format this nicer
    \textbf{Additional syntax}
    \begin{displaymath}
      \begin{array}{ccl}
        %% expressions
        e,f,g
        &\bnfeq& ... \pipe \setlit{\vec{v}} \pipe \unit_L \pipe e \vee_L e \pipe \tforin{L}{x \in e} e\\
        \textsf{expressions}
        && \tfix{\fineq{L}}{x}{e} \pipe \tfixle{\eq{L}}{x}{e}{e}\\
        && \iter{\eq{A}}{e}{x}{e} \pipe \iterstep{\eq{A}}{e}{e}{x}{e}\\
        && \iterle{\eq{A}}{e}{e}{x}{e} \pipe \iterlestep{\eq{A}}{e}{e}{e}{x}{e}
        \vspace{0.5em}\\
        %% values
        v,u,w
        &\bnfeq& \fn\bind{x} e \pipe (v, v) \pipe \ms{in}_i\; v
        \pipe \ms{true} \pipe \ms{false} \pipe \setlit{\vec{v}}\\
        \textsf{values}
        \vspace{0.5em}\\
        %% contexts
        E
        &\bnfeq& \hole \pipe E\;e \pipe v\;E \pipe (E, e) \pipe (v, E) \pipe \ms{in}_i\;E
        \pipe \pi_i \; E\\
        \textsf{evaluation}
        && E \vee_L e \pipe v \vee_L E \pipe \tforin{L}{x \in E} e\\
        \textsf{contexts}
        && \ifthen{E}{e}{e}\\
        && \case{E}{x}{e}{x}{e}\\
        && \iter{\eq{A}}{E}{x}{e} \pipe \iterstep{\eq{A}}{v}{E}{x}{e}\\
        && \iterle{\eq{A}}{E}{e}{x}{e} \pipe \iterle{\eq{A}}{v}{E}{x}{e}\\
        && \iterlestep{\eq{A}}{v}{v}{E}{x}{e}
      \end{array}
    \end{displaymath}
  \end{minipage}
  ~
  %% OPSEM: Rules for (v \ale u : A)
  \begin{minipage}{0.45\textwidth}
    \centering
    \textbf{Rules for $v \ale u : \eq{A}$ and $v \aeq u : \eq{A}$}
    \begin{mathpar}
      \infer{\ms{false} \ale \ms{false} : \bool}{}
      \and
      \infer{\ms{false} \ale \ms{true} : \bool}{}
      \and
      \infer{\ms{true} \ale \ms{true} : \bool}{}
      \and
      %% rules for set inequality
      \infer[\rn{\subseteq}]
            { \setlit{\vec{v_i}} \ale \setlit{\vec{u_i}} : \Set{\eq{A}} }
            { \forall{v_i}\,\exists{u_j}\; (v_i \aeq u_j : \eq{A}) }
            \and
            \infer%% [\rn{\ale_{\x}}]
                { (v_1, u_1) \ale (v_2, u_2) : \eq{A} \x \eq{B} }
                { v_1 \ale v_2 : \eq{A} & u_1 \ale u_2 : \eq{B} }
                \and
                \infer%% [\rn{\ale_{+}}]
                    { \ms{in}_i\; v \ale \ms{in}_i\; u : \eq{A}_1 + \eq{A}_2 }
                    { v \ale u : \eq{A}_i }
                    \and
                    \infer[\rn{{\aeq}}]{v \aeq u : \eq{A}}{v \ale u : \eq{A} & u \ale v : \eq{A}}
    \end{mathpar}
  \end{minipage}
  %% OPSEM: Rules for \step
  \vspace{1em}
  \begin{displaymath}
    \begin{array}{ccl}
      \multicolumn{3}{c}{\textbf{$\beta$-reductions}}\\
      (\fn\bind{x}e_1) \; e_2 &\step& \sub{e_2/x} e_1\\
      \pi_i \; (v_1, v_2) &\step& v_i\\
      \rawcase{\ms{in}_i\,v}{\widevec{\ms{in_j}\,x_j \cto e_j}}
      &\step& \sub{v/x_i} e_i\\
      \ifthen{\ms{true}}{e_1}{e_2} &\step& e_1\\
      \ifthen{\ms{false}}{e_1}{e_2} &\step& e_2

      %% rules for unit
      \vspace{1em}\\
      \multicolumn{3}{c}{\textbf{Evaluating }\unit}\\
      \unit_2 &\step& \ms{false}\\
      \unit_{\Set{A}} &\step& \{\}\\
      \unit_{L \x M} &\step& (\unit_L, \unit_M)\\
      \unit_{A \to L} &\step& \fn\bind{x} \unit_L\\
      \unit_{A \mto L} &\step& \fn\bind{x} \unit_L

      %% rules for \vee
      \vspace{1em}\\
      \multicolumn{3}{c}{\textbf{Evaluating }\vee}\\
      \ms{false} \vee_2 v &\step& v\\
      \ms{true} \vee_2 v &\step& \ms{true}\\
      %% the rule we've all been waiting for
      \setlit{\vec{v}} \vee_{\Set{A}} \setlit{\vec{u}} &\step& \setlit{\vec{v}, \vec{u}}\\
      (v_1, v_2) \vee_{L \x M} (u_1, u_2) &\step& (v_1 \vee_L u_1, v_2 \vee_M u_2)\\
      v \vee_{A \to L} u &\step& \fn\bind{x} v\;x \vee_L u\;x\\
      v \vee_{A \mto L} u &\step& \fn\bind{x} v\;x \vee_L u\;x

    \end{array}
    \begin{array}{ccl}

      %% rules for \bigvee
      %% \vspace{1em}\\
      \multicolumn{3}{c}{\textbf{Evaluating }\bigvee}\\
      \tforin{L}{x \in \{\}} e &\step& \unit_L\\
      \tforin{L}{x \in \setlit{v, \vec{u}}} e
      &\step& \sub{v/x} e \vee_L \tforin{L}{x \in \setlit{\vec{u}}} e

      %% rules for \ms{fix}
      \vspace{1em}\\
      \multicolumn{3}{c}{\textbf{Evaluating \ms{fix} and \ms{iter}}}\\
      \tfix{\fineq{L}}{x}{e} &\step& \iter{\fineq{L}}{\unit_{\fineq{L}}}{x}{e}\\
      \iter{\eq{A}}{v}{x}{e} &\step& \iterstep{\eq{A}}{v}{\sub{v/x} e}{x}{e}\\
      \iterstep{\eq{A}}{v_1}{v_2}{x}{e}
      &\step& \begin{cases}
        v_1 & \text{if}~{v_1 \aeq v_2 : \eq{A}}\\
        \iter{\eq{A}}{v_2}{x}{e} & \text{otherwise}
      \end{cases}\\
      %% rules for fixle, iterle
      \tfixle{\eq{L}}{x}{e_\top}{e} &\step& \iterle{\eq{L}}{e_\top}{\unit_{\eq{L}}}{x}{e}\\
      \iterle{\eq{A}}{v_\top}{v}{x}{e}
      &\step& \begin{cases}
        \iterlestep{\eq{A}}{v_\top}{v}{\sub{v/x} e}{x}{e} & \text{if}~{v \ale v_\top : \eq{A}}\\
        v_\top & \text{otherwise}
      \end{cases}\\
      \iterlestep{\eq{A}}{v_\top}{v_1}{v_2}{x}{e}
      &\step& \begin{cases}
        v_1 &\text{if}~{v_1 \aeq v_2 : \eq{A}}\\
        \iterle{\eq{A}}{v_\top}{v_2}{x}{e} & \text{otherwise}
      \end{cases}
    \end{array}
  \end{displaymath}

  \caption{Operational semantics}
  \label{fig:opsem}
\end{figure*}


\section{An Operational Semantics}% {A na\"ive operational semantics}

We consider the denotational semantics to be primary in Datafun; as with
Datalog, any implementation technique is valid so long as it lines up with these
semantics. As a proof of concept, however, we present a simple call-by-value
structural operational semantics in Figure \ref{fig:opsem} and show that all
well-typed terms terminate.

In our operational semantics we:
\begin{enumerate}
\item Assume all semilattice operations ($\unit$, $\vee$, $\bigvee$, and
  $\ms{fix}$) are subscripted with their type.
\item Drop the distinction between discrete and monotone variables, and write
  $x,y$ for arbitrary variables.
\item Add \ms{iter} expressions, which occur as intermediate forms in the
  evaluation of \ms{fix}.
\item Classify some expressions $e$ as values $v$, and add a value-form
  $\setlit{\vec{v}}$ for finite sets.
\end{enumerate}

We use a small-step operational semantics using \emph{evaluation contexts} $E$
after the style of \citet{reduction-contexts} to enforce a call-by-value
evaluation order; an evaluation context $E$ is an expression with a hole in it
(written $\hole{}$) such that whatever is in the hole is next in line to be
evaluated (if it is not a value already). To fill the hole $\hole$ in an
evaluation context $E$ with the expression $e$, we write $E[e]$.

We define a relation $e \step e'$ for expressions $e$ whose outermost structure
is immediately reducible; we extend this relation to all expressions with the
rule
\[\infer{E[e] \step E[e']}{e \step e'}\]

In our rules for $e \step e'$ where $e$ is an \ms{iter} expression we make use
of a decidable ordering test on values, $v \ale u : \eq{A}$, and a corresponding
equality test $v \aeq u : \eq{A}$. We define these using inference rules, but
they are easily seen to be decidable by induction on $\eq{A}$. The quantifiers
in the premise of the rule $\rn{\subseteq}$ range over finite domains, and thus
pose no issue.


\subsection{Computing \ms{fix}-ed points via \ms{iter}-ation}

Our implementation strategy for $\fix{x}{e}$ is exactly that suggested by the
proof of Lemma \ref{lem:fixed-points-finite-height-posets}: starting from
$\unit$, iteratively apply $\fn\bind{x} e$ until a fixed point is
reached.

To model this iteration, we introduce $\ms{iter}$ forms into our
syntax.  The fixed point expression $\tfix{\fineq{L}}{x}{e}$
immediately steps to the form
$\iter{\fineq{L}}{\unit_{\fineq{L}}}{x}{e}$, which can be thought of
as an initial state of the iteration, starting with $\unit_{\fineq{L}}$.

The intuition for iterating to the fixed point is that we apply the 
body and then check to see if the result changed. This is why we also 
introduce the two-place version $\iterstep{\eq{A}}{v_1}{e_2}{x}{e}$ 
which remembers the old value $v_1$, so that we can test it against $e_2$ (when it
reaches a value $v_2$) to determine
whether we've reached a fixed point. If not, we can continue to iterate from
$v_2$ with $\iter{\fineq{L}}{v_2}{x}{e}$. 

The $\ms{iter}_\le$ forms are similar, but additionally check that the iteration
value never exceeds their first argument, to implement the clamping behavior of
$\fixle{x}{e_\top}{e}$.


\subsection{A logical relation for termination}

To prove that all well-typed terms terminate according to our operational
semantics, we use a logical relations argument.

The standard approach of interpreting each type as a partial equivalence
relation (PER) on closed terms turns out not to be sufficient in our case, and
we need to extend it to prove termination. Just as posets are sets equipped with
an order structure, we define our semantic types as PERs equipped with a
preorder respecting the PER structure. The intuition is that since we needed the
order structure in the denotational semantics to prove the definedness of fixed
points, we will similarly need an order structure on the syntax to prove the
termination of fixed points. Therefore, we inductively define relations
$\lr{A}{a \ale b}$ for each type $A$, then show how to understand these
relations as preordered PERs.

As a matter of notation, $a,b,c$ range over closed expressions; $\gamma, \sigma$
over substitutions containing only closed expressions; and $\ms{Ctx}(\GG)$ is
the set of all substitutions of closed expressions for the variables in $\GG$.

Because it simplifies our definitions and proofs, we introduce an additional
pseudo-type $\disc{A}$, which orders $A$ \emph{discretely}, $x \le y \iff x =
y$.
\[\begin{array}{rccl}
  \text{types} &
  A &\bnfeq& ... \pipe \disc{A}
\end{array}\]
This represents the $\Disc{\U{-}}$ comonad on \cPoset{} present in our
denotational semantics. Observe that under this interpretation $A \uto B$ is
just $\disc{A} \mto B$. In particular, if we let $\den{\disc{A}} =
\Disc{|\den{A}|}$ then $\den{A \uto B} = \den{\disc{A} \mto B} =
\Disc{|\den{A}|} \arr \den{B}$.

\newcommand{\tiff}{\mathrm{iff}}

We define the following relations:
\[\begin{array}{ccl}
  \lr{A}{a \ale b}  && \text{definition given below}\\
  \lr{\GG}{\gamma \ale \sigma}
  &\tiff& \forall(x \of A \in \GG)\ \lr{A}{\gamma(x) \ale \sigma(x)}\\
  && (\text{for}~\gamma,\sigma \in \ms{Ctx}(\GG))\\
  \lrcx{\GG}{A}{e_1 \ale e_2}
  &\tiff& \forall(\lr{\GG}{\gamma_1 \ale \gamma_2})\
  \commsq{A}{\gamma_1}{\gamma_2}{e_1}{e_2}\\
  \commsq{A}{\gamma_1}{\gamma_2}{e_1}{e_2}
  &\tiff& \forall(i = 1,2)\ \lr{A}{\gamma_i(e_1) \ale \gamma_i(e_2)}\\
  && \hspace{3.7em}\mathrel{\land} \lr{A}{\gamma_1(e_i) \ale \gamma_2(e_i)}
\end{array}\]
$\commsq{A}{\gamma_1}{\gamma_2}{e_1}{e_2}$ may be seen as a transitive square:
\begin{center}
  \tikzset{
    no line/.style={draw=none,
      commutative diagrams/every label/.append style={/tikz/auto=false}}}
  {\begin{tikzcd}
      \gamma_1(e_1) \ALER \ALED & \gamma_1(e_2) \ALED\\
      \gamma_2(e_1) \ALER & \gamma_2(e_2)
    \end{tikzcd}}
\end{center}

As a matter of notation, for any relation $\lr{X}{Y \ale Z}$, we write:
\[\begin{array}{ccl}
  \lr{X}{Y \aeq Z} &\tiff& \lr{X}{Y \ale Z} \land \lr{X}{Z \ale Y}\\
  \lr{X}{Y}     &\tiff& \lr{X}{Y \ale Y}
\end{array}\]

We now give the definition of $\lr{A}{a \ale b}$ by induction on $A$:
\[\begin{array}{ccl}
  %% discreteness comonad
  \lr{\disc{A}}{a \ale b} &\tiff& \lr{A}{a \aeq b}\\
  %% booleans
  \lr{\bool}{a \ale b} &\tiff&
  a \steps v \land b \steps u \land v \ale u : \bool\\
  %% sets
  \lr{\Set{A}}{a \ale b} &\tiff&
  a \steps \setlit{\vec{v_i}} \land b \steps \setlit{\vec{u_i}}\\
  && \mathrel{\land} \forall v_i\,\exists u_j\; (\lr{A}{v_i \aeq u_j})\\
  %% sums
  \lr{A_1 + A_2}{a \ale b} &\tiff&
  a \steps \ms{in}_i\;v \land b \steps \ms{in}_i\;u \land \lr{A_i}{v \ale u}\\
  %% products
  \lr{A_1 \x A_2}{a \ale b} &\tiff&
  a \steps (v_1, v_2) \land b \steps (u_1, u_2)\\
  && \mathrel{\land} \forall i\; (\lr{A_i}{v_i \ale u_i})\\
  %% discrete functions
  \lr{A \to B}{a \ale b} &\tiff& \lr{\disc{A} \mto B}{a \ale b}\\
  %% monotone functions
  \lr{A \mto B}{a \ale b} &\tiff&
  a \steps \fn\bind{x} e_1 \land b \steps \fn\bind{x} e_2\\
  && \mathrel{\land} (\lrcx{x\of A}{B}{e_1 \ale e_2})
\end{array}\]

It may not be immediately obvious that (for a given $A$) the relation $\lr{A}{a
  \ale b}$ can be seen as a preordered PER. This requires the following two
theorems, proven by induction on $A$:

\begin{theorem}[Partial reflexivity]
  If $\lr{A}{a \ale b}$ then $\lr{A}{a \ale a}$ and $\lr{A}{b \ale b}$.
\end{theorem}

\begin{theorem}[Transitivity]
  If $\lr{A}{a \ale b}$ and $\lr{A}{b \ale c}$, then $\lr{A}{a \ale c}$.
\end{theorem}

From these it follows immediately that $\lr{A}{a \aeq b}$ is a PER, and
$\lr{A}{a \ale b}$ forms a preorder over this PER which respects it.

\begin{theorem}[Termination]
  If $\lr{A}{a}$ then $a \steps v$.
\end{theorem}
\begin{proof}
  By cases on the definition of $\lr{A}{a \ale a}$.
\end{proof}

\begin{theorem}[Fundamental theorem]
  If $\J{\GD}{\GG}{e}{A}$ then $\lrcx{\disc{\GD},\GG}{A}{a}$.
\end{theorem}
\begin{proof}
  By induction on $\J{\GD}{\GG}{e}{A}$; full proof available at
  \url{https://github.com/rntz/datafun/}. The key case is the fixed point rule,
  whose proof is a syntactic version of the proof of definedness of fixed points
  in the denotational semantics.
\end{proof}

It follows as immediate corollary of termination and the fundamental property that
every closed, well-typed program terminates. 
