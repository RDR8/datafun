\section{Introduction}

The phrase ``declarative programming'' is as popular as it is
ambiguous, with seemingly hundreds of disparate senses in which it is
used. However, two of those usages stand out for popularity: both
\emph{functional} and \emph{logic} programming languages are generally
deemed declarative languages. Despite this common epithet, the logic
and functional programming traditions have largely evolved
independently of one another (with a few honorable exceptions such as
Mercury~\cite{mercury}, Curry~\cite{curry} and
Kanren~\cite{kanren}). This could be seen as an occasion for sorrow,
but we prefer to view it as an opportunity: as functional language
designers, we can look to logic languages to discover new ideas to
steal.

A Prolog program can be understood as a collection of logical axioms
formulated as Horn clauses (i.e., first-order formulas of the form
$\forall \vec{x}.\;P_1 \land \ldots \land P_n \to Q$, where $P_i$ and
$Q$ are atomic formulas).  Execution of a Prolog program can be
understood as running a proof search algorithm on these clauses to
figure out whether a particular formula is derivable or not.

In other words, functional and logic programming languages embody the
Curry-Howard correspondence in two different ways. In a functional
language, types are propositions, terms are proofs, and program
evaluation corresponds to proof normalization. On the other hand, for
logic programming languages, \emph{terms} are propositions, and
program evaluation corresponds to \emph{proof search}.

Due to the undecidability of general theorem proving, designers of
logic programming languages have to both be very careful about the
kinds of formulas they admit as programs, and to be careful about the
proof search algorithm they implement. Prolog offers a very expressive
language --- full Horn clauses --- and so it faces an undecidable
proof search problem. So to make its operational behaviour
predictable, its designers specified a particular proof search
strategy, namely depth-first goal-directed search. As a result,
programmers can reliably reason about the behaviour of a Prolog
program, at the cost of making certain logically natural programs go
into infinite loops. (Notoriously, transitive closure calculations are
much less elegant in Prolog than one might hope, since their most
natural specification is best computed with a breadth-first proof
search strategy.)

However, this view of Prolog suggests that there are other possible
design choices, such as restricting the logical language sufficiently
to make the proof search problem decidable. One of the oldest such
variants is Datalog~\cite{datalog}, which can be seen as a subset of
Prolog satisfying three restrictions on the clauses defining a program:
\begin{enumerate}
\item Terms occurring in predicates are forbidden from being
  composites: they can only be ground terms, or variables. This
  restriction ensures that deduction will never introduce new terms
  that do not occur in the source of the program.
\item Variables occurring in the head of a clause (i.e., the
  consequent of a Horn clause) must also occur (positively) in the
  body (i.e., the premises of a Horn clause).
\item No predicate can be negated unless it has already been fully
  defined.  This is sometimes called ``stratified negation''.
\end{enumerate}
These are drastic restrictions making the language Turing
incomplete. For example, it is even impossible to implement arithmetic
in Datalog, since adding 2 and 3 produces 5, which is a new term not
equal to either 2 or 3!  As functional programmers are well aware,
though, there is often power in restraint: for example, in a total
functional language, the compiler is free to switch between strict and
lazy evaluation as it deems fit.

Over the last decade or so, this freedom has been put to very good
use, with Datalog appearing at the heart of a a wild variety of
applications in both research and industry.  For example, Whaley and
Lam \cite{whaley-lam,whaley-phd} implemented pointer analysis
algorithms in Datalog, and found that they could reduce the size of
their analyses from thousands of lines of C code to \emph{tens} of
lines of Datalog code, while still retaining competitive
performance. Semmle has developed the .QL
language~\cite{semmlecode,ql-inference} based on Datalog for analysing
source code (which was used to analyze the code for NASA's Curiosity
Mars rover), and LogicBlox has developed the LogiQL~\cite{logicblox}
language for business analytics. The Boom project at Berkeley has
developed the Bloom language for distributed programming~\cite{bloom},
and the Datomic cloud database~\cite{datomic} uses Datalog (embedded
in Clojure) as its query language. Microsoft's SecPAL
language~\cite{secpal} uses Datalog as the foundation of its
decentralised authorization specification language.

All of these applications are built on the foundation of Datalog, but
one common feature is that they all needed to extend it.








\paragraph{Contributions}
\begin{itemize}
\item We describe Datafun, a typed language capturing the expressive power of
  Datalog and extending it to support higher-order functional programming.
  Datafun's key feature is to \emph{track monotonicity with types}.

\item We present examples illustrating the expressive power of Datafun,
  including relational-algebra-style operations, transitive closure, CYK
  parsing, and dataflow analysis.

\item We identify the semantic structures underpinning Datalog, and use this to
  give a denotational semantics for Datafun in terms of a pair of adjunctions
  between \cSet{}, \cPoset{}, \cSL{}.

\item We have a prototype implementation of Datafun in Racket. \todo{(CITE)}
\end{itemize}

%% Contributions (as summarized by Michael):
% - Datafun, like Datalog but functional
% - examples, incl. both datalog examples & things datalog can’t do
% - key ingredient is monotonicity; ``found'' semantics by analyzing
%   datalog: two adjunctions, three categories
% - prototype implementation

%% Contributions (as written by Neel):

% - We describe Datafun, a type theory for a language capturing the expressive
%   power of Datalog and extends it to both relax the constructor term
%   restriction and to support higher-order functional programming.

% - We give a variety of examples that illustrate the expressive power of
%   Datafun, such as CYK parsing, dataflow analysis, and transitive closure on
%   graphs, etc. Many of these examples are traditional examples of Datalog,
%   but we are also able to support things like first-class relations (eg,
%   generic transitive closure) and higher-order functions (example using
%   monotonicity and HO?). (doing a fix-point code analysis / parsing something
%   & dispatching on result?)

% - We identify the semantic structures underpinning Datalog, and use this to
%   give a denotational semantics for Datafun in terms of a pair of adjunctions
%   between Set, Poset, and the category of semilattices with finitary joins.

% - We have a prototype implementation of Datafun in Racket.

% Local Variables:
% TeX-master: "datafun"
% End:
