%% FIGURE: Typing rules
\begin{figure*}
  %% \boxed{\ensuremath{\mathsz{10pt}{\J{\Delta}{\Gamma}{e}{A}}}}
  \begin{mathpar}
    \infer[\rn{\rt{var}}]{\J{\GD}{\GG}{x}{A}}{x\of A \in \GD}
    \and
    \infer[\rn{\rt{var}^+}]{\J{\GD}{\GG}{\m{x}}{A}}{\m{x}\of A \in \GG}
    %% function rules
    \and
    \infer[\rn{\fn}]{\J{\GD}{\GG}{\fn\bind{x} e}{A \uto B}}{
      \J{\GD,x\of A}{\GG}{e}{B}}
    \and
    \infer[\rn{\rt{app}}]{\J{\GD}{\GG}{e_1\;e_2}{B}}{
      \J{\GD}{\GG}{e_1}{A \uto B} &
      \J{\GD}{\cdot}{e_2}{A}}
    \and
    \infer[\rn{\monofn}]{\J{\GD}{\GG}{\fn\bind{x} e}{A \mto B}}{
      \J{\GD}{\GG,x \of A}{e}{B}}
    \and
    \infer[\rn{\rt{app}^+}]{\J{\GD}{\GG}{e_1\;e_2}{B}}{
      \J{\GD}{\GG}{e_1}{A \mto B} &
      \J{\GD}{\GG}{e_2}{A}}
    %% product & sum rules
    \and
    \infer[\rn{\rt{pair}}]{\J{\GD}{\GG}{(e_1,e_2)}{A_1 \x A_2}}{
      \J{\GD}{\GG}{e_i}{A_i}}
    \and
    \infer[\rn{\pi}]{\J{\GD}{\GG}{\pi_i\;e}{A_i}}{\J{\GD}{\GG}{e}{A_1 \x A_2}}
    \and
    \infer[\rn{\rt{in}}]{\J{\GD}{\GG}{\ms{in}_i\;e}{A_1 + A_2}}{
      \J{\GD}{\GG}{e}{A_i}
    }
    \and
    \infer[\rn{\rt{case}}]{\J{\GD}{\GG}{\case{e}{x}{e_1}{x}{e_2}}{C}}{
      \J{\GD}{\cdot}{e}{A_1 + A_2} &
      \J{\GD,x\of A_i}{\GG}{e_i}{C}}
    \and
    \infer[\rn{\rt{case}^+}]{
      \J{\GD}{\GG}{\case{e}{\m{x}}{e_1}{\m{x}}{e_2}}{C}
    }{
      \J{\GD}{\GG}{e}{A_1 + A_2} &
      \J{\GD}{\GG,\m{x}\of A_i}{e_i}{C}
    }
    %% boolean rules
    %% \and
    %% \infer[\rn{=}]{\J{\GD}{\GG}{e_1 = e_2}{\bool}}{\J{\GD}{\GG}{e_i}{\eq{A}}}
    \and
    \infer[\rn{\rt{true}}]{\J{\GD}{\GG}{\ms{true}}{\bool}}{}
    \and
    \infer[\rn{\rt{false}}]{\J{\GD}{\GG}{\ms{false}}{\bool}}{}
    \and
    \infer[\rn{\rt{if}}]{\J{\GD}{\GG}{\ifthen{e}{e_1}{e_2}}{A}}{
      \J{\GD}{\cdot}{e}{\bool} &
      \J{\GD}{\GG}{e_i}{A}}
    \and
    \infer[\rn{\rt{if}^+}]{\J{\GD}{\GG}{\ifthen{e}{e_1}{\unit}}{L}}{
      \J{\GD}{\GG}{e}{\bool} &
      \J{\GD}{\GG}{e_1}{L}}
    %% set & semilattice rules
    \and
    \infer[\unit]{\J{\GD}{\GG}{\unit}{L}}{}
    \and
    \infer[\rn{\vee}]{\J{\GD}{\GG}{e_1 \vee e_2}{L}}{\J{\GD}{\GG}{e_i}{L}}
    \and
    \infer[\rn{\{\}}]{\J{\GD}{\GG}{\{e\}}{\Set{A}}}{\J{\GD}{\cdot}{e}{A}}
    \and
    \infer[\rn{\bigvee}]{\J{\GD}{\GG}{\letin{x}{e_1}{e_2}}{L}}{
      \J{\GD}{\GG}{e_1}{\Set{A}} &
      \J{\GD,x\of A}{\GG}{e_2}{L}}
    \and
    \infer[\rn{\{:\}}]{\J{\GD}{\GG}{\singlemap{e_1}{e_2}}{\Map{A}{B}}}{
      \J{\GD}{\cdot}{e_1}{A} &
      \J{\GD}{\GG}{e_2}{B}}
    \and
    \infer[\rn{\rt{fix}}]{\J{\GD}{\GG}{\fix{\m{x}}{e}}{\fineq{L}}}{
      \J{\GD}{\GG,x\of L}{e}{\fineq{L}}}
    \and
    \infer[\rn{\rt{fix}_{\le}}]{
      \J{\GD}{\GG}{\fixle{\m{x}}{e_1}{e_2}}{\eq{L}}
    }{
      \J{\GD}{\GG}{e_1}{\eq{L}} &
      \J{\GD}{\GG,x \of \eq{L}}{e_2}{\eq{L}}}
  \end{mathpar}

  \caption{Typing rules for core Datafun}
  \label{fig:typing-rules}
\end{figure*}


\section{Typing rules}

Datafun's typing judgment $\J{\GD}{\GG}{e}{A}$ is defined by the inference rules
given in Figure \ref{fig:typing-rules}. We gloss $\J{\GD}{\GG}{e}{A}$ as
follows: ``expression $e$ has type $A$ using variables from $\GD \cup \GG$, and
moreover the value of $e$ is \emph{monotone} with respect to the variables in
$\GG$''.

The context $\GD$ types ordinary variables; $\GG$, monotone variables. Both
admit the usual structural rules of exchange, weakening, and contraction.
Variables from either context may be used freely (rules \rt{var}, $\rt{var}^+$).
\todo{explain?}

\subsection{Functions and application}
Two function types require two function introduction rules: the ordinary
$\lambda$ and the monotone $\lambda^+$. These simply introduce variables into
their respective contexts. Monotone function application $\rt{app}^+$ is
perfectly standard, but ordinary function application \rt{app} has a pecularity:
the argument $e_2$ gets an \emph{empty} monotone context.

To understand why, recall our gloss: the application $e_1\;e_2$ must be monotone
in $\GG$. But $e_1$ is an ordinary, and in general \emph{non-monotone}, function
$A \uto B$: there is no guarantee that it respect any order on its argument. We
work around this scoff-law behavior on $e_1$'s part by ensuring its argument
$e_2$ is \emph{constant} with respect to $\GG$---which we accomplish by simply
prohibiting $e_2$ from using any of $\GG$'s variables.

This technique of \emph{wiping clean} the monotone context to guarantee
constancy\footnote{Wherever we write ``constant'' in this section, substitute
  ``constant with respect to the monotone context''. The ordinary context is
  never ``wiped clean'', and behaves entirely as it would in a simply-typed
  $\lambda$-calculus.} of a subterm recurs in several other rules. \todo{Mention
  comonads and modal logic?}

\subsection{Products and sums}
The pairing and projection rules, \rt{pair} and $\pi$, are completely standard,
as is the \rt{in} rule for sum introduction. Sum elimination, however, splits
into two rules, \rt{case} and $\rt{case}^+$. $\rt{case}^+$ requires its branches
to be monotone in the variable $\m{x}$ it introduces, and consequently its
subject $e$ is permitted access to the monotone context $\GG$. \rt{case},
however, analyses its subject $e$ as a constant --- wiping clean its monotone
context --- and thus is allowed to introduce the variable $x$ into the ordinary
context $\GD$. \todo{rewrite for clarity}

\subsection{Booleans}
While \rt{true} and \rt{false} are straightforward, there are two rules for
boolean elimination, \rt{if} and $\rt{if}^+$. This is because in Datafun, $1$
plus $1$ does not equal $\bool$: booleans are \emph{not} a sum of
units.\footnote{For simplicity, we have omitted the unit type 1 from our
  presentation of Datafun, but it is easy enough to imagine including it.} At
the type $1 + 1$, $\ms{in}_1 \triv$ and $\ms{in}_2 \triv$ are incomparable. But
in Datafun, $\ms{true} > \ms{false}$. Therefore, to eliminate a boolean in a
monotone fashion, one must ensure one's \emph{then}-branch is always greater
than one's \emph{else}-branch.

Thus Datafun has two \ms{if} rules. First, \rt{if}, where the boolean $e$ being
analysed is constant (has an empty monotone context), and so the branches $e_1$,
$e_2$ may be arbitrary expressions.

Second, $\rt{if}^+$, where the subject $e$ has full access to $\GG$, but the
\ms{if}-expression must have \emph{semilattice type}, and the \emph{else}-branch
is constrained to be $\unit$ --- the least value, thus smaller than $e_1$.

This is a conservative approach: there are many semantically monotone, but
untypeable, \ms{if}-terms. However, it is complete for semilattice types, for in
that case $(\ifthen{e}{e_1}{e_2})$ may be rewritten $(e_2 \vee
\ifthen{e}{e_1}{\unit})$; as long as $e_1 \ge e_2$ and so $e_2 \vee e_1 = e_1$,
this will not change the meaning of the expression.\footnote{If Datafun were
  Turing-complete, this would no longer be the case; consider the case where
  $e_2$ diverges.}

Thus the only meaningful restriction here is to semilattice types. In practice,
we have yet to find a case where this is problematic.

\todo{Mention that $e_1 \ge e_2$ is in general statically undecidable (rice's
  theorem), and that a typesystem to check it would amount to dependent typing?}

\subsection{Semilattices and sets}
The semilattice $\unit$ and $\vee$ operations are typed by their
correspondingly-named rules. Since $\vee$ is by definition monotone, its
arguments have full access to the monotone context $\GG$.

Sets are simply the free semilattice on an underlying set, so aside from the
semilattice rules,

\subsection{Maps}
Finite maps are an abstract type, and most of their interface is exposed through
primitive functions (see Figure \ref{fig:primitives}). The only typing rule is
$\{:\}$, the introduction rule for singleton maps. By analogy with the
introduction rule for singleton sets $\{\}$,

\subsection{Fixed points}

\newpage
