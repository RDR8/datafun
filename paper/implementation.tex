\section{Implementation}
We have built a proof-of-concept implementation of Datafun in Racket, available
at \texttt{[link omitted for double-blind review]}. In addition to core Datafun,
it supports pattern-matching, variant types, record types, dictionaries,
subtyping, antitone functions, and unbounded (potentially nonterminating) fixed
points. We implement everything in a naive style, and perform no optimizations.

\paragraph{Type inference}
As a practical matter, type-checking needs to distinguish between ordinary and
monotone $\lambda$, application, \ms{case}, \ms{let}, and \ms{if}. In our
implementation we solve this in two ways:
\begin{enumerate}
\item Bidirectional type inference~\cite{bidirectional} determines whether
  $\fn$s and applications are ordinary or monotone.

\item For $\ms{if}$, $\ms{case}$, and $\ms{let}$, the programmer annotates which
  form is intended; for example, $(\ifthen{e}{e_1}{\unit})$ is written
  (\texttt{when e then e1}) to indicate the rule $\ms{if}^+$ applies.
\end{enumerate}

We believe that this scheme could be extended to support polymorphism
in the style of \citet{dunfield-bidir}. However, it would not be an
entirely off-the-shelf affair, since we would want to add support for
polymorphism over the tones of function, so that, for example,
$\fn\bind{f}\fn\bind{x} f\;x$ can be assigned the principal type
$\forall\bind{o\of\ms{tone}}\forall\bind{\alpha,\beta \of \ms{type}}
(\alpha \overset{o}\to \beta) \mto (\alpha \overset{o}\to \beta)$,
where $\overset{o}\to$ indicates a function of tone $o$; a tone may be
empty (for an ordinary function) or ${+}$ for a monotone function.

% 
% We speculate that
% bidirectional inference could be replaced by a Damas-Milner \todo{CITE} style
% algorithm, which infers a principal type for any term without any annotation at
% all, \emph{if} we add polymorphism, tone-polymorphism, and subtyping---so that,
% 

%% \todo{explain subtyping?}
%% \todo{explain antitonicity?}
%% \todo{explain ordering on dictionaries?}
