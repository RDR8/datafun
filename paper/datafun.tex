\documentclass[preprint]{sigplanconf}

\usepackage{rntz}

% The following \documentclass options may be useful:

% preprint      Remove this option only once the paper is in final form.
% 10pt          To set in 10-point type instead of 9-point.
% 11pt          To set in 11-point type instead of 9-point.
% numbers       To obtain numeric citation style instead of author/year.

%% \newcommand{\cL}{{\cal L}}

\begin{document}

\special{papersize=8.5in,11in}
\setlength{\pdfpageheight}{\paperheight}
\setlength{\pdfpagewidth}{\paperwidth}

\conferenceinfo{ICFP '16}{Month d--d, 2016, City, ST, Country}
\copyrightyear{2016}
\copyrightdata{978-1-nnnn-nnnn-n/yy/mm}
\copyrightdoi{nnnnnnn.nnnnnnn}

% Uncomment the publication rights you want to use.
%\publicationrights{transferred}
%\publicationrights{licensed}     % this is the default
%\publicationrights{author-pays}

% These are ignored unless 'preprint' option specified.
\titlebanner{preprint}
\preprintfooter{Datafun, or, Datalog with datatypes (PREPRINT)}

\title{Datafun}
\subtitle{or, Datalog with datatypes}

\authorinfo{Michael Arntzenius\and Neelakantan R. Krishnaswami}
           {University of Birmingham}
           {Email2/3}

\maketitle

\begin{abstract}
This is the text of the abstract.
\end{abstract}

\category{CR-number}{subcategory}{third-level}

% general terms are not compulsory anymore,
% you may leave them out
\terms
term1, term2

\keywords
keyword1, keyword2

\section{Introduction}

The text of the paper begins here.

\section{Datafun}

\begin{figure}
  \[\begin{array}{lccl}
  \text{types} & A, B &::=& \bool \pipe \N\\
  \text{expressions} & e &::=&
  \end{array}\]
  \caption{Datafun syntax}
\end{figure}

\paragraph{Contributions}

\acks

Acknowledgments, if needed.

% We recommend abbrvnat bibliography style.

\bibliographystyle{abbrvnat}
\bibliography{datafun}

%% \appendix
%% \section{Appendix Title}

%% This is the text of the appendix, if you need one.


\end{document}
