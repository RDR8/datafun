\documentclass[preprint]{sigplanconf}

% The following \documentclass options may be useful:

% preprint      Remove this option only once the paper is in final form.
% 10pt          To set in 10-point type instead of 9-point.
% 11pt          To set in 11-point type instead of 9-point.
% numbers       To obtain numeric citation style instead of author/year.

\usepackage{rntz}

\renewcommand{\pipe}{\mathrel{\mid}} %overrides rntz.sty
\newcommand{\uto}{\to}
\newcommand{\rn}[1]{\mathsz{8pt}{\ms{#1}}} %for inference rule names
\def\arraystretch{1.1}                     %a little more spacing, please
\renewcommand{\FS}[1]{\ms{FS}\;{#1}}
\newcommand{\FM}[2]{\ms{Map}\;{#1}\;{#2}}

%% taken from the source for Dunfield & Krishnaswami's paper "Complete and Easy
%% Bidirectional Typechecking for Higher-Rank Polymorphism"
\newdimen\zzfontsz
\newcommand{\fontsz}[2]{\zzfontsz=#1%
{\fontsize{\zzfontsz}{1.2\zzfontsz}\selectfont{#2}}}
\newcommand{\mathsz}[2]{\text{\fontsz{#1}{$#2$}}}

\begin{document}

\special{papersize=8.5in,11in}
\setlength{\pdfpageheight}{\paperheight}
\setlength{\pdfpagewidth}{\paperwidth}

\conferenceinfo{ICFP '16}{18--24 September, 2016, Nara, Nara, Japan}
\copyrightyear{2016}
\copyrightdata{978-1-nnnn-nnnn-n/yy/mm}
\copyrightdoi{nnnnnnn.nnnnnnn}

% Uncomment the publication rights you want to use.
%\publicationrights{transferred}
%\publicationrights{licensed}     % this is the default
%\publicationrights{author-pays}

% These are ignored unless 'preprint' option specified.
\titlebanner{preprint}
\preprintfooter{Datafun: a Functional Datalog (PREPRINT)}

\title{Datafun: a Functional Datalog}
\subtitle{}

%% \title{Datafun}
%% \subtitle{A Functional Datalog}

%% \title{Datafun}
%% \subtitle{or, Datalog with datatypes}

%% \title{Datalog with Datatypes}
%% \subtitle{Toward a functional language of finite sets}

\authorinfo{Alfred Nony Mouskovi\v{c}}
           {}
           {a.mouse@example.org}

%% \authorinfo{Michael Arntzenius\and Neelakantan R. Krishnaswami}
%%            {University of Birmingham}
%%            {daekharel@gmail.com, N.Krishnaswami@cs.bham.ac.uk}

\maketitle


\begin{abstract}
This is the text of the abstract.
\end{abstract}

\category{CR-number}{subcategory}{third-level}

% general terms are not compulsory anymore,
% you may leave them out
\terms
term1, term2

\keywords
keyword1, keyword2


\section{Introduction}

The phrase ``declarative programming'' is as popular as it is ambiguous, with
seemingly hundreds of disparate senses in which it is used. However, two of
those usages stand out for popularity: both \emph{functional} and \emph{logic}
programming languages are generally deemed declarative languages. Despite this
shared epithet, the logic and functional programming traditions have largely
evolved independently of one another (with a few honorable exceptions such as
Mercury~\cite{mercury}, Curry~\cite{curry} and Kanren~\cite{kanren}). This could
be seen as an occasion for sorrow, but we prefer to view it as an opportunity:
as functional language designers, we can look to logic languages to discover new
ideas to steal.

A Prolog program can be understood as a collection of logical axioms formulated
as Horn clauses (i.e., first-order formulas of the form $\forall \vec{x}.\;P_1
\land \ldots \land P_n \to Q$, where $P_i$ and $Q$ are atomic formulas).
Execution of a Prolog program can be understood as running a proof search
algorithm on these clauses to figure out whether a particular formula is
derivable or not.

In other words, functional and logic programming languages embody the
Curry-Howard correspondence in two different ways. In a functional language,
types are propositions, terms are proofs, and program evaluation corresponds to
proof normalization. On the other hand, for logic programming languages,
\emph{terms} are propositions, and program evaluation corresponds to \emph{proof
  search}.

Since proof search is in general undecidable, designers of logic programming
languages must be careful both about the kinds of formulas they admit as
programs, and about the proof search algorithm they implement. Prolog offers a
very expressive language --- full Horn clauses --- and so faces an undecidable
proof search problem. Therefore, Prolog specifies its proof search strategy:
depth-first goal-directed/top-down search. This lets Prolog programmers reason
about the behaviour of their programs; however, it also means many logically
natural programs fail to terminate. Notoriously, transitive closure calculations
are much less elegant in Prolog than one might hope, since their most natural
specification is best computed with a bottom-up (aka ``forwards chaining'')
proof search strategy.

This view of Prolog suggests considering other possible design choices, such as
restricting the logical language so as to make proof search decidable. One of
the oldest such variants is Datalog~\cite{datalog}, a syntactic subset of Prolog
satisfying three restrictions:

\begin{enumerate}
\item Programs must be \emph{constructor-free}: only atomic terms and variables
  are permitted to appear as arguments to predicates. This ensures that deduction
  will not introduce terms that do not occur in the source of the
  program.

\item Clauses are \emph{range-restricted}: all variables in the
  consequent (head) of a clause must also occur positively in its
  premises (body).

\item Programs are limited to \emph{stratified negation}: the
  negation of a predicate may be used in a definition only if it has
  already been fully defined. That is, within the recursive
  definition of a predicate, it cannot be used in negated form.
\end{enumerate}

These restrictions make Datalog Turing-\emph{in}complete: all queries are
decidable. As functional programmers are well aware, though, there is power in
restraint: for example, in a total functional language, the compiler may switch
between strict and lazy evaluation at will. Similarly, in Datalog decidability
means that implementations are free to use forwards chaining, and so can easily
support queries (like reachability and transitive closure) which are difficult
to implement in ordinary Prolog.

%% PREVIOUS PARAGRAPH:
%
% Thanks to this decidability, Datalog implementations are free to
% tailor their proof search strategy to the program being executed.
% Consequently Datalog programs can be both concise and efficient. For
% example, \citet{whaley-lam} implemented pointer analysis algorithms in
% Datalog, and found that they could reduce their analyses from
% \emph{thousands} to \emph{tens} of lines of code while retaining
% competitive performance.

Over the last decade or so, this freedom has been put to good use, with Datalog
appearing at the heart of a a wild variety of applications in both research and
industry. For example, Whaley and Lam \cite{whaley-lam,whaley-phd} implemented
pointer analysis algorithms in Datalog, and found that they could reduce their
analyses from thousands of lines of C code to \emph{tens} of lines of Datalog
code, while retaining competitive performance. Semmle has developed the .QL
language~\cite{semmlecode,ql-inference} based on Datalog for analysing source
code (which was used to analyze the code for NASA's Curiosity Mars rover), and
LogicBlox has developed the LogiQL~\cite{logicblox} language for business
analytics. The Boom project at Berkeley has developed the Bloom language for
distributed programming~\cite{bloom}, and the Datomic cloud
database~\cite{datomic} uses Datalog (embedded in Clojure) as its query
language. Microsoft's SecPAL language~\cite{secpal} uses Datalog as the
foundation of its decentralised authorization specification language.

In all of these cases, the use of Datalog permits giving specifications and
implementations which are dramatically shorter and clearer than alternatives
implemented in more conventional languages. However, while all of these
applications are built on a foundation of Datalog, they all also extend it
significantly. For example, it is impossible even to implement arithmetic in
Datalog, since adding 2 and 3 produces 5, which is a new term not equal to
either 2 or 3! As a result, even though Datalog has a very clean semantics, its
metatheory needs to be re-established once again for each application-specific
extension to it.

As a result, it would be very desirable to understand what makes Datalog tick,
so that we can embed it into a more expressive language \emph{without}
sacrificing the properties that make it so powerful within its domain. In this
way, extensions can become ``a small matter of programming'', without having to
do a custom redesign of the language for each application.

%% TODO: cite the jargon file for SMOP?

In this paper, we present Datafun, a typed functional language which permits
programming in the style of Datalog, while still supporting the full power of
higher-order functional programming.

\paragraph{Contributions}
\begin{itemize}
\item We describe Datafun, a typed language capturing the expressive power of
  Datalog and extending it to support higher-order functional programming.
  Datafun's key feature is to \emph{track monotonicity with types}. This permits
  us to use typing to analyze fixed point computations in a way ensuring their
  termination.

\item We present examples illustrating the expressive power of Datafun,
  including relational-algebra-style operations, transitive closure, CYK
  parsing, and dataflow analysis. Some of these examples are familiar from
  Datalog, but many of them go well beyond what can be expressed in it,
  illustrating the benefits of our approach.

\item We identify the semantic structures underpinning Datalog, and
  use this to give a denotational semantics for Datafun in terms of a
  pair of adjunctions between \cSet{}, \cPoset{}, \cSL{}.

\item We have a prototype implementation of Datafun in Racket, which
  has been used to implement all of the examples in this paper, and is
  available at \texttt{[link omitted for double-blind review]}. 
\end{itemize}

%% Contributions (as summarized by Michael):
% - Datafun, like Datalog but functional
% - examples, incl. both datalog examples & things datalog can’t do
% - key ingredient is monotonicity; ``found'' semantics by analyzing
%   datalog: two adjunctions, three categories
% - prototype implementation

%% Contributions (as written by Neel):

% - We describe Datafun, a type theory for a language capturing the expressive
%   power of Datalog and extends it to both relax the constructor term
%   restriction and to support higher-order functional programming.

% - We give a variety of examples that illustrate the expressive power of
%   Datafun, such as CYK parsing, dataflow analysis, and transitive closure on
%   graphs, etc. Many of these examples are traditional examples of Datalog,
%   but we are also able to support things like first-class relations (eg,
%   generic transitive closure) and higher-order functions (example using
%   monotonicity and HO?). (doing a fix-point code analysis / parsing something
%   & dispatching on result?)

% - We identify the semantic structures underpinning Datalog, and use this to
%   give a denotational semantics for Datafun in terms of a pair of adjunctions
%   between Set, Poset, and the category of semilattices with finitary joins.

% - We have a prototype implementation of Datafun in Racket.

% Local Variables:
% TeX-master: "datafun"
% End:



%% FIGURE: CORE SYNTAX
\begin{figure}
  \[\begin{array}{ccl}
    %% types
    A, B     &\bnfeq& \bool \pipe \N \pipe \str \pipe \Set{A}
                      \pipe \Map{A}{B}
    \\
    \textsf{types} && A + B \pipe A \x B \pipe A \uto B \pipe A \mto B
    \vspace{0.5em}\\
    %% semilattice types
    L, M         &\bnfeq& \bool \pipe \N \pipe \Set{A} \pipe \Map{A}{L}
    \\
    \textsf{semilattice types} && L \x M \pipe A \uto L \pipe A \mto L
    \vspace{0.5em}\\
    %% equality types
    \eq{A}, \eq{B} &\bnfeq& \bool \pipe \N \pipe \str \pipe \Set{\eq{A}}
                            \pipe \Map{\eq{A}}{\eq{B}}
    \\
    \textsf{eqtypes} && \eq{A} + \eq{B} \pipe \eq{A} \x \eq{B}
    \vspace{0.5em}\\
    %% finite equality types
    \fineq{A},\fineq{B}
    &\bnfeq& \bool \pipe \Set{\fineq{A}}
       \pipe \Map{\fineq{A}}{\fineq{B}}
       \pipe \fineq{A} + \fineq{B} \pipe \fineq{A} \x \fineq{B}\\
    \textsf{finite eqtypes}
    \vspace{0.5em}\\
    %% contexts
    \GD,\GG &\bnfeq& \cdot \pipe \GD, x\of A\\
    \textsf{contexts}\\
    %% expressions
    e &\bnfeq& x \pipe \fn\bind{x} e \pipe e\;e
    \pipe (e,e) \pipe \pi_i\;e \pipe \ms{in}_i\;e\\
    \textsf{terms}
    && \case{e}{x}{e}{x}{e}\\
    && e = e \pipe \ms{true} \pipe \ms{false} \pipe \ifthen{e}{e}{e}\\
    && \singleset{e} \pipe \singlemap{e}{e}
       \pipe \unit \pipe e \vee e \pipe \letin{x}{e}{e}\\
    && \fix{x}{e} \pipe \fixle{x}{e}{e}
  \end{array}\]
  \caption{Syntax of core Datafun}
  \label{fig:syntax}
\end{figure}


\section{Datafun, informally}

We give the core syntax of Datafun in Figure \ref{fig:syntax}. Datafun is a
simply-typed $\lambda$-calculus extended in four major ways:

\begin{enumerate}
\item We add a type of finite sets, $\Set{A}$. We use finite sets to represent
  Datalog predicates; one might also think of them as tables or views in a
  database setting. We also include finite maps $\Map{A}{B}$, as they are useful
  for some examples.

  \todo{TODO: make clearer bit about what we use finite sets for. don't rely on
    Datalog so heavily.}

\item We add a type of \emph{monotone functions}, $A \mto B$. Consequently
  Datafun has two flavors of variable, \emph{ordinary} and \emph{monotone}.
  Ordinary functions introduce ordinary variables; monotone functions introduce
  monotone variables. This distinction is left latent in the formal syntax,
  \todo{TODO: why?}, but for clarity our examples put monotone variables in
  \textbf{bold}.

  In order for ``monotone'' to have meaning, our types are implicitly partially
  ordered. \todo{Discuss orderings}. Sets are ordered by inclusion: $a \le b :
  \Set{A}$ iff $a \subseteq b$. Maps are ordered both by inclusion and
  pointwise.

\item We add a \emph{monotone least fixed point} term, $(\fix{\m{x}}{e})$,
  where $e$ must be monotone in $\m{x}$. This allows us to express
  Datalog-style recursion.

  \todo{Explain:
    \begin{itemize}
    \item compute fixpoint (at least conceptually) by iteration
    \item start from a least element
    \item to ensure termination, need finite height of lattice
    \item finiteness guarantees finite height!
    \item (\emph{only then} mention analogy to Datalog)
  \end{itemize}}

  To ensure termination, the type at which this fixed-point is taken must be
  \emph{finite} (this corresponds directly to Datalog's requirement that
  predicates be finite). To ensure uniqueness of the least fixed point, the type
  must have a \emph{least element}, which we guarantee by only allowing fixed
  points at \emph{semilattice types} (see point \ref{item:semilattice-types},
  just below).

\item\label{item:semilattice-types} Generalizing the empty set $\emptyset$ and
  the union operator $\cup$, we identify a subset of types that have a
  \emph{least element} $\unit$ and support a \emph{least-upper bound} operator
  $\vee$. We call these \emph{semilattice types}\footnote{Technically, the
    partial orderings on these types form \emph{join-semilattices with a least
      element}. For brevity's sake, we call these structures simply
    ``semilattices.''}, and denote them by the metavariables $L,M$.

  Semilattice types are the ``glue'' of Datafun; they do not serve one single
  purpose, but simplify the language by satisfying many small ones at once.

  First, semilattice types provide a natural comprehension-style eliminator for
  sets: given $e : \Set{A}$ we may write $\letin{x}{e}{e'}$ for the
  least-upper-bound, over all elements $x \in e$, of $e'$, as long as $e' : L$
  is of semilattice type. Notably, if $e'$ is itself of set type, this provides
  a monadic bind operation. \todo{explain testing for set membership?}

  Second, we restrict fixed-points $(\fix{\m{x}}{e})$ to semilattice types,
  which guarantees that we take

  \todo{Explain how products of semilattice being semilattice + monotone
    fixed-points account for mutual recursion.}

  \todo{Rewrite this to be clearer.}

\end{enumerate}


%% FIGURE: SYNTAX SUGAR
\begin{figure}
  \[\begin{array}{lccl}
  %% expressions
  \textsf{terms} &
  e &\bnfeq& ... \pipe e \isin e \pipe \setfor{e}{\mc{L}}
             \pipe \forin{\mc{L}}{e}\\
  &&& \mathcal{C}[\;e]^* \pipe \rawcase{e}{[{p} \to {e}]^*}
  \vspace{0.5em}\\
  %% patterns

  %% TODO: maybe remove the pattern-matching stuff? since we don't explain how
  %% to translate it & we also use various other sugar we don't explain how to
  %% translate, why do we include only pattern-matching here?
  \textsf{patterns} &
  p &\bnfeq& \pwild \pipe x \pipe (p,p)
             \pipe \ms{true} \pipe \ms{false} \pipe \mathcal{C}[\;p]^*
  \vspace{0.5em}\\
  \textsf{constructors} & \mathcal{C} && \text{are abstract identifiers}
  \vspace{0.5em}\\
  %% loop clauses
  \textsf{loops} &
  \mc{L} &\bnfeq& \mc{L}, \mc{L} \pipe p \in e \pipe e
  \end{array}\]

  %% the desugaring syntax-expansion itself
  \begin{eqnarray*}
    %% e_1 \isin e_2            &\expandsto& \forin{x \in e_2} e_1 = x\\
    \setfor{e}{\mc{L}}       &\expandsto& \forin{\mc{L}}{\{e\}}\\
    \forin{\mc{L}_1,\mc{L}_2}{e}
    &\expandsto& \forin{\mc{L}_1}{\forin{\mc{L}_2}{e}}\\
    \forin{p\in e_1}{e_2}
    &\expandsto& \letin{x}{e_1}{\rawcase{x}{p \to e_2;\,\pwild \to \unit}}\\
    \forin{e_1}{e_2} &\expandsto& \ifthen{e_1}{e_2}{\unit}
    %% e_1 \x e_2 &\expandsto& \setfor{(a,b)}{a \in e_1, b \in e_2}\\
    %% e_1 \bullet e_2 &\expandsto& \setfor{(a,c)}{(a,b) \in e_1, (b,c) \in e_2}
    %% \ms{let}~x = e_1 ~\ms{in}~ e_2
    %% &\expandsto& (\fn\bind{x} e_2)\; e_1\\
    %% \ms{let}~[x_i = e_i]^* ~\ms{in}~ e
    %% &\expandsto& [\ms{let}~x_i = e_i~\ms{in}]^* e\\
    %% \rawcase{e}{[p \to e]^*} &\expandsto& \text{(omitted, see \todo{CITE})}
  \end{eqnarray*}
  \caption{Syntax sugar}
  \label{fig:sugar}
\end{figure}


\section{Examples}

For purposes of these examples, we use a simple Haskell-like syntax for
top-level type and function definitions. We also permit ourselves infix
notation, \ms{let}-binding, $n$-ary tuples, $n$-ary sum types with named
constructors, pattern-matching\todo{(CITE)}, and additional syntax sugar given
in Figure \ref{fig:sugar} \todo{(TODO: mention \& cite monadic query syntax)}.
All of these conveniences are supported (with slightly different concrete
syntax) in our implementation.

For clarity, we set the names of top-level variables in \textsf{sans-serif};
ordinary variables in $script$; and monotone variables in \m{bold}.

\todo{TODO: Explain nonlinear variable usage means equality.}

\todo{\begin{itemize}
\item composition of relations?
\item \texttt{make}-style topological sort?
\item SQL-style examples?
\item translating relational algebra into datafun?
\end{itemize}}

%% Examples

\subsection{Filtering, mapping, and cross products}

Although Datafun as presented does not have polymorphism, we give our examples
the most general possible type schema. \todo{why?}

\todo{these examples are perhaps a good place to point out subtleties of
  monotonicity typing}

\[\begin{array}{l}
\fname{map} ~:~ (A \uto B) \to \Set{A} \mto \Set{B}\\
\fname{map}\;f\;\m{A} = \setfor{f\;x}{x \in \m{A}}\\
\\
\fname{filter} ~:~ (A \uto \bool) \mto \Set{A} \mto \Set{A}\\
\fname{filter}\;\m{f}\;\m{A} = \setfor{x}{x \in \m{A}, \m{f}\; x}\\
\\
(\times) ~:~ \Set{A} \mto \Set{B} \mto \Set{A \x B}\\
\m{A} \times \m{B} = \setfor{(a,b)}{a \in \m{A}, b \in \m{B}}
\end{array}\]

\TODO

\subsection{Testing for membership}

\todo{Use this as example of $\bigvee$ at non-set (here, boolean) type.}

\[\begin{array}{l}
(\isin) ~:~ \eq{A} \to \Set{\eq{A}} \mto \bool\\
x \isin \m{A} = \forin{y \in \m{A}} x = y
\end{array}\]

\todo{TODO: discuss nonlinear pattern-matching}

An equivalent definition, making use of nonlinear pattern-matching, would be:
\[\begin{array}{l}
x \isin \m{A} = \forin{x \in \m{A}} \ms{true}
\end{array}\]

\todo{Nonlinear pattern matching makes implicit the equality test and attendant
  restriction to equality types, but is often extremely convenient.}

\subsection{Composition of relations}

\todo{mention eqtypes again here? or use this to show them off instead of
  $({\isin})$?} \todo{work phrase ``higher-order'' in here somewhere?}

\noindent As an example of a function over relations, consider:
\[\begin{array}{l}
(\bullet) : \Set{A \x \eq{B}} \mto \Set{\eq{B} \x C} \mto \Set{A \x C}\\
\m{R} \bullet \m{S} = \setfor{(a,c)}{(a,b) \in \m{R}, (b,c) \in \m{S}}
\end{array}\]

\todo{TODO: mention you can extend to all relational algebra? give example?}

\subsection{Transitive closure}

Consider the following Datalog program:
\begin{verbatim}
parent(earendil, elrond).
parent(elrond, arwen).
ancestor(X, Y) :- parent(X, Y).
ancestor(X, Z) :- ancestor(X, Y), ancestor(Y, Z).
\end{verbatim}

\todo{Discuss how this works in Datalog, but not in Prolog, b/c Prolog is
  defined by operational semantics of unification while Datalog is denotational,
  least-model semantics. It also works in Datafun!}

\todo{Neel suggests using distinction b/w backward \& forward chaining here,
  rather than operational/denotational. see Logical Algorithms paper by
  McAllister \& co for phrasing?}

In Datafun, we may write this as:

\[\begin{array}{l}
\mathbf{data}~\ms{person} =
\ctor{E\"arendil} ~|~ \ctor{Elrond} ~|~ \ctor{Arwen}\\
\fname{parent},~\ms{ancestor} ~:~ \Set{\ms{person} \x \ms{person}}\\
\ms{parent} =
\{(\ctor{E\"arendil}, \ctor{Elrond}), (\ctor{Elrond}, \ctor{Arwen})\}\\
\ms{ancestor} = \fix{\m{X}} \ms{parent} \vee (\m{X} \bullet \m{X})
%% \setfor{(a,c)}{(a,b) \in \m{X}, (b,c) \in \m{X}}
\\
\end{array}\]

\subsubsection{Transitive closure with an upper bound}

If you know the type over which you are taking your transitive closure is
finite, this suffices. But the more general way to take a fixed-point is to give
an upper bound which you know the desired answer will not exceed. For this we
write $(\fixle{\m{x}}{e_\top} e)$

\todo{mention using $\str$ as example}

\todo{explain what happens when $e \ge e_\top$ --- it gets clamped}

\[\begin{array}{l}
\ms{person} ~:~ \Set{\str}\\
\ms{person} = \{\texttt{"e\"arendil"}, \texttt{"elrond"}, \texttt{"arwen"}\}\\
\ms{parent}, \ms{ancestor} ~:~ \Set{\str \x \str}\\
\ms{parent} = \{(\texttt{"e\"arendil"}, \texttt{"elrond"}),
(\texttt{"elrond"}, \texttt{"arwen"})\}\\
\ms{ancestor} = \fixle{\m{X}}{(\ms{person} \x \ms{person})}
\ms{parent} \vee (\m{X} \bullet \m{X})
%% \ms{ancestor} = \fixle{\m{X}}{%
%%   \setfor{(a,b)}{a \in \ms{person}, b \in \ms{person}}
%% }
%% \\
%% \hspace{4.87em}\ms{parent} \vee
%% \setfor{(a,c)}{(a,b) \in \m{X}, (b,c) \in \m{X}}
\end{array}\]

\subsubsection{Generic transitive closure}

For any finite eqtype $\fineq{A}$, we may write:
\[\begin{array}{l}
\ms{trans} ~:~ \Set{\fineq{A} \x \fineq{A}} \mto \Set{\fineq{A} \x \fineq{A}}
\vspace{0.3em}\\
%% \ms{trans}\ E = \fix{X} E \vee \setfor{(a,c)}{(a,b) \in E, (b,c) \in X}
%% \ms{trans}\ \m{E} = \fix{\m{X}} \m{E} \vee %
%% \setfor{(a,c)}{(a,b) \in \m{X}, (b,c) \in \m{X}}
\ms{trans}\ \m{E} = \fix{\m{X}} \m{E} \vee (\m{X} \bullet \m{X})
\end{array}\]

\noindent Similarly, for any eqtype $\eq{A}$, we may write:
\[\begin{array}{l}
\ms{trans} ~:~
\Set{\eq{A}} \mto \Set{\eq{A} \x \eq{A}} \mto \Set{\eq{A} \x \eq{A}}
\vspace{0.3em}\\
%% \ms{trans}\ \m{V}\ \m{E} = %
%% \ms{fix}~ \m{S} \le \setfor{(a,b)}{a\in \m{V}, b \in \m{V}}\\
%% \hspace{5.35em}\ms{is}~ \m{E} \vee %
%% \setfor{(a,c)}{(a,b) \in \m{S}, (b,c) \in \m{S}}\\
\ms{trans}\ \m{V}\ \m{E} = %
\fixle{\m{S}}{(\m{V} \x \m{V})} \m{E} \vee (\m{S} \bullet \m{S})
\end{array}\]

\TODO This already shows how we go beyond Datalog's .


%% FIGURE: PRIMITIVES
\begin{figure}
  %% TODO: remove unused primitives.
  \[\begin{array}{cll}
  \le &\of& \eq{A} \uto \eq{A} \mto \bool\\
  \fname{keys}     &:& \Map{A}{B} \mto \Set{A}\\
  \fname{entries}  &:& \Map{A}{B} \uto \Set{A \x B}\\
  \fname{tabulate} &:& \Set{A} \mto (A \to B) \mto \Map{A}{B}\\
  %% \fname{getWith}  &:& \Map{\eq{A}}{B} \mto \eq{A} \uto (B \mto L) \mto L\\
  \fname{get}      &:& \Map{\eq{A}}{L} \mto \eq{A} \uto L\\
  %% \fname{substrings} &\of& \ms{Str} \to \Set{\ms{Str}}\\
  %% \fname{size}     &:& \Set{\eq{A}} \mto \N\\
  \fname{range}    &:& \N \uto \N \mto \Set{\N}\\
  \fname{length}   &:& \str \uto \N\\
  \fname{substring} &:& \str \uto \N \uto \N \uto \str
  \end{array}\]
  \caption{Primitive functions and their type schemes}
\end{figure}

\subsection{CYK parsing}

\[\begin{array}{l}
\mathbf{type}~\ms{sym} = \str\\
\mathbf{data}~\ms{rule} = \ctor{String}~\str ~|~ \ctor{Cat}~\ms{sym}~\ms{sym}\\
\mathbf{type}~\ms{grammar} = \Set{\ms{sym} \x \ms{rule}}\\
\mathbf{type}~\ms{fact} = \ms{sym} \x \N \x \N\\
\fname{parse} ~:~ \str \uto \ms{grammar} \mto \Set{\ms{sym}}\\
\fname{iter} ~:~ \str \uto \ms{grammar} \mto \Set{\ms{fact}}
\mto \Set{\ms{fact}}\\
\fname{parse}\;w\;\m{g} =\\
\hspace{1em} \ms{let}~ n = \ms{length}\;w\\
\hspace{2.375em}\m{u} =
  \setfor{(a,i,j)}{(a,\pwild) \in \m{g}, i \in \ms{range}\;0\;n,
  j\in\fname{range}\;i\;n}\\
\hspace{1em}\ms{in}~\setfor{a}{(a, 0, n) \in %
  \fixle{\m{X}}{\m{u}} \ms{iter}\;w\;\m{g}\;\m{X}}\\
\fname{iter}\;w\;\m{g}\;\m{X} =\\
\hspace{1em}\phantom{\vee~}\setfor{(a,i,k)}{(a, \ctor{Cat}~b~c) \in \m{g},
  (b,i,j) \in \m{X}, (c,j,k) \in \m{X}}\\
\hspace{1em}\vee~ \{(a,i,i+\fname{length}\;s)\\
\hspace{2.1em}|~ (a, \ctor{String}~s) \in \m{g},
i \in \fname{range}\;0\;(n-\fname{length}\;s),\\
\hspace{2.2em}\phantom{|~} s = \fname{substring}\;w\;i\;(i+\ms{length}\;s)\}
\end{array}\]

\TODO Since Datalog provides no way to represent a \emph{grammar} as a piece of
data (because it's compound, not an atom), there is simply no way in Datalog
proper to express a generic CYK parser. This demonstrates the benefits of moving
to a functional language.

\TODO Moreover, Datalog's strategy for proving termination involves a
constructor restriction to ensure all relations are finite. Primitives such as
\ms{range} and \ms{substring} violate this restriction (as relations, they are
infinite); it is not immediately obvious that Datalog programs extended with
these primitives remain terminating. Our use of bounded fixed-points to
guarantee termination is robust under such extensions; as long as all primitive
functions are total, Datafun programs always terminate.

\subsection{Dataflow analysis}

\TODO


%% FIGURE: Typing rules
\begin{figure*}
  \boxed{\ensuremath{\mathsz{10pt}{\J{\Delta}{\Gamma}{e}{A}}}}
  \begin{mathpar}
    \infer[\rn{var}]{\J{\GD}{\GG}{x}{A}}{x\of A \in \GD \cup \GG}
    %% function rules
    \and
    \infer[\fn]{\J{\GD}{\GG}{\fn\bind{x} e}{A \uto B}}{
      \J{\GD,x\of A}{\GG}{e}{B}}
    \and
    \infer[\monofn]{\J{\GD}{\GG}{\fn\bind{x} e}{A \mto B}}{
      \J{\GD}{\GG,x \of A}{e}{B}}
    \and
    \infer[\rn{app}]{\J{\GD}{\GG}{e_1\;e_2}{B}}{
      \J{\GD}{\GG}{e_1}{A \uto B} &
      \J{\GD}{\cdot}{e_2}{A}}
    \and
    \infer[\rn{app^+}]{\J{\GD}{\GG}{e_1\;e_2}{B}}{
      \J{\GD}{\GG}{e_1}{A \mto B} &
      \J{\GD}{\GG}{e_2}{A}}
    %% product & sum rules
    \and
    \infer[\rn{pair}]{\J{\GD}{\GG}{(e_1,e_2)}{A_1 \x A_2}}{
      \J{\GD}{\GG}{e_i}{A_i}}
    \and
    \infer[\rn{\pi}]{\J{\GD}{\GG}{\pi_i\;e}{A_i}}{\J{\GD}{\GG}{e}{A_1 \x A_2}}
    \and
    \infer[\rn{in}]{\J{\GD}{\GG}{\ms{in}_i\;e}{A_1 + A_2}}{\J{\GD}{\GG}{e}{A_i}}
    \and
    \infer[\rn{case}]{\J{\GD}{\GG}{\case{e}{x}{e_1}{x}{e_2}}{C}}{
      \J{\GD}{\cdot}{e}{A_1 + A_2} &
      \J{\GD,x\of A_i}{\GG}{e_i}{C}}
    \and
    \infer[\rn{case^+}]{\J{\GD}{\GG}{\case{e}{x}{e_1}{x}{e_2}}{C}}{
      \J{\GD}{\GG}{e}{A_1 + A_2} &
      \J{\GD}{\GG,x\of A_i}{e_i}{C}}
    %% boolean rules
    \and
    \infer[\rn{=}]{\J{\GD}{\GG}{e_1 = e_2}{\bool}}{\J{\GD}{\GG}{e_i}{\eq{A}}}
    \and
    \infer[\rn{true}]{\J{\GD}{\GG}{\ms{true}}{\bool}}{}
    \and
    \infer[\rn{false}]{\J{\GD}{\GG}{\ms{false}}{\bool}}{}
    \and
    \infer[\rn{if}]{\J{\GD}{\GG}{\ifthen{e}{e_1}{e_2}}{A}}{
      \J{\GD}{\cdot}{e}{\bool} &
      \J{\GD}{\GG}{e_i}{A}}
    \and
    \infer[\rn{if^+}]{\J{\GD}{\GG}{\ifthen{e}{e_1}{\unit}}{L}}{
      \J{\GD}{\GG}{e}{\bool} &
      \J{\GD}{\GG}{e_1}{L}}
    %% set & semilattice rules
    \and
    \infer[\rn{\{\}}]{\J{\GD}{\GG}{\{e\}}{\Set{A}}}{\J{\GD}{\cdot}{e}{A}}
    \and
    \infer[\rn{\{:\}}]{\J{\GD}{\GG}{\singlemap{e_1}{e_2}}{\Map{A}{B}}}{
      \J{\GD}{\cdot}{e_1}{A} &
      \J{\GD}{\GG}{e_2}{B}}
    \and
    \infer[\unit]{\J{\GD}{\GG}{\unit}{L}}{}
    \and
    \infer[\rn{\vee}]{\J{\GD}{\GG}{e_1 \vee e_2}{L}}{\J{\GD}{\GG}{e_i}{L}}
    \and
    \infer[\rn{\bigvee}]{\J{\GD}{\GG}{\letin{x}{e_1}{e_2}}{L}}{
      \J{\GD}{\GG}{e_1}{\Set{A}} &
      \J{\GD,x\of A}{\GG}{e_2}{L}}
    \and
    \infer[\rn{fix}]{\J{\GD}{\GG}{\fix{x}{e}}{\fineq{L}}}{
      \J{\GD}{\GG,x\of L}{e}{\fineq{L}}}
    \and
    \infer[\rn{fix_{\le}}]{\J{\GD}{\GG}{\fixle{x}{e_1}{e_2}}{\eq{L}}}{
      \J{\GD}{\GG}{e_1}{\eq{L}} &
      \J{\GD}{\GG,x \of \eq{L}}{e_2}{\eq{L}}}
  \end{mathpar}

  \caption{Typing rules for core Datafun}
\end{figure*}


\section{Typing rules}


\section{Semantics}

\begin{figure}
  \todo{TODO: DIAGRAM HERE}
  \caption{Semantic categories of Datafun}
  \label{fig:sem-cats}
\end{figure}

\begin{figure}
  \begin{center}
    \begin{tabular}{cl}
      \multicolumn{2}{c}{\textbf{Set notation}}\\
      $\U{P}$ & Underlying set of the poset $P$\\
      $\stringset$ & Set of strings\\
      $A \boxtimes B$ & Cartesian product of sets $A$, $B$\\
      $A \boxplus B$ & Disjoint union of sets $A$, $B$\\
      $A \Arr B$ & Functions from set $A$ to set $B$
      \vspace{0.5em}\\
      \multicolumn{2}{c}{\textbf{Poset notation}}\\
      \two & Two-element poset $\{0,1\}$, with $0 < 1$\\
      $P + Q$ & Disjointly-ordered poset on disjoint union of $P,Q$\\
      $P \x Q$ & Pointwise poset on pairs of $P$s and $Q$s\\
      $P \arr Q$ & Pointwise poset on monotone maps $\ms{Poset}(P, Q)$\\
      $\Disc{A}$ & Discrete poset on underlying set $A$\\
      $\FM{A}{P}$ & Poset of finite maps from the set $A$ to the poset $P$
      \vspace{0.5em}\\
      \multicolumn{2}{c}{\textbf{Semilattice notation}}\\
      $L \lol M$ & Pointwise poset on $\ms{Semilat}(L,M)$\\
      $\Sups{P}$ & Free semilattice on a poset $P$\\
      $\FS{A}$ & Free semilattice on a set $A$; same as $\Sups{(\Disc{A})}$\\
    \end{tabular}
  \end{center}

  \caption{Semantic notation}
  \label{fig:sem-notation}
\end{figure}


We give a denotational semantics for Datafun in terms of three categories
(\ms{Set}, \ms{Poset}, and \ms{Semilat}) and two adjunctions between them (see
Figure \ref{fig:sem-cats}). We use nonstandard notation to avoid confusion
between sets and posets (see Figure \ref{fig:sem-notation}). We take less care
to distinguish posets and semilattices, since while a set can be partially
ordered in many ways, a poset either \emph{is} or \emph{is not} a semilattice.

\subsection{The category \ms{Semilat}}

The category \ms{Semilat} is the category of join-semilattices with least
elements (for brevity, simply ``semilattices'').

Directly, a semilattice is a poset $L$ with a least element $\unit$ in which any
two elements $a,b$ have a least-upper-bound $a \vee b$. From $\unit$ and $\vee$
it follows that any finite subset $X \subseteq_{\ms{fin}} \U{L}$ has a least
upper bound, written $\bigvee X$.

A morphism $f \in \ms{Semilat}(L, M)$ is a function from $\U{L}$ to $\U{M}$
satisfying:
\begin{eqnarray*}
  f(a \vee_A b) &=& f(a) \vee_B f(b)\\
  f(\unit_A) &=& \unit_B
\end{eqnarray*}

\ms{Semilat} is a subcategory of \ms{Poset}; every \ms{Semilat}-morphism $f$ is
monotone, since $a \le b \iff a \vee b = b$, and so from $a \le b$ we know $f(a)
\vee f(b) = f(a \vee b) = f(b)$, thus $f(a) \le f(b)$.

\subsection{Denotation of Datafun types}
\begin{figure}
  \[\begin{array}{rcl}
  \den{\cdot} &\in& \ms{Poset}_0\\
  \den{\bool} &=& \two\\
  \den{\N} &=& \N_\le\\
  \den{\str} &=& \Disc{\mathbb{S}}\\
  \den{A \x B} &=& \den{A} \x \den{B}\\
  \den{A + B} &=& \den{A} + \den{B}\\
  \den{A \mto B} &=& \den{A} \arr \den{B}\\
  \den{A \uto B} &=& \Disc{\U{\den{A}}} \arr \den{B}\\
  \den{\Set{A}} &=& \FS{\U{\den{A}}}\\
  \den{\Map{A}{B}} &=& \FM{\U{\den{A}}}{\den{B}}
  \end{array}\]
  \caption{Denotations of Datafun types}
  \label{fig:sem-types}
\end{figure}

Datafun types denote posets as shown in Figure \ref{fig:sem-types}. Datafun's
semilattice types denote posets which are also semilattices --- and moreover,
semilattices where the least-upper-bound operator $\bigvee$ is
\emph{computable}. \todo{should this have a proof?} \todo{EXTEND TO CONTEXTS}

\subsection{Denotation of Datafun terms}

We give a denotation for typing derivations with the following signature:
\begin{eqnarray*}
  \den{\J{\GD}{\GG}{e}{A}} &\in&
  \ms{Set}(\U{\den{\GD}}, \ms{Poset}(\den{\GG}, \den{A}))
  %% \\ &\in&
  %% \U{\den{\GD}} \Arr \U{\den{\GG} \arr \den{A}}
\end{eqnarray*}

Colloquially, $\J{\GD}{\GG}{e}{A}$ denotes a function from $\den{\GD} \x
\den{\GG}$ to $\den{A}$ that must be monotone in $\den{\GG}$ (but not in
$\den{\GD}$).


\section{Implementation}
We have built a proof-of-concept implementation of Datafun in Racket
\todo{CITE}. In addition to core Datafun, it supports pattern-matching, variant
types, record types, subtyping, antitone functions, and unbounded (potentially
nonterminating) fixed points. It performs no optimizations whatsoever.

\todo{LINK}


%% not sure how many sections to split this up into.
\section{Tradeoffs, limitations, and design decisions}

\paragraph{Finite and bounded fixed-points} \TODO Discuss
Datalog's constructor restriction (name?), not being sure how to encode it in a
type theory, its disadvantages, and the disadvantages of our approaches ---
finite \& bounded fixed-points.

bounded fixed-points strictly more general than finitary-type fixed-points, but
at possible runtime cost, but finite types very restrictive in practice

\TODO In particular, could bounded fix-points have adverse performance
implications?

\paragraph{Termination} \TODO nontermination and declarativeness.

For example, one powerful optimization technique is \emph{loop reordering} (in
SQL terminology, \emph{join reordering}), that is, taking advantage of the
equation
\begin{eqnarray*}
  \forin{x \in e_1} \forin{y \in e_2} e
  &=& \forin{y \in e_2} \forin{x \in e_1} e
\end{eqnarray*}
when $x,y \notin \ms{FV}(e_1) \cup \ms{FV}(e_2)$. (\TODO Explain why join
reordering is powerful). But this equation does not always hold in the presence
of nontermination; for example, if $e_1 = \unit$ and $e_2$ diverges. \TODO
finish up

\paragraph{Type inference} Typechecking needs to distinguish between ordinary
and monotone $\lambda$, application, \ms{case}, \ms{let}, and \ms{if}. In our
implementation we solve this in two ways:
\begin{enumerate}
\item Bidirectional type inference \todo{CITE} determines whether $\fn$s and
  applications are ordinary or monotone.
\item For $\ms{if}$, $\ms{case}$, and $\ms{let}$, the programmer annotates which
  form is intended; for example, $(\ifthen{e}{e_1}{\unit})$ is written
  (\texttt{when e then e1}) to indicate the rule $\ms{if}^+$ applies.
\end{enumerate}

It remains an open question how much annotation is necessary. See
Section \ref{sec:futurework}, \emph{Related and future work}.

\paragraph{User-defined usls}
The two fundamental usl types Datafun provides are booleans and sets; other type
constructors such as products, functions, and finite maps merely preserve usl
structure where they find it. One might contemplate allowing the programmer to
define their own usl structures using something like Haskell's
\texttt{newtype}/\texttt{instance}. Unfortunately, this \TODO
\begin{itemize}
\item User-defined functions give the compiler no guarantee that they are
  commutative, associative, and idempotent.
\item Without further information it would be impossible to
\end{itemize}

Unfortunately, there is no simple characterization of the space of all possible
computable usls.

\paragraph{Lexical types}
\todo{\begin{itemize}
\item Pro: allow expressing many things more generally, for example, map lookup
\item Con: have no good general monotone elimination rule
\item Con: complicate the type theory
\end{itemize}}


\section{Related and future work}
\label{sec:futurework}

\paragraph{Optimization}
\begin{itemize}
\item \TODO the datalog literature
\end{itemize}

\paragraph{Polymorphism}
\begin{itemize}
\item \TODO quantification over different classes of type variable (ordinary,
  equality, lattice); amounts to a typeclass system, so this is not new work.
\item \TODO tone polymorphism and why you need it for principal types; e.g.
  what is the type of $\fn\bind{f}\fn\bind{x} f\;x$?
\end{itemize}

\paragraph{Type inference} blah

\TODO Ref Dunfield \& Krishnaswami, higher-order bidirectional type inference.

\todo{REWRITE} We speculate that bidirectional inference could be replaced by a
Damas-Milner \todo{CITE} style algorithm, which infers a principal type for any
term without any annotation at all, \emph{if} we add polymorphism,
tone-polymorphism, and subtyping---so that, for example, $\fn\bind{f}\fn\bind{x}
f\;x$ can be assigned the principal type
$\forall\bind{o\of\ms{tone}}\forall\bind{\alpha,\beta \of \ms{type}} (\alpha
\overset{o}\to \beta) \mto (\alpha \overset{o}\to \beta)$, where
$\overset{o}\to$ indicates a function of tone $o$; a tone may be empty (for an
ordinary function) or ${+}$ for a monotone function.

%% ACKNOWLEDGMENTS
\acks

Acknowledgments, if needed.

% We recommend abbrvnat bibliography style.

\bibliographystyle{abbrvnat}
\bibliography{datafun}

%% \appendix
%% \section{Appendix Title}

%% This is the text of the appendix, if you need one.


\end{document}
