\documentclass[preprint]{sigplanconf}

% The following \documentclass options may be useful:

% preprint      Remove this option only once the paper is in final form.
% 10pt          To set in 10-point type instead of 9-point.
% 11pt          To set in 11-point type instead of 9-point.
% numbers       To obtain numeric citation style instead of author/year.

\usepackage{datafun}
%%\renewcommand{\todo}[1]{}


%% ---------- Setup ----------
\begin{document}

\special{papersize=8.5in,11in}
\setlength{\pdfpageheight}{\paperheight}
\setlength{\pdfpagewidth}{\paperwidth}

\conferenceinfo{ICFP '16}{18--24 September, 2016, Nara, Nara, Japan}
\copyrightyear{2016}
\copyrightdata{978-1-nnnn-nnnn-n/yy/mm}
\copyrightdoi{nnnnnnn.nnnnnnn}

% Uncomment the publication rights you want to use.
%\publicationrights{transferred}
%\publicationrights{licensed}     % this is the default
%\publicationrights{author-pays}


%% ---------- The title ----------
% These are ignored unless 'preprint' option specified.
\titlebanner{preprint}
\preprintfooter{Datafun: a Functional Datalog (PREPRINT)}

\title{Datafun: a Functional Datalog}
\subtitle{}

%% \title{Datafun}
%% \subtitle{A Functional Datalog}

%% \title{Datafun}
%% \subtitle{or, Datalog with datatypes}

%% \title{Datalog with Datatypes}
%% \subtitle{Toward a functional language of finite sets}

\authorinfo{}{}{}

%% \authorinfo{Michael Arntzenius\and Neelakantan R. Krishnaswami}
%%            {University of Birmingham}
%%            {daekharel@gmail.com, N.Krishnaswami@cs.bham.ac.uk}

\maketitle


%% ---------- The abstract ----------
\begin{abstract}
  Datalog may be considered either an unusually powerful query language or a
  carefully limited logic programming language. It has been successfully applied
  in a wide variety of problem domains thanks to hitting a ``sweet spot'' of
  expressivity, optimizability, and declarativeness. However, most use-cases
  require extending Datalog in an application-specific manner. In this paper we
  define Datafun, an analogue of Datalog supporting higher-order functional
  programming. The key idea is to \emph{track monotonicity via types}.

  \todo{Get Neel to suggest more stuff to put in abstract.}
\end{abstract}

% \category{CR-number}{subcategory}{third-level}
%
% % general terms are not compulsory anymore,
% % you may leave them out
% \terms
% term1, term2
%
% \keywords
% keyword1, keyword2


%% ---------- Paper body ----------
%% Section 1: Introduction
\section{Introduction}

The phrase ``declarative programming'' is as popular as it is ambiguous, with
seemingly hundreds of disparate senses in which it is used. However, two of
those usages stand out for popularity: both \emph{functional} and \emph{logic}
programming languages are generally deemed declarative languages. Despite this
shared epithet, the logic and functional programming traditions have largely
evolved independently of one another (with a few honorable exceptions such as
Mercury~\cite{mercury}, Curry~\cite{curry} and Kanren~\cite{kanren}). This could
be seen as an occasion for sorrow, but we prefer to view it as an opportunity:
as functional language designers, we can look to logic languages to discover new
ideas to steal.

A Prolog program can be understood as a collection of logical axioms formulated
as Horn clauses (i.e., first-order formulas of the form $\forall \vec{x}.\;P_1
\land \ldots \land P_n \to Q$, where $P_i$ and $Q$ are atomic formulas).
Execution of a Prolog program can be understood as running a proof search
algorithm on these clauses to figure out whether a particular formula is
derivable or not.

In other words, functional and logic programming languages embody the
Curry-Howard correspondence in two different ways. In a functional language,
types are propositions, terms are proofs, and program evaluation corresponds to
proof normalization. On the other hand, for logic programming languages,
\emph{terms} are propositions, and program evaluation corresponds to \emph{proof
  search}.

Since proof search is in general undecidable, designers of logic programming
languages must be careful both about the kinds of formulas they admit as
programs, and about the proof search algorithm they implement. Prolog offers a
very expressive language --- full Horn clauses --- and so faces an undecidable
proof search problem. Therefore, Prolog specifies its proof search strategy:
depth-first goal-directed/top-down search. This lets Prolog programmers reason
about the behaviour of their programs; however, it also means many logically
natural programs fail to terminate. Notoriously, transitive closure calculations
are much less elegant in Prolog than one might hope, since their most natural
specification is best computed with a bottom-up (aka ``forwards chaining'')
proof search strategy.

This view of Prolog suggests considering other possible design choices, such as
restricting the logical language so as to make proof search decidable. One of
the oldest such variants is Datalog~\cite{datalog}, a syntactic subset of Prolog
satisfying three restrictions:

\begin{enumerate}
\item Programs must be \emph{constructor-free}: only atomic terms and variables
  are permitted to appear as arguments to predicates. This ensures that deduction
  will not introduce terms that do not occur in the source of the
  program.

\item Clauses are \emph{range-restricted}: all variables in the
  consequent (head) of a clause must also occur positively in its
  premises (body).

\item Programs are limited to \emph{stratified negation}: the
  negation of a predicate may be used in a definition only if it has
  already been fully defined. That is, within the recursive
  definition of a predicate, it cannot be used in negated form.
\end{enumerate}

These restrictions make Datalog Turing-\emph{in}complete: all queries are
decidable. As functional programmers are well aware, though, there is power in
restraint: for example, in a total functional language, the compiler may switch
between strict and lazy evaluation at will. Similarly, in Datalog decidability
means that implementations are free to use forwards chaining, and so can easily
support queries (like reachability and transitive closure) which are difficult
to implement in ordinary Prolog.

%% PREVIOUS PARAGRAPH:
%
% Thanks to this decidability, Datalog implementations are free to
% tailor their proof search strategy to the program being executed.
% Consequently Datalog programs can be both concise and efficient. For
% example, \citet{whaley-lam} implemented pointer analysis algorithms in
% Datalog, and found that they could reduce their analyses from
% \emph{thousands} to \emph{tens} of lines of code while retaining
% competitive performance.

Over the last decade or so, this freedom has been put to good use, with Datalog
appearing at the heart of a a wild variety of applications in both research and
industry. For example, Whaley and Lam \cite{whaley-lam,whaley-phd} implemented
pointer analysis algorithms in Datalog, and found that they could reduce their
analyses from thousands of lines of C code to \emph{tens} of lines of Datalog
code, while retaining competitive performance. Semmle has developed the .QL
language~\cite{semmlecode,ql-inference} based on Datalog for analysing source
code (which was used to analyze the code for NASA's Curiosity Mars rover), and
LogicBlox has developed the LogiQL~\cite{logicblox} language for business
analytics. The Boom project at Berkeley has developed the Bloom language for
distributed programming~\cite{bloom}, and the Datomic cloud
database~\cite{datomic} uses Datalog (embedded in Clojure) as its query
language. Microsoft's SecPAL language~\cite{secpal} uses Datalog as the
foundation of its decentralised authorization specification language.

In all of these cases, the use of Datalog permits giving specifications and
implementations which are dramatically shorter and clearer than alternatives
implemented in more conventional languages. However, while all of these
applications are built on a foundation of Datalog, they all also extend it
significantly. For example, it is impossible even to implement arithmetic in
Datalog, since adding 2 and 3 produces 5, which is a new term not equal to
either 2 or 3! As a result, even though Datalog has a very clean semantics, its
metatheory needs to be re-established once again for each application-specific
extension to it.

As a result, it would be very desirable to understand what makes Datalog tick,
so that we can embed it into a more expressive language \emph{without}
sacrificing the properties that make it so powerful within its domain. In this
way, extensions can become ``a small matter of programming'', without having to
do a custom redesign of the language for each application.

%% TODO: cite the jargon file for SMOP?

In this paper, we present Datafun, a typed functional language which permits
programming in the style of Datalog, while still supporting the full power of
higher-order functional programming.

\paragraph{Contributions}
\begin{itemize}
\item We describe Datafun, a typed language capturing the expressive power of
  Datalog and extending it to support higher-order functional programming.
  Datafun's key feature is to \emph{track monotonicity with types}. This permits
  us to use typing to analyze fixed point computations in a way ensuring their
  termination.

\item We present examples illustrating the expressive power of Datafun,
  including relational-algebra-style operations, transitive closure, CYK
  parsing, and dataflow analysis. Some of these examples are familiar from
  Datalog, but many of them go well beyond what can be expressed in it,
  illustrating the benefits of our approach.

\item We identify the semantic structures underpinning Datalog, and
  use this to give a denotational semantics for Datafun in terms of a
  pair of adjunctions between \cSet{}, \cPoset{}, \cSL{}.

\item We have a prototype implementation of Datafun in Racket, which
  has been used to implement all of the examples in this paper, and is
  available at \texttt{[link omitted for double-blind review]}. 
\end{itemize}

%% Contributions (as summarized by Michael):
% - Datafun, like Datalog but functional
% - examples, incl. both datalog examples & things datalog can’t do
% - key ingredient is monotonicity; ``found'' semantics by analyzing
%   datalog: two adjunctions, three categories
% - prototype implementation

%% Contributions (as written by Neel):

% - We describe Datafun, a type theory for a language capturing the expressive
%   power of Datalog and extends it to both relax the constructor term
%   restriction and to support higher-order functional programming.

% - We give a variety of examples that illustrate the expressive power of
%   Datafun, such as CYK parsing, dataflow analysis, and transitive closure on
%   graphs, etc. Many of these examples are traditional examples of Datalog,
%   but we are also able to support things like first-class relations (eg,
%   generic transitive closure) and higher-order functions (example using
%   monotonicity and HO?). (doing a fix-point code analysis / parsing something
%   & dispatching on result?)

% - We identify the semantic structures underpinning Datalog, and use this to
%   give a denotational semantics for Datafun in terms of a pair of adjunctions
%   between Set, Poset, and the category of semilattices with finitary joins.

% - We have a prototype implementation of Datafun in Racket.

% Local Variables:
% TeX-master: "datafun"
% End:


%% Section 2: Datafun, informally
%% FIGURE: CORE SYNTAX
\begin{figure}
  \[\begin{array}{ccl}
    %% types
    A, B     &\bnfeq& \bool \pipe \N \pipe \str \pipe \Set{A}
    \pipe A + B \pipe A \x B\\
    \textsf{types} && A \uto B \pipe A \mto B
    \vspace{0.5em}\\
    %% semilattice types
    L, M         &\bnfeq& \bool \pipe \N \pipe \Set{A}
    \pipe L \x M \pipe A \uto L \pipe A \mto L\\
    \textsf{semilattice types}
    \vspace{0.5em}\\
    %% equality types
    \eq{A}, \eq{B} &\bnfeq& \bool \pipe \N \pipe \str \pipe \Set{\eq{A}}
    \pipe \eq{A} + \eq{B} \pipe \eq{A} \x \eq{B}\\
    \textsf{eqtypes} &&
    \vspace{0.5em}\\
    %% finite equality types
    \fineq{A},\fineq{B}
    &\bnfeq& \bool \pipe \Set{\fineq{A}}
             \pipe \fineq{A} + \fineq{B} \pipe \fineq{A} \x \fineq{B}\\
    \textsf{finite eqtypes}
    \vspace{0.5em}\\
    %% contexts
    \GD &\bnfeq& \cdot \pipe \GD, x\of A\\
    \GG &\bnfeq& \cdot \pipe \GG{},\m{x}\of A\\
    \textsf{contexts}
    \vspace{0.5em}\\
    %% expressions
    e &\bnfeq& x \pipe \m{x} %% \pipe n \pipe s
    \pipe \fn\bind{x} e \pipe \fn\bind{\m{x}} e
    \pipe e\;e\\
    \textsf{terms}
    && (e,e) \pipe \pi_1\;e \pipe \pi_2\;e
    \pipe \ms{in}_1\;e \pipe \ms{in}_2\;e\\
    && \case{e}{x}{e}{x}{e}\\
    && \case{e}{\m{x}}{e}{\m{x}}{e}\\
    && \ms{true} \pipe \ms{false} \pipe \ifthen{e}{e}{e}\\
    && \singleset{e} \pipe \unit \pipe e \vee e \pipe \letin{x}{e}{e}\\
    && \fix{\m{x}}{e} \pipe \fixle{\m{x}}{e}{e}
    %% \vspace{0.5em}\\
    %% x, \m{x} && \text{variables}\\
    %% n && \text{numerals}\\
    %% s && \text{string literals}
  \end{array}\]

  %% \todo{Remove $\N$ from semilattice types \& give it the discrete order? We
  %%   never use it as a semilattice type.}

  \caption{Syntax of core Datafun}
  \label{fig:syntax}
\end{figure}


\section{Datafun, Informally}
\label{sec:informally}

We give the core syntax of Datafun in Figure \ref{fig:syntax}. Datafun is a
simply-typed $\lambda$-calculus extended in four major ways:

\begin{enumerate}
\item We add a type of finite sets, $\Set{A}$.

  %% \todo{Describe why sets are useful?}

  %% We use finite sets to represent Datalog predicates; one might also think of
  %% them as tables or views in a database setting.

\item We add a type of \emph{monotone functions}, $A \mto B$. Consequently
  Datafun has two flavors of variable: ordinary variables, which we call
  \emph{discrete}, and \emph{monotone} variables. We write discrete variables in
  $script$ and monotone variables in \m{bold}.

  In order for ``monotone'' to have meaning, our types are implicitly partially
  ordered:
  \begin{itemize}
  \item Booleans $\bool$ are ordered $\ms{false} < \ms{true}$.
  \item Natural numbers $\N$ have the usual order: $0 < 1 < 2 < ...$.
  \item We have no particular use-case for comparing strings $\str$ in
    this paper, so we order them discretely; $a \le b$ iff $a = b$.
    %% \todo{Better explanation?}
  \item Pairs and functions are ordered pointwise:
    \begin{itemize}
    \item $(a, x) \le (b, y)$ iff $a \le b \wedge x \le y$
    \item $f \le g$ iff $\forall \bind{x} f\;x \le g\;x$
    \end{itemize}
  \item Sum types are ordered disjointly: $\ms{in}_i\; a \le
    \ms{in}_i\; b$ iff $a \le b$, but $\ms{in}_1\; a$ and $\ms{in}_2\; b$ are
    never comparable.
  \item Sets are ordered by inclusion: $a \le b$ iff $a \subseteq b$.
  \end{itemize}

  %% TODO: explain why fixed points are useful?
\item We add a term $(\fix{\m{x}}{e})$ denoting the least fixed point of the
  monotone function $(\fn\bind{\m{x}} e)$. This is computed (modulo
  optimizations) by iteration, starting from the smallest value of the desired
  type and halting once a fixed point is found. This strategy constrains the
  types of \ms{fix} terms in several ways:
  \begin{itemize}
  \item The type must have a smallest value. We enforce this using semilattice
    types (see item \ref{item:semilattice-types}, below).

  \item The type must support equality tests, to determine when a fixed point
    has been reached. We call a type supporting equality tests an \emph{eqtype}.

  \item To ensure termination, the type must have finite height.\footnote{The
    height of a poset is the cardinality of its largest chain (totally-ordered
    subset).} We conservatively approximate this property by limiting \ms{fix}
    to finite types.
  \end{itemize}

  In summary, \ms{fix} may only be used at \emph{finite semilattice eqtypes}.

  %% \todo{TODO: connection to Datalog via finiteness of predicates}

  %% \todo{explain $\fixle{\m{x}}{e_1}{e_2}$?}

\item\label{item:semilattice-types} Generalizing the empty set $\emptyset$ and
  union $\cup$, we identify a subset of types that have a \emph{least element}
  $\unit$ and a \emph{least upper bound} operator $\vee$. We call these
  \emph{semilattice types}\footnote{Technically, the partial orderings on these
    types form \emph{join-semilattices with a least element}. For brevity's
    sake, we call these structures simply ``semilattices.''}, and denote them by
  the metavariables $L,M$.

  Semilattice types serve two purposes. First, as already mentioned, they
  guarantee the presence of a least element, needed to compute \ms{fix} terms.

  %% \todo{Explain how products of semilattice being semilattice + monotone
  %%   fixed-points account for mutual recursion.}

  Second, they provide a natural eliminator for sets. Given $e_1 : \Set{A}$, we
  write $\letin{x}{e_1}{e_2}$ for the least upper bound, over all elements $x
  \in e_1$, of $e_2$, provided $e_2$ has some semilattice type $L$. If $e_2$ is
  a set, for example, this provides the set type's monadic ``bind'' operation.
  For example, $\forin{x \in \setlit{1,2,3}} \{10 \cdot x, x^2\}$ denotes the
  set $\{1, 4, 9, 10, 20, 30\}$.

\end{enumerate}


%% Section 3: examples
%% FIGURE: SYNTAX SUGAR
\begin{figure}
  \[\begin{array}{lccl}
  %% expressions
  \textsf{terms} &
  e &\bnfeq& ... \pipe \setlit{\vec{e}}
             \pipe \setfor{e}{\mc{L}}
             \pipe \forin{\mc{L}}{e}\\
  &&& \mathcal{C}\;\vec{e} \pipe \rawcase{e}{[{p} \cto {e}]^*}
  \vspace{0.5em}\\
  %% patterns
  %%
  %% TODO: maybe remove the pattern-matching stuff? since we don't explain how
  %% to translate it & we also use various other sugar we don't explain how to
  %% translate, why do we include only pattern-matching here?
  \textsf{patterns} &
  p &\bnfeq& \pwild \pipe x \pipe \bound{e} \pipe (p,p)
             \pipe \ms{true} \pipe \ms{false} \pipe \mathcal{C}\;\vec{p}
  \vspace{0.5em}\\
  \textsf{constructors} & \mathcal{C} && \text{are abstract identifiers}
  \vspace{0.5em}\\
  %% loop clauses
  \textsf{loops} &
  \mc{L} &\bnfeq& \mc{L}, \mc{L} \pipe p \in e \pipe e
  \end{array}\]

  %% the desugaring syntax-expansion itself
  \begin{eqnarray*}
    \setlit{} &\expandsto& \unit\\
    \setlit{e,\vec{e_i}} &\expandsto& \setlit{e} \vee \setlit{\vec{e_i}}\\
    \setfor{e}{\mc{L}}       &\expandsto& \forin{\mc{L}}{\{e\}}\\
    \forin{\mc{L}_1,\mc{L}_2}{e}
    &\expandsto& \forin{\mc{L}_1}{\forin{\mc{L}_2}{e}}\\
    \forin{p\in e_1}{e_2} &\expandsto&
    \letin{x}{e_1}{\rawcase{x}{p \cto e_2;\,\pwild \cto \unit}}\\
    \forin{e_1}{e_2} &\expandsto& \ifthen{e_1}{e_2}{\unit}
    %% \ms{let}~x = e_1 ~\ms{in}~ e_2
    %% &\expandsto& (\fn\bind{x} e_2)\; e_1\\
    %% \ms{let}~[x_i = e_i]^* ~\ms{in}~ e
    %% &\expandsto& [\ms{let}~x_i = e_i~\ms{in}]^* e\\
    %% \rawcase{e}{[p \cto e]^*} &\expandsto& \text{(omitted, see \todo{CITE})}
  \end{eqnarray*}
  \caption{Syntax sugar}
  \label{fig:sugar}
\end{figure}


\section{Examples}

For purposes of these examples, we use a simple Haskell-like syntax for
top-level type and function definitions. We also permit ourselves infix
notation, \ms{let}-binding, $n$-ary tuples, $n$-ary sum types with named
constructors, pattern-matching (including non-linear patterns), and additional
syntax sugar given in Figure \ref{fig:sugar}. All of these conveniences are
supported (with slightly different concrete syntax) in our implementation.

%% \todo{(TODO: mention \& cite monadic query syntax)}

For clarity, we set the names of top-level variables in \textsf{sans-serif};
ordinary variables in $script$ or \mi{italic} (for long variable names); and
monotone variables in \m{bold}.

Although Datafun as presented does not have polymorphism, we give our examples
their most general possible type schemes.

%% IDEAS FOR MORE EXAMPLES:
%%
%% \begin{itemize}
%% \item \texttt{make}-style topological sort?
%% \item SQL-style examples? SQL vs Datalog vs Datafun?
%% \item translating relational algebra into datafun?
%% \end{itemize}


\subsection{Filtering, mapping, and cross products}

Armed with the syntactic sugar given in Figure \ref{fig:sugar}, basic set
operations such as map, filter, and cross-product are easy first examples:
\[\begin{array}{l}
\fname{map} ~:~ (A \uto B) \uto \Set{A} \mto \Set{B}\\
\fname{map}\;f\;\m{A} = \setfor{f\;x}{x \in \m{A}}\\
\\
\fname{filter} ~:~ (A \uto \bool) \mto \Set{A} \mto \Set{A}\\
\fname{filter}\;\m{f}\;\m{A} = \setfor{x}{x \in \m{A}, \m{f}\; x}\\
\\
(\times) ~:~ \Set{A} \mto \Set{B} \mto \Set{A \x B}\\
\m{A} \times \m{B} = \setfor{(a,b)}{a \in \m{A}, b \in \m{B}}
\end{array}\]

Worth noting here are the subtleties of monotonicity typing. For example,
\ms{map} is not monotone in its function argument, while \fname{filter} is.
Recalling that sets are ordered by inclusion, this is straightforward enough ---
observe, for example, that:
\begin{eqnarray*}
 \fname{map}\;(\le 0)\;\setlit{0,1}
 &\not\subseteq& \fname{map}\;(\le 1)\;\setlit{0,1}\\
 \fname{filter}\;(\le 0)\;\setlit{0,1}
 &\subseteq& \fname{filter}\;(\le 1)\;\setlit{0,1}
\end{eqnarray*}

However, it is perhaps unclear how Datafun's type system ``knows''
\fname{filter} is monotone in \m{f}. In brief, Datafun knows that application
$(\m{f}\;x)$ is monotone in the function, and moreover, testing a boolean guard
$(\m{f}\;x)$ in a set-comprehension such as $\setfor{x}{x \in \m{A}, \m{f}\;x}$
is monotone in the guard expression. A fuller explanation is given in Section
\ref{sec:typing-rules}.

%% Consider the desugaring of \ms{filter}:
%% \[\begin{array}{l}
%% \fname{filter}\;\m{f}\;\m{A} = \forin{x \in \m{A}}
%% \ifthen{\m{f}\;x}{\singleset{x}}{\unit}
%% \end{array}\]
%% \fname{filter}'s type asserts that it uses \m{f} monotonically. This in turn
%% requires that $(\ifthen{\m{f}\;x}{\setlit{x}}{\unit})$ increases monotonically
%% as the value of \m{f} increases.


%% FIGURE: PRIMITIVES
\begin{figure}
  %% TODO: remove unused primitives.
  \[\begin{array}{cll}
  \neg &\of& \bool \uto \bool\\
  =   &\of& \eq{A} \uto \eq{A} \uto \bool\\
  \le &\of& \eq{A} \uto \eq{A} \mto \bool\\
  %% \fname{keys}     &:& \Map{A}{B} \mto \Set{A}\\
  %% \fname{entries}  &:& \Map{A}{B} \uto \Set{A \x B}\\
  %% \fname{tabulate} &:& \Set{A} \mto (A \uto B) \mto \Map{A}{B}\\
  %% \fname{getWith}  &:& \Map{\eq{A}}{B} \mto \eq{A} \uto (B \mto L) \mto L\\
  %% \fname{get}      &:& \Map{\eq{A}}{L} \mto \eq{A} \uto L\\
  %% \fname{substrings} &\of& \ms{Str} \uto \Set{\ms{Str}}\\
  %% \fname{size}     &:& \Set{\eq{A}} \mto \N\\
  \fname{range}    &:& \N \uto \N \mto \Set{\N}\\
  \fname{length}   &:& \str \uto \N\\
  \fname{substring} &:& \str \uto \N \uto \N \uto \str
  \end{array}\]
  \caption{Primitive functions and their type schemes}
  \label{fig:primitives}
\end{figure}


\subsection{Equality, membership, and intersection}

So long as the type of a set's elements supports equality, we can test whether
the set contains a value $x$ as follows:
\[\begin{array}{l}
(\isin) ~:~ \eq{A} \uto \Set{\eq{A}} \mto \bool\\
x \isin \m{A} = \forin{y \in \m{A}} x = y
\end{array}\]

The expression $\forin{y \in \m{A}} x = y$ takes the least upper bound, at
boolean type, for every $y \in \m{A}$, of the value of $x = y$. Since booleans
are ordered $\ms{false} < \ms{true}$, ``least upper bound'' is simply logical
disjunction! %% In plain English, this code says ``a set \m{A} contains an element
%% $x$ if some element $y \in \m{A}$ tests equal to $x$''.

Similarly, we can define set intersection by testing for equality:
\[\begin{array}{l}
(\cap) ~:~ \Set{\eq{A}} \mto \Set{\eq{A}} \mto \Set{\eq{A}}\\
\m{A} \cap \m{B} = \setfor{x}{x \in \m{A}, y \in \m{B}, x = y}
\end{array}\]

\todo{FIXME: rewrite to talk about dot-patterns instead of nonlinear
  pattern-matching}

However, explicitly testing for equality can become tedious, so we usually use
\emph{nonlinear pattern-matching} instead --- that is, we bind the same-named
variable multiple times, which indicates it must have an equal value at each
occurrence:
\[\begin{array}{l}
(\cap) ~:~ \Set{\eq{A}} \mto \Set{\eq{A}} \mto \Set{\eq{A}}\\
\m{A} \cap \m{B} = \setfor{x}{x \in \m{A}, \bound{x} \in \m{B}}
\end{array}\]

This is merely syntax sugar for an equality test, so the condition that the
set's element type support equality remains in force.

%% \todo{TODO: mention you can extend to all relational algebra? give example?}


\subsection{Composition of relations}

One extremely useful operator it is convenient to define using nonlinear pattern
matching is composition of finite relations (that is, sets of pairs):
\[\begin{array}{l}
(\bullet) : \Set{A \x \eq{B}} \mto \Set{\eq{B} \x C} \mto \Set{A \x C}\\
\m{R} \bullet \m{S} = \setfor{(a,c)}{(a,b) \in \m{R}, (\bound{b},c) \in \m{S}}
\end{array}\]

This already demonstrates a capability Datafun has that Datalog does not:
defining operators over relations. A Datalog program defining binary predicates
\texttt{r} and \texttt{s} which wished to compose those predicates would have to
define a new top-level predicate:

\begin{verbatim}
r(X,Y) :- (...).
s(X,Y) :- (...).
rs(A,C) :- r(A,B), s(B,C).
\end{verbatim}

In Datafun, we simply define $(\bullet)$ and use it inline as needed. We shall
see the use of this in later examples.

%% \todo{work phrase ``higher-order'' in here somewhere?}


\subsection{Transitive closure}

Consider the following Datalog program, authored perhaps by a J.R.R. Tolkien
aficionado wishing to trace the geneologies of their favorite work, \textit{The
  Silmarillion}:
\begin{verbatim}
parent(earendil, elrond).
parent(elrond, arwen).
ancestor(X, Y) :- parent(X, Y).
ancestor(X, Z) :- ancestor(X, Y), ancestor(Y, Z).
\end{verbatim}

%% \todo{Discuss how this works in Datalog, but not in Prolog, b/c Prolog is
%%   defined by operational semantics of unification while Datalog is denotational,
%%   least-model semantics. It also works in Datafun!}

%% \todo{Neel suggests using distinction b/w backward \& forward chaining here,
%%   rather than operational/denotational. see Logical Algorithms paper by
%%   McAllister \& co for phrasing?}

This defines a binary \texttt{parent} relation, along with its transitive
closure, \texttt{ancestor}. The Datafun equivalent is:
\[\begin{array}{l}
\mathbf{data}~\ms{person} =
\ctor{E\"arendil} ~|~ \ctor{Elrond} ~|~ \ctor{Arwen}\\
\fname{parent},~\ms{ancestor} ~:~ \Set{\ms{person} \x \ms{person}}\\
\ms{parent} =
\{(\ctor{E\"arendil}, \ctor{Elrond}), (\ctor{Elrond}, \ctor{Arwen})\}\\
\ms{ancestor} = \fix{\m{X}} \ms{parent} \vee (\m{X} \bullet \m{X})
%% \setfor{(a,c)}{(a,b) \in \m{X}, (b,c) \in \m{X}}
\end{array}\]

The type \ms{person} represents the domain of our \ms{parent} and \ms{ancestor}
relations. \ms{parent} is simply a list of parent-child pairs. \ms{ancestor} is
where the action is at: since the Datalog predicate \texttt{ancestor} is defined
recursively, \ms{ancestor} is defined as a least fixed point --- in this case,
of the the equation
\begin{equation*}
  \m{X} = \ms{parent} \vee (\m{X} \bullet \m{X})
\end{equation*}
Informally, we may read this as stating that a pair is in \m{X} if it is in
either \ms{parent} or the composition of \m{X} with itself. This requires that
\m{X} contain the transitive closure of \ms{parent}. And since we take the
\emph{least} fixed point of this equation, \ms{ancestor} contains \emph{exactly}
the transitive closure of \ms{parent}. Voil\`a!


\subsubsection{Transitive closure with an upper bound}

The preceding explanation glosses over a critical requirement: \ms{fix} may only
be used at \emph{finite semilattice eqtypes}. \ms{ancestor} has type
$\Set{\ms{person} \x \ms{person}}$. Does this suffice? It's certainly a
semilattice, since it's a set type. Since \ms{person} is effectively a sum of
units, it supports equality, and sets and products of eqtypes are themselves
eqtypes. Likewise, \ms{person} is finite, and products and sets of finite types
are themselves finite.

So! We find ourselves in the clear, for now. However, the restriction of
\ms{fix} to finite types can be quite limiting in practice. So Datafun provides
a more general way to take a fixed-point: specify an \emph{upper bound} which
the fixed point may not exceed. For this we write $(\fixle{\m{x}}{e_\top} e)$,
where $e_\top$ is our upper bound.

Suppose, for example, we represent our \textit{dramatis personae} as strings (an
infinite type). We may write:
\[\begin{array}{l}
\ms{persons} ~:~ \Set{\str}\\
\ms{persons} = \{\texttt{"e\"arendil"}, \texttt{"elrond"}, \texttt{"arwen"}\}\\
\ms{parent}, \ms{ancestor} ~:~ \Set{\str \x \str}\\
\ms{parent} = \{(\texttt{"e\"arendil"}, \texttt{"elrond"}),
(\texttt{"elrond"}, \texttt{"arwen"})\}\\
\ms{ancestor} = \fixle{\m{X}}{(\ms{persons} \x \ms{persons})}
\ms{parent} \vee (\m{X} \bullet \m{X})
\end{array}\]

Instead of a \ms{person} type, we have \ms{person} \emph{set}, which we use to
construct an upper bound on our fixed-point: $(\ms{person} \x \ms{person})$, the
complete binary relation. Since every string in \ms{parent} is also in
\ms{person}, the transitive closure of \ms{parent} cannot exceed this upper
bound.

However, this invariant is left to the programmer to check. What if a sloppy
programmer should mistakenly include a person in \ms{parent} not present in
\ms{person}? More generally, what if the fixed point $(\fixle{\m{x}}{e_\top} e)$
is trying to compute exceeds $e_\top$? (Or indeed, no such fixed point exists?)

In that case, the value of $(\fixle{\m{x}}{e_\top} e)$ is \emph{clamped} to the
upper bound $e_\top$. This ensures Datafun programs terminate even in the
presence of sloppy programmers, and although they may not have the value you
expect, that value is at least predictable.


\subsubsection{Generic transitive closure}

Thus far we have only considered taking the transitive closure of a relation we
have already defined. But consider: for any finite eqtype $\fineq{A}$, we may
write:
\[\begin{array}{l}
\ms{trans} ~:~ \Set{\fineq{A} \x \fineq{A}} \mto \Set{\fineq{A} \x \fineq{A}}
\vspace{0.3em}\\
%% \ms{trans}\ E = \fix{X} E \vee \setfor{(a,c)}{(a,b) \in E, (b,c) \in X}
%% \ms{trans}\ \m{E} = \fix{\m{X}} \m{E} \vee %
%% \setfor{(a,c)}{(a,b) \in \m{X}, (b,c) \in \m{X}}
\ms{trans}\ \m{E} = \fix{\m{X}} \m{E} \vee (\m{X} \bullet \m{X})
\end{array}\]
Similarly, for any eqtype $\eq{A}$, we may write:
\[\begin{array}{l}
\ms{trans} ~:~
\Set{\eq{A}} \mto \Set{\eq{A} \x \eq{A}} \mto \Set{\eq{A} \x \eq{A}}
\vspace{0.3em}\\
%% \ms{trans}\ \m{V}\ \m{E} = %
%% \ms{fix}~ \m{S} \le \setfor{(a,b)}{a\in \m{V}, b \in \m{V}}\\
%% \hspace{5.35em}\ms{is}~ \m{E} \vee %
%% \setfor{(a,c)}{(a,b) \in \m{S}, (b,c) \in \m{S}}\\
\ms{trans}\ \m{V}\ \m{E} = %
\fixle{\m{S}}{(\m{V} \x \m{V})} \m{E} \vee (\m{S} \bullet \m{S})
\end{array}\]

In this way, we can abstract away from choice of underlying relation and define
transitive closure generically. Using functions as a means of abstraction is of
course familiar and unremarkable to functional programmers, but it is simply not
possible in Datalog.


\subsection{CYK parsing}
Parsing can be understood logically, with a parse tree representing a
proof that a certain string belongs to a language described by a
context-free grammar. As a result, it is possible to formulate parsing
in terms of proof search~\cite{deductive-parsing}. One of the
simplest algorithms for parsing context free grammars is the
Cocke-Younger-Kasami (CYK) algorithm for parsing with grammars in
Chomsky normal form.\footnote{In Chomsky normal form, each production
  is of the form $A \to B \cdot C$ or $A \to \vec{a}$, with $A,B,C$
  ranging over nonterminals, and $\vec{a}$ over nonempty strings of
  terminals.}  Given a grammar $G$, we begin by introducing a family
of predicates (sometimes called \emph{facts} or \emph{items}) $A(i,j)$,
with one $A$ for each nonterminal, and $i$ and $j$ representing
indices into a string. Given a word $w$, we write $w[i,n]$ for the
$n$-element substring of $w$ beginning at position $i$. Then, we can
specify the CYK algorithm with the following two inference rules:

\begin{mathpar}
  \inferrule*{B(i, j) \\ C(j, k) \\ (A \to B\; C) \in G}
             {A(i, k)}
  \and
  \inferrule*{ (A \to \vec{a}) \in G \\ w[i,n] = \vec{a} }
             {A(i,i+n)}
\end{mathpar}
Then, the predicate $A(i,j)$ means that $A$ is derivable from the
substring of $w$ running from $i$ to $j$, and so the whole word $w$ is
derivable from the start symbol $S$ if $S(0, \mathit{length}\;w)$ is
derivable.

In Datafun, this rule-based description of the algorithm can be
transliterated almost directly into code. We begin by introducing a
few basic types.
\[\begin{array}{l}
\mathbf{type}~\ms{sym} = \str\\
\mathbf{data}~\ms{rule} = \ctor{String}~\str ~|~ \ctor{Concat}~\ms{sym}~\ms{sym}\\
\mathbf{type}~\ms{grammar} = \Set{\ms{sym} \x \ms{rule}}\\
\mathbf{type}~\ms{fact} = \ms{sym} \x \N \x \N\\
\end{array}\]
The $\ms{sym}$ type is a type synonym representing nonterminal names
with strings. The $\ms{rule}$ type is the type of the right-hand-sides
of productions in Chomsky normal form -- either a string, or a pair of
nonterminals. A $\ms{grammar}$ is just a set of productions -- a set
of pairs of nonterminals paired with their rules. The type $\ms{fact}$
is the type representing the atomic facts derived by the CYK inference
system -- they are triples of the rulename, the start position, and
the end position.

With these types in hand, we can write the CYK algorithm as a fixed
point computation. In fact, it is convenient to break it into two
pieces, by first defining the function whose fixed point we take. So
we can write down the $\fname{iter}$ function, which represents one step of
the fixed point iteration.
\[\begin{array}{l}
\fname{iter} ~:~ \str \uto \ms{grammar} \mto \Set{\ms{fact}} \mto \Set{\ms{fact}}\\
\fname{iter} \;\mi{text} \;\m{G} \;\m{chart} =\\
\hspace{1em}\phantom{\vee~}
\{(a,i,k) ~|~ (a, \ctor{Concat}~b~c) \in \m{G},\\
\hspace{6.25em} (\bound{b},i,j) \in \m{chart},\\
\hspace{6.25em} (\bound{c},\bound{j},k) \in \m{chart}\}\\
\hspace{1em}\vee~ \{(a,i,i+\fname{length}\;s)\\
\hspace{2.1em}|~ (a, \ctor{String}~s) \in \m{G},\\
\hspace{2.2em}\phantom{|~} i \in \fname{range}\;0\;(n-\fname{length}\;s),\\
\hspace{2.2em}\phantom{|~}
s = \fname{substring} \;\mi{text} \;i \;(i+\ms{length}\;s)\}
\end{array}\]
This function works by taking a string $\mi{text}$ and a grammar $\m{G}$, and
then taking a set of facts $\m{chart}$, and taking a union. The first clause is
a set comprehension, saying that we return $(a, i, k)$ if $(b, i, j)$ and $(c,
j, k)$ are in $\m{chart}$ -- this corresponds to applications of the first rule.
The second clause corresponds to the second rule above, saying that $(a, i, i +
\ms{length}\;s)$ is a generated fact if $s$ is a substring of $\mi{text}$ at
position $i$.

We can then use $\fname{iter}$ to implement the $\fname{parse}$ function.
%% parse
\[\begin{array}{l}
\fname{parse} ~:~ \str \uto \ms{grammar} \mto \Set{\ms{sym}}\\
\fname{parse} \;\mi{text} \;\m{G} =\\
\hspace{1em} \ms{let}~ n = \ms{length} \;\mi{text}\\
\hspace{2.375em}\m{bound} =
  \{(a,i,j) ~|~ (a,\pwild) \in \m{G},\\
\hspace{10.5em}i \in \ms{range}\;0\;n, \\
\hspace{10.5em}j\in\fname{range}\;i\;n\}\\
\hspace{2.375em} \m{chart} = \fixle{\m{C}}{\m{bound}}
  \ms{iter} \;\mi{text} \;\m{G} \;\m{C}\\
\hspace{1em}\ms{in}~\setfor{a}{(a, 0, \bound{n}) \in \m{chart}}\\
%% %% iter with \forin
%% \\
%% \fname{iter} \;\mi{text} \;\m{G} \;\m{chart} =\\
%% \hspace{1em}\phantom{\vee~}
%% (\bigvee((a, \ctor{Concat} \;b \;c) \in \m{G},\\
%% \hspace{1.25em}\phantom{\vee~ \bigvee(}
%% (b,i,j) \in \m{chart}, (c,j,k) \in \m{chart})\\
%% \hspace{1.25em}\phantom{\vee~}\, \setlit{(a,i,k)})\\
%% \hspace{1em}\vee~ (\bigvee((a, \ctor{String} \;s) \in \m{G},\\
%% \hspace{1.25em}\phantom{\vee~\bigvee(}
%% i \in \ms{range} \;0 \;(n - \ms{length} \; s),\\
%% \hspace{1.25em}\phantom{\vee~\bigvee(}
%% s = \ms{substring} \;\mi{text} \;i \;(i+\ms{length}\;s))\\
%% \hspace{1.25em}\phantom{\vee~}\, \setlit{(a,i,i+ \ms{length}\;s)})
%% \\
%% %% iter with case. I like this version best.
%% \\
%% \fname{iter} \;\mi{text} \;\m{G} \;\m{chart} =\\
%% \hspace{1em}\forin{(a,r) \in \m{G}}\\
%% \hspace{1.875em}\ms{case}~ r\\
%% \hspace{2.4em}\ms{of}~
%% %% \hspace{3.05em}\pipe
%% \ctor{Concat} \;b \;c \cto \{(a,i,k) ~|~ (b,i,j) \in \m{chart},\\
%% \hspace{14.42em}(c,j,k) \in \m{chart}\}\\
%% \hspace{3.05em}\pipe \ctor{String} \;s \cto
%% \{\,(a, i, i+\ms{length}\;s)\\
%% \hspace{9em}|~ i \in \ms{range} \;0 \;(n-\ms{length}\;s),\\
%% \hspace{9em}\phantom{|~}
%% s = \ms{substring} \;\mi{text} \;i \;(i+\ms{length}\;s)\}
%% \\
%% iter. Neel prefers this. People know set-comprehension.
\end{array}\]
This function just takes the fixed point of $\fname{iter}$ --
almost. Because facts are triples $\ms{sym} \x \N \x \N$, sets of
facts may in general grow unboundedly.  To ensure termination, we
construct a set $\m{bound}$ to bound the sets of facts we consider in
our fixed point computation, by bounding the symbols to names found in
the grammar $\m{G}$, and the indices to positions of the string. Since
all of these are finite, we know that the computation of $\m{chart}$
as a bounded fixed point will terminate. Then, having computed the
fixed point, we can check chart to see if $(a, 0, \ms{length}\;\mi{text})$
is derivable.

There are three things worth noting about this program. First, it is
not expressible in Datalog. Because Datalog provides no way to
represent a \emph{grammar} as a piece of data (it's compound, not an
atom), there is simply no way in Datalog to express a \emph{generic}
parser taking a grammar as an input. This demonstrates one of the key
benefits of moving to a functional language like Datafun.

Moreover, Datalog programs must be \emph{constructor-free}, to ensure all
relations are finite. Primitives such as \ms{range} and \ms{substring} violate
this restriction (as relations, they are infinite); it is not immediately
obvious that Datalog programs extended with these primitives remain terminating.
Our use of bounded fixed-points to guarantee termination is robust under such
extensions; as long as all primitive functions are total, Datafun programs
always terminate.

Finally, having computed a set via a fixed point, we can test whether
or not an element is in that set \emph{or not} -- the ability to test
for negative information after the fixed point computation completes
corresponds to a use of stratified negation in Datalog.


\subsection{Dataflow analysis}
In this section, we show how some simple dataflow analyses can be expressed in
Datafun. We begin with the types in these programs.
\[\begin{array}{l}
\textbf{type}~\ms{var} = \str\\
\textbf{type}~\ms{label} = \N\\
\textbf{data}~\ms{oper} = \ctor{Eq} \pipe \ctor{Le}
\pipe \ctor{Add} \pipe \ctor{Sub} \pipe\ctor{Mul}\pipe\ctor{Div}\\
\textbf{data}~\ms{atom} = \ctor{Var}\;\ms{var} \pipe \ctor{Num}\;\N\\
\textbf{data}~\ms{expr} = \ctor{Atom}\;\ms{atom}
\pipe \ctor{Apply}\;\ms{oper}\;\ms{atom}\;\ms{atom}\\
\textbf{data}~\ms{stmt} =
\ctor{Assign} \;\ms{var} \;\ms{expr}
\pipe \ctor{If} \;\ms{expr} \;\ms{label}\;\ms{label} \\
\textbf{type}~\ms{program} = \Set{\ms{label} \x \ms{stmt}}
\end{array}\]
The basic idea is that we represent a program as a kind of control
flow graph. Each node of this graph has a $\ms{label}$, which is a
natural number, and contains a statement of type $\ms{stmt}$, which is
either an assignment of an expression (of type $\ms{expr}$) to a
variable (of type $\ms{var}$), or a conditional jump.  A program is
then just the set of nodes -- i.e., a set of label, statement pairs --
with the invariant that the relation is functional (i.e., if $(l, s)$
and $(l,s')$ are both in a program, then $s = s'$).

In what follows, we use a few trivial functions whose definitions are omitted
for space reasons.
\[\begin{array}{l}
%% omitted functions
\ms{labels} ~:~ \ms{program} \uto \Set{\ms{label}}\\
\ms{vars} ~:~ \ms{program} \uto \Set{\ms{var}}\\
\ms{uses} ~:~ \ms{stmt} \uto \Set{\ms{var}}\\
\ms{defines} ~:~ \ms{stmt} \uto \Set{\ms{var}}
\end{array}\]
The $\ms{labels}$ function returns the set of labels in a program. The
$\ms{vars}$ function returns the set of variables used in a program (both in
expressions and as targets for assignments). The $\ms{uses}$ function
returns the set of variables used by the expressions in a statement. The
$\ms{defines}$ function returns the set of variables defined by a statement
(i.e., at most one variable -- the target of the assignment).

Given a program, we define the 1-step control flow graph with the $\ms{flow}$
function.
\[\begin{array}{l}
%% control flow
%% TODO: use long variable name for argument.
\textbf{type}~\ms{flow} = \Set{\ms{label} \x \ms{label}}\\
\fname{flow} ~:~ \ms{program} \uto \ms{flow}\\
\fname{flow}\;c = \forin{(i,s) \in c}\\
\hspace{4em}\ms{case}~ s ~\ms{of}~
\ctor{If} \;\pwild \;j \;k \cto \setlit{(i,j),(i,k)}\\
\hspace{7.45em}\pipe\pwild \cto \setfor{(i,i+1)}{i+1 \isin \ms{labels}\;c}
\end{array}
\]
It says that if $(i, s)$ is a node of the program, then if $s$ is a conditional
jump $\ctor{If} \;\pwild \;j \;k$, then control can flow from $i$ to $j$, and
from $i$ to $k$ -- i.e., we add both $(i, j)$ and $(i, k)$ to the set of edges.
Otherwise, it's an assignment, and control flows to the next statement (i.e., we
add $(i, i+1)$ to the set of edges).

Now, we can define liveness analysis, one of the classic ``backwards'' dataflow
analyses. The type of $\ms{live}$ say that given a program and its flow graph,
it returns a set of label/variable pairs, which determine a relation saying
for each label which variables are live.
%% live code analysis
\[\begin{array}{l}
\ms{live} ~:~ \ms{program} \uto \ms{flow} \uto \Set{\ms{label} \x \ms{var}}\\
\ms{live} \;\mi{code} \;\mi{flow} =\\
\hspace{2em} \fixle{\m{Live}}{ %
  \ms{labels}\;\mi{code} \x \ms{vars}\;\mi{code}}\\
\hspace{2em}\forin{(i,\mi{stmt}) \in \mi{code}}\\
\hspace{2.875em} (\phantom{\vee~}\setfor{(i,v)}{v \in \ms{uses}\;\mi{stmt}}\\
\hspace{3.2em} \vee~ \{(i,v) ~|~ (\bound{i},j) \in \mi{flow},\\
\hspace{7.4em}(\bound{j},v) \in \m{Live},\\
\hspace{7.4em}\neg (v \isin \ms{defines}\; \mi{stmt})\})
\end{array}\]
For a statement $\mi{stmt}$ at label $i$, we say that the variable
$v$ is live at $i$ if $v$ is used by $\mi{stmt}$. The variable $v$
is also live at $i$ if control flows from $i$ to $j$, and and $v$
is live at $j$, assuming that $\mi{stmt}$ isn't a definition site for $v$.

When computing this analysis, we again need to use a bounded fixed
point, which we do by taking the Cartesian product of the labels and
variables occuring in the program.


Next, we give one of the classic forwards dataflow analyses,
reaching definitions. This analysis is used to figure out whether
an assignment (a ``definition'') can influence the value of later
expressions or not.
%% reaching definitions analysis
\[\begin{array}{l}
\ms{reachingDefinitions} ~:~ \ms{program} \uto \ms{flow}
\uto \Set{(\ms{label} \x \ms{var}) \x \ms{label}}\\
\ms{reachingDefinitions} \;\mi{code} \;\mi{flow} =\\
\hspace{2em}\fixle{\m{RD}}{%
  (\ms{labels}\;\mi{code} \x \ms{vars}\;\mi{code}) \x \ms{labels}\;\mi{code} }\\
\hspace{2em}\forin{(i,\mi{stmt}) \in \mi{code}}\\
\hspace{2.875em} (
\phantom{\vee~}\setfor{((i,v), i)}{v \in \ms{defines}\;\mi{stmt}}\\
\hspace{3.2em} \vee~ \{((l,v), i) ~|~ (j,\bound{i}) \in \mi{flow},\\
\hspace{8.95em}((l,v), \bound{j}) \in \m{RD},\\
\hspace{8.95em}\neg(v \isin \ms{defines}\;\mi{stmt})\})
\end{array}\]
We define a function $\fname{reachingDefinitions}$ which takes a
program and a set of flows as arguments, and returns a relation of
type $\Set{(\ms{label} \x \ms{var}) \x \ms{label}}$. An entry $((l,v),
i)$ in this relation means the definition of $v$ at $l$ reaches program
point $i$.

This is then computed as a fixed point of two clauses. First, if there
is a definition $v$ at program point $i$, then $i$ is reached by that
definition. Second, if $(l,v)$ reaches $j$, and $j$ flows to $i$, then
$(l,v)$ reaches $i$ as long as $v$ is not re-defined at $i$.

As \citet{whaley-lam} observed, Datalog makes it very easy to express
dataflow analyses, and it is similarly easy in Datafun.


%% Section 4: Typing rules
%% FIGURE: Typing rules
\begin{figure*}
  %% \boxed{\ensuremath{\mathsz{10pt}{\J{\Delta}{\Gamma}{e}{A}}}}
  \begin{mathpar}
    \infer[\rn{\rt{var}}]{\J{\GD}{\GG}{x}{A}}{x\of A \in \GD}
    \and
    \infer[\rn{\rt{var}^+}]{\J{\GD}{\GG}{\m{x}}{A}}{\m{x}\of A \in \GG}
    %% function rules
    \and
    \infer[\rn{\fn}]{\J{\GD}{\GG}{\fn\bind{x} e}{A \uto B}}{
      \J{\GD,x\of A}{\GG}{e}{B}}
    \and
    \infer[\rn{\rt{app}}]{\J{\GD}{\GG}{e_1\;e_2}{B}}{
      \J{\GD}{\GG}{e_1}{A \uto B} &
      \J{\GD}{\cdot}{e_2}{A}}
    \and
    \infer[\rn{\monofn}]{\J{\GD}{\GG}{\fn\bind{x} e}{A \mto B}}{
      \J{\GD}{\GG,x \of A}{e}{B}}
    \and
    \infer[\rn{\rt{app}^+}]{\J{\GD}{\GG}{e_1\;e_2}{B}}{
      \J{\GD}{\GG}{e_1}{A \mto B} &
      \J{\GD}{\GG}{e_2}{A}}
    %% product & sum rules
    \and
    \infer[\rn{\rt{pair}}]{\J{\GD}{\GG}{(e_1,e_2)}{A_1 \x A_2}}{
      \J{\GD}{\GG}{e_i}{A_i}}
    \and
    \infer[\rn{\pi}]{\J{\GD}{\GG}{\pi_i\;e}{A_i}}{\J{\GD}{\GG}{e}{A_1 \x A_2}}
    \and
    \infer[\rn{\rt{in}}]{\J{\GD}{\GG}{\ms{in}_i\;e}{A_1 + A_2}}{
      \J{\GD}{\GG}{e}{A_i}
    }
    \and
    \infer[\rn{\rt{case}}]{\J{\GD}{\GG}{\case{e}{x}{e_1}{x}{e_2}}{C}}{
      \J{\GD}{\cdot}{e}{A_1 + A_2} &
      \J{\GD,x\of A_i}{\GG}{e_i}{C}}
    \and
    \infer[\rn{\rt{case}^+}]{
      \J{\GD}{\GG}{\case{e}{\m{x}}{e_1}{\m{x}}{e_2}}{C}
    }{
      \J{\GD}{\GG}{e}{A_1 + A_2} &
      \J{\GD}{\GG,\m{x}\of A_i}{e_i}{C}
    }
    %% boolean rules
    %% \and
    %% \infer[\rn{=}]{\J{\GD}{\GG}{e_1 = e_2}{\bool}}{\J{\GD}{\GG}{e_i}{\eq{A}}}
    \and
    \infer[\rn{\rt{true}}]{\J{\GD}{\GG}{\ms{true}}{\bool}}{}
    \and
    \infer[\rn{\rt{false}}]{\J{\GD}{\GG}{\ms{false}}{\bool}}{}
    \and
    \infer[\rn{\rt{if}}]{\J{\GD}{\GG}{\ifthen{e}{e_1}{e_2}}{A}}{
      \J{\GD}{\cdot}{e}{\bool} &
      \J{\GD}{\GG}{e_i}{A}}
    \and
    \infer[\rn{\rt{if}^+}]{\J{\GD}{\GG}{\ifthen{e}{e_1}{\unit}}{L}}{
      \J{\GD}{\GG}{e}{\bool} &
      \J{\GD}{\GG}{e_1}{L}}
    %% set & semilattice rules
    \and
    \infer[\unit]{\J{\GD}{\GG}{\unit}{L}}{}
    \and
    \infer[\rn{\vee}]{\J{\GD}{\GG}{e_1 \vee e_2}{L}}{\J{\GD}{\GG}{e_i}{L}}
    \and
    \infer[\rn{\{\}}]{\J{\GD}{\GG}{\{e\}}{\Set{A}}}{\J{\GD}{\cdot}{e}{A}}
    \and
    \infer[\rn{\bigvee}]{\J{\GD}{\GG}{\letin{x}{e_1}{e_2}}{L}}{
      \J{\GD}{\GG}{e_1}{\Set{A}} &
      \J{\GD,x\of A}{\GG}{e_2}{L}}
    %% \and
    %% \infer[\rn{\{:\}}]{\J{\GD}{\GG}{\singlemap{e_1}{e_2}}{\Map{A}{B}}}{
    %%   \J{\GD}{\cdot}{e_1}{A} &
    %%   \J{\GD}{\GG}{e_2}{B}}
    \and
    \infer[\rn{\rt{fix}}]{\J{\GD}{\GG}{\fix{\m{x}}{e}}{\fineq{L}}}{
      \J{\GD}{\GG,\m{x}\of L}{e}{\fineq{L}}}
    \and
    \infer[\rn{\rt{fix}_{\le}}]{
      \J{\GD}{\GG}{\fixle{\m{x}}{e_1}{e_2}}{\eq{L}}
    }{
      \J{\GD}{\GG}{e_1}{\eq{L}} &
      \J{\GD}{\GG,\m{x} \of \eq{L}}{e_2}{\eq{L}}}
  \end{mathpar}

  \caption{Typing rules for core Datafun}
  \label{fig:typing-rules}
\end{figure*}


\section{Typing rules}
\label{sec:typing-rules}

Datafun's typing judgment $\J{\GD}{\GG}{e}{A}$ is defined by the inference rules
given in Figure \ref{fig:typing-rules}. We gloss $\J{\GD}{\GG}{e}{A}$ as
follows: ``expression $e$ has type $A$ using variables from $\GD \cup \GG$, and
moreover the value of $e$ is \emph{monotone} with respect to the variables in
$\GG$''.

The context $\GD$ types ordinary variables; $\GG$, monotone variables. Both
admit the usual structural rules of exchange, weakening, and contraction.
Variables from either context may be used freely (rules \rt{var}, $\rt{var}^+$).
\todo{explain?}

\subsection{Functions and application}
Two function types require two function introduction rules: the ordinary
$\lambda$ and the monotone $\lambda^+$. These simply introduce variables into
their respective contexts. Monotone function application $\rt{app}^+$ is
perfectly standard, but ordinary function application \rt{app} has a pecularity:
the argument $e_2$ gets an \emph{empty} monotone context.

To understand why, recall our gloss: the application $e_1\;e_2$ must be monotone
in $\GG$. But $e_1$ is an ordinary, and in general \emph{non-monotone}, function
$A \uto B$: there is no guarantee that it respect any order on its argument.
(Suppose, for example, $e_2$ were some monotone variable $\m{x} : A \in \GG$.)
We work around this scoff-law behavior on $e_1$'s part by ensuring its argument
$e_2$ is \emph{constant} with respect to $\GG$---which we accomplish by simply
prohibiting $e_2$ from using any of $\GG$'s variables.

This technique of \emph{wiping clean} the monotone context to guarantee
constancy\footnote{Wherever we write ``constant'' in this section, substitute
  ``constant with respect to the monotone context''. The ordinary context is
  never ``wiped clean'', and behaves entirely as it would in a simply-typed
  $\lambda$-calculus.} of a subterm recurs in several other rules. Readers
familiar with judgmental formulations of modal logics of necessity such as
\citet{jrml} may notice a feeling of \textit{d\'ej\`a vu}; indeed, there is a
hidden comonad at work here. But we are getting ahead of ourselves. For more on
that, turn to Section \ref{sec:semantics}.

\todo{Get Neel to provide more citations here?}

\todo{Mention comonads and modal logic?}

\subsection{Products and sums}
The pairing and projection rules, \rt{pair} and $\pi$, are completely standard,
as is the \rt{in} rule for sum introduction. Sum elimination, however, splits
into two rules, \rt{case} and $\rt{case}^+$. $\rt{case}^+$ requires its branches
to be monotone in the variable $\m{x}$ it introduces, and consequently its
subject $e$ is permitted access to the monotone context $\GG$. \rt{case},
however, analyses its subject $e$ as a constant --- wiping clean its monotone
context --- and thus is allowed to introduce the variable $x$ into the ordinary
context $\GD$. \todo{rewrite for clarity}

\subsection{Booleans}
\label{sec:typing-rules-booleans}

While \rt{true} and \rt{false} are straightforward, there are two rules for
boolean elimination, \rt{if} and $\rt{if}^+$. This is because in Datafun, $1$
plus $1$ does not equal $\bool$: booleans are \emph{not} a sum of
units.\footnote{For simplicity, we have omitted the unit type 1 from our
  presentation of Datafun, but it is easy enough to imagine including it.} At
the type $1 + 1$, $\ms{in}_1 \triv$ and $\ms{in}_2 \triv$ are incomparable. But
in Datafun, $\ms{true} > \ms{false}$. Therefore, to eliminate a boolean in a
monotone fashion, one must ensure one's \emph{then}-branch is always greater
than one's \emph{else}-branch.

Thus Datafun has two \ms{if} rules. First, \rt{if}, where the boolean $e$ being
analysed is constant (has an empty monotone context), and so the branches $e_1$,
$e_2$ may be arbitrary expressions.

Second, $\rt{if}^+$, where the subject $e$ has full access to $\GG$, but the
\ms{if}-expression must have \emph{semilattice type}, and the \emph{else}-branch
is constrained to be $\unit$ --- the least value, thus smaller than $e_1$.

This is a conservative approach: there are many semantically monotone, but
untypeable, \ms{if}-terms. However, it is complete for semilattice types, for in
that case $(\ifthen{e}{e_1}{e_2})$ may be rewritten $(e_2 \vee
\ifthen{e}{e_1}{\unit})$; as long as $e_1 \ge e_2$ and so $e_2 \vee e_1 = e_1$,
this will not change the meaning of the expression, only (potentially) its
execution efficiency.

Thus the only meaningful restriction here is to semilattice types. In practice,
we have yet to find a case where this is problematic.

\todo{Mention that $e_1 \ge e_2$ is in general statically undecidable (rice's
  theorem), and that a typesystem to check it would amount to dependent typing?}

\subsection{Semilattices and sets}
The semilattice $\unit$ and $\vee$ operations are typed by their
correspondingly-named rules. As $\vee$ is monotone, its arguments have full
access to the monotone context $\GG$.

Recall that sets are ordered by inclusion: although $2 \le 3$, nonetheless
$\singleset{2} \not\le \singleset{3}$. For this reason the rule $\{\}$ for
constructing a singleton set $\singleset{e}$ wipes clean its element $e$'s
monotone context. Datafun does not need empty-set or union operators, since
$\unit$ and $\vee$ generalize them.

Finally, we come to $\bigvee$, the set-comprehension rule. This rule has the
flavor of a monadic ``bind'' operation, but generalized to a result of any
semilattice type. This operation is naturally monotone both in the set $e_1$
being iterated over and in the expression $e_2$ which we are taking the least
upper bound of. Moreover, $e_2$ is \emph{not} required to be monotone in the
elements $x \in e_1$. \todo{Why not?}

%% \subsection{Maps}
%% Finite maps are an abstract type, and most of their interface is exposed through
%% primitive functions (see Figure \ref{fig:primitives}). The only typing rule is
%% $\{:\}$, the introduction rule for singleton maps. By analogy with the
%% introduction rule for singleton sets $\{\}$,

\subsection{Fixed points}

\todo{explain \rt{fix}, $\rt{fix}_\le$}


%% Section 5: Semantics
\begin{figure}
   \tikzset{
   no line/.style={draw=none,
     commutative diagrams/every label/.append style={/tikz/auto=false}}}
\begin{center}
{\large
   \begin{tikzcd}
        \mathbf{\mathbf{Set}}     \arrow[bend left=35]{r}[name=F]{\mathsf{Disc}}
                                  \arrow[rr, bend left=60, "\mathsf{FS}"]
      & \mathbf{\mathbf{Poset}}   \arrow[bend left=35]{l}[name=U]{\U{-}}
                                  \arrow[to path={(F) -- (U)\tikztonodes}, no line]{}{\bot}
                                  \arrow[bend left=35]{r}[name=H]{\slF}
      & \mathbf{\mathbf{SemiLat}} \arrow[bend left=35]{l}[name=K]{\slU}
                                  \arrow[to path={(H) -- (K)\tikztonodes}, no line]{}{\bot}
   \end{tikzcd}
}
\end{center}
  \caption{Semantic categories of Datafun}
  \label{fig:sem-cats}
\end{figure}

\begin{figure}
  \begin{center}
    \begin{tabular}{cl}
      %% \multicolumn{2}{c}{\textbf{Set notation}}\\
      $\U{P}$ & Underlying set of the poset $P$\\
      $\stringset$ & Set of strings\\
      %% $A \boxtimes B$ & Cartesian product of sets $A$, $B$\\
      %% $A \boxplus B$ & Disjoint union of sets $A$, $B$\\
      %% $A \Arr B$ & Functions from set $A$ to set $B$
      %% \vspace{0.5em}\\
      %% \multicolumn{2}{c}{\textbf{Poset and semilattice notation}}\\
      \one & One-element poset $\{\triv\}$\\
      \two & Two-element poset $\{\sff,\stt\}$, with $\sff < \stt$\\
      $\N_\le$ & The naturals $\N$, as a (totally ordered) poset\\
      $P + Q$ & Disjointly-ordered poset on disjoint union of $P,Q$\\
      $P \x Q$ & Pointwise poset on pairs of $P$s and $Q$s\\
      $P \arr Q$ & Pointwise poset on monotone maps $\cPoset(P, Q)$\\
      %% $L \lol M$ & Pointwise poset on $\cSL(L,M)$\\
      $\slF\;P$ & Free semilattice on a poset $P$\\
      $\slU\;{L}$ & Underlying poset of a semilattice $L$\\
      $\Disc{A}$ & Discrete poset on underlying set $A$\\
      $\FS{A}$ & Free semilattice on a set $A$; same as $\slF\;(\Disc{A})$\\
      $\below{x}{P}$ &
      The sub-poset of $P$ below $x$: $\{y \in P ~|~ y \le x\}$
      %% \\ $\FM{A}{P}$ & Poset of finite maps from the set $A$ to the poset $P$
    \end{tabular}
  \end{center}

  \caption{Semantic notation}
  \label{fig:sem-notation}
\end{figure}

%% \todo{``Finite maps'' ambiguous terminology? Gibbons uses it differently, for
%%   example.}


\section{Semantics and metatheory}
\label{sec:semantics}

%% \todo{Somewhere in this we should talk about the comonad $\Disc{\U{\cdot}}$, the
%%   monad $\slU\;(\FS{\U{\cdot}})$, and adjoint logic.}

%% \todo{Should we talk about how $\Disc{\U{A + B}} = \Disc{\U{A} \boxplus \U{B}} =
%%   \Disc{\U{A}} + \Disc{\U{B}}$ and what this has to do with \ms{case}? Honestly
%%   I don't remember that exactly myself.}

We give a denotational semantics for Datafun in terms of three categories
(\cSet{}, \cPoset{}, and \cSL{}) and two adjunctions between them (see Figure
\ref{fig:sem-cats}). We present the notation we use in Figure
\ref{fig:sem-notation}; we take care to distinguish between sets and posets, and
since posets are more central to our semantics, most of our notation concerns
them. We take less care to distinguish posets and semilattices, since while a
set can be partially ordered in many ways, a poset either \emph{is} or \emph{is
  not} a semilattice.

\subsection{The category \cSL{}}

\cSL{} is the category of join-semilattices with least elements, which we call
simply ``semilattices''.

Directly, a semilattice is a poset $L$, with a least element $\unit$, in which
any two elements $a,b$ have a least-upper-bound $a \vee b$. From $\unit$ and
$\vee$ it follows that any finite subset $X \subseteq_{\ms{fin}} \U{L}$ has a
least upper bound, written $\bigvee X$.

A morphism $f \in \cSL(L, M)$ is a function from $\U{L}$ to $\U{M}$ satisfying:
\begin{eqnarray*}
  f(a \vee_A b) &=& f(a) \vee_B f(b)\\
  f(\unit_A) &=& \unit_B
\end{eqnarray*}

\cSL{} is a subcategory of \ms{Poset}; every \cSL{}-morphism $f$ is monotone,
since $a \le b \iff a \vee b = b$, and so from $a \le b$ we know $f(a) \vee f(b)
= f(a \vee b) = f(b)$, thus $f(a) \le f(b)$. Since it is a subcategory, we will
typically not explicitly write the forgetful functor $\mathsf{U}(L)$ which sends
semilattices to posets by forgetting the lattice structure.


\subsection{Denotation of Datafun types}
\begin{figure}
  \[\begin{array}{rcl}
  \den{A} &\in& \cPoset_0\\
  \den{\bool} &=& \two\\
  \den{\N} &=& \N_\le\\
  \den{\str} &=& \Disc{\mathbb{S}}\\
  \den{A \x B} &=& \den{A} \x \den{B}\\
  \den{A + B} &=& \den{A} + \den{B}\\
  \den{A \mto B} &=& \den{A} \arr \den{B}\\
  \den{A \uto B} &=& \Disc{\U{\den{A}}} \arr \den{B}\\
  \den{\Set{A}} &=& \FS{\U{\den{A}}}%% \\
  %% \den{\Map{A}{B}} &=& \FM{\U{\den{A}}}{\den{B}}
  %% \end{array}\]\[\begin{array}{rcl}
  \\\\
  \den{\GD}, \den{\GG} &\in& \cPoset_0\\
  \den{\cdot} &=& \one\\
  \den{\GD, x\of A} &=& \den{\GD} \x \den{A}\\
  \den{\GG{}, \m{x}\of A} &=& \den{\GG} \x \den{A}
  \end{array}\]
  \caption{Denotations of Datafun types and contexts}
  \label{fig:sem-types}
\end{figure}

Datafun types and contexts denote posets as shown in Figure \ref{fig:sem-types}.
To complete our semantics, we will need a few simple lemmas about the
denotations of Datafun types. First, we need to know that our semilattice types
are semilattices, and that our finite types are finite:

\begin{lemma}
  The denotation $\den{L}$ of a semilattice type $L$ is a semilattice.
\end{lemma}

\begin{lemma}
  The poset $\den{\fineq{A}}$ denoted by a finite eqtype $\fineq{A}$ is finite.
\end{lemma}

Second, to show that bounded fixed-points $(\fixle{\m{x}}{e_\top}{e})$
terminate, we need any possible $e_\top$ to pick out a finite-height sub-poset:

\begin{lemma}
  For any semilattice equality type $\eq{L}$, for any $x \in \den{\eq{L}}$, the
  height of $\below{x}{\den{\eq{L}}}$ is finite.
\end{lemma}

\paragraph{}
All of these are trivial to prove by induction over types.


%% TERM DENOTATION FIGURE
\newcommand{\fux}[2]{\Den{\vcenter{\infer{#1}{#2}}}}
%% \newcommand{\fuxn}[3]{\Den{\vcenter{\infer[\rn{#1}]{#2}{#3}}}}
\newcommand{\dg}{\;\delta\;\gamma}

\begin{figure*}
  \[\begin{array}{rcll}
  \textbf{Derivation}\;\phantom{\dg} && \textbf{Denotation}
  \vspace{.8em}\\
  \den{\J{\GD}{\GG}{e}{A}}\;\phantom{\dg} &\in&
  \cSet(\U{\den{\GD}},\,\cPoset(\den{\GG}, \den{A}))
  %% \U{\den{\GD}} \Arr \U{\den{\GG} \arr \den{A}}
  \vspace{.8em}\\
  \fux{\J{x_1\of A_1, ..., x_n\of A_N}{\GG}{x_i}{A_i}}{
    \phantom{.}}\dg
  &=& \pi_i\;\delta
  \vspace{.8em}\\
  \fux{\J{\GD}{\m{x}_1\of L_1, ..., \m{x}_n\of L_n}{\m{x}_i}{L_i}}{
    \phantom{.}}\dg
  &=& \pi_i\;\gamma
  \vspace{.8em}\\

  %% function rules
  \fux{\J{\GD}{\GG}{\fn\bind{x} e}{A \uto B}}{
    \J{\GD,x\of A}{\GG}{e}{B}}\dg
  &=& x \mapsto \den{e}\;\tuple{\delta,x}\;\gamma
  \vspace{.8em}\\
  \fux{\J{\GD}{\GG}{\fn\bind{\m{x}} e}{A \mto B}}{
    \J{\GD}{\GG,\m{x}\of A}{e}{B}}\dg
  &=& x \mapsto \den{e} \;\delta \;\tuple{\gamma,x}
  \vspace{.8em}\\
  \fux{\J{\GD}{\GG}{e_1\;e_2}{B}}{
    \J{\GD}{\GG}{e_1}{A \uto B} &
    \J{\GD}{\cdot}{e_2}{A}} \dg
  &=& \den{e_1}\dg\;(\den{e_2}\;\delta\;\triv)
  \vspace{.8em}\\
  \fux{\J{\GD}{\GG}{e_1\;e_2}{B}}{
    \J{\GD}{\GG}{e_1}{A \mto B} &
    \J{\GD}{\GG}{e_2}{B}} \dg
  &=& \den{e_1}\dg\;(\den{e_2}\dg)
  \vspace{.8em}\\

  %% product types
  \fux{\J{\GD}{\GG}{(e_1, e_2)}{A_1 \x A_2}}{
    \J{\GD}{\GG}{e_i}{A_i}}\dg
  &=& \pair{\den{e_1}\dg}{\den{e_2}\dg}
  \vspace{.8em}\\
  \fux{\J{\GD}{\GG}{\pi_i\;e}{A_i}}{
    \J{\GD}{\GG}{e}{A_1 + A_2}}\dg
  &=& \pi_i\;(\den{e}\dg)
  \vspace{.8em}\\

  %% sum type rules
  \fux{\J{\GD}{\GG}{\ms{in}_i\;e}{A_1 + A_2}}{
    \J{\GD}{\GG}{e}{A_i}}\dg
  &=& \ms{in}_i\;(\den{e}\dg)
  \vspace{.8em}\\
  \fux{\J{\GD}{\GG}{\case{e}{x}{e_1}{x}{e_2}}{B}}{
    \J{\GD}{\cdot}{e}{A_1 + A_2} &
    \J{\GD,x\of A_i}{\GG}{e_i}{B}}
  \dg
  &=&
  \begin{cases}
    \den{e_1}\;\pair{\delta}{x}\;\gamma
    &\text{if }\den{e}\;\delta\;\triv = \ms{in}_1\;x\\
    \den{e_2}\;\pair{\delta}{x}\;\gamma
    &\text{if }\den{e}\;\delta\;\triv = \ms{in}_2\;x\\
  \end{cases}
  \vspace{.8em}\\
  \fux{\J{\GD}{\GG}{\case{e}{\m{x}}{e_1}{\m{x}}{e_2}}{B}}{
    \J{\GD}{\GG}{e}{A_1 + A_2} &
    \J{\GD}{\GG,\m{x}\of A_i}{e_i}{B}
  }\dg
  &=&
  \begin{cases}
    \den{e_1}\;\delta\;\pair{\gamma}{x}
    &\text{if }\den{e} \dg = \ms{in}_1\;x\\
    \den{e_2}\;\delta\;\pair{\gamma}{x}
    &\text{if }\den{e} \dg = \ms{in}_2\;x\\
  \end{cases}
  \vspace{.8em}\\

  %% boolean rules
  \fux{\J{\GD}{\GG}{\ms{true}}{\bool}}{\phantom{.}}\dg
  &=& \stt
  \vspace{.8em}\\
  \fux{\J{\GD}{\GG}{\ms{false}}{\bool}}{\phantom{.}}\dg
  &=& \sff
  \vspace{.8em}\\
  \fux{\J{\GD}{\GG}{\ifthen{e}{e_1}{e_2}}{A}}{
    \J{\GD}{\cdot}{e}{\bool} &
    \J{\GD}{\GG}{e_i}{A}} \dg
  &=&
  \begin{cases}
    \den{e_1}\dg & \text{if}~ \den{e}\;\delta\;\triv = \stt\\
    \den{e_2}\dg & \text{if}~ \den{e}\;\delta\;\triv = \sff
  \end{cases}
  \vspace{.8em}\\

  \fux{\J{\GD}{\GG}{\ifthen{e}{e_1}{\unit}}{L}}{
    \J{\GD}{\GG}{e}{\bool} &
    \J{\GD}{\GG}{e_1}{L}} \dg
  &=&
  \begin{cases}
    \den{e_1}\dg & \text{if}~ \den{e}\dg = \stt\\
    \unit_{\den{L}} & \text{if}~ \den{e}\dg = \sff
  \end{cases}
  \vspace{.8em}\\

  %% semilattice rules
  \fux{\J{\GD}{\GG}{\unit}{L}}{\phantom{.}}\dg
  &=& \unit_{\den{L}}
  \vspace{.8em}\\
  \fux{\J{\GD}{\GG}{e_1 \vee e_2}{L}}{
    \J{\GD}{\GG}{e_i}{L}}\dg
  &=& \den{e_1}\dg \vee_{\den{L}} \den{e_2}\dg
  \vspace{.8em}\\

  %% set intro/elim rules
  \fux{\J{\GD}{\GG}{\singleset{e}}{\Set{A}}}{\J{\GD}{\cdot}{e}{A}}\dg
  &=& \{\den{e}\;\delta\;\triv\}
  \vspace{.8em}\\
  \fux{\J{\GD}{\GG}{\letin{x}{e_1}{e_2}}{L}}{
    \J{\GD}{\GG}{e_1}{\Set{A}} &
    \J{\GD,x\of A}{\GG}{e_2}{L}}\dg
  &=& \displaystyle\bigvee \left\{
  \den{e_2}\;\tuple{\delta,x}\;\gamma
  ~|~ {x \in \den{e_1}\dg}\right\}
  \vspace{.8em}\\

  %% fix rules
  \fux{\J{\GD}{\GG}{\fix{\m{x}}{e}}{\fineq{L}}}{
    \J{\GD}{\GG,\m{x}\of \fineq{L}}{e}{\fineq{L}}
  }\dg
  &=&
  \lfpin{\den{\fineq{L}}}{(x \mapsto \den{e}\;\delta\;\pair{\gamma}{x})}
  \vspace{0.8em}\\
  \fux{\J{\GD}{\GG}{\fixle{\m{x}}{e_1}{e_2}}{\eq{L}}}{
    \J{\GD}{\GG}{e_1}{\eq{L}} &
    \J{\GD}{\GG,\m{x}\of \eq{L}}{e_2}{\eq{L}}}\dg
  &=&
  \lfpin{\below{\den{e_1}\dg}{\den{\eq{L}}}}{
    \left(x \mapsto
    %% \ms{clamp}(\den{e_2}\;\delta\;\pair{\gamma}{x}, \den{e_1} \dg)
    \begin{cases}
      \den{e_2}\;\delta\;\pair{\gamma}{x}
      %% & \text{if}~\den{e_2}\;\delta\;\pair{\gamma}{x} \le \den{e_1}\dg\\
      & \text{if it's} \le \den{e_1}\dg\\
      \den{e_1} \dg & \text{otherwise}
    \end{cases}
    \right)}
  \end{array}\]

  \caption{Denotations of Datafun typing derivations}
  \label{fig:sem-terms}
\end{figure*}


\subsection{Denotation of Datafun terms}

In Figure \ref{fig:sem-terms} we give a denotation for typing derivations with
the following signature:
\begin{eqnarray*}
  \den{\J{\GD}{\GG}{e}{A}} &\in&
  \cSet(\U{\den{\GD}}, \cPoset(\den{\GG}, \den{A}))
  %% \\ &\in&
  %% \U{\den{\GD}} \Arr \U{\den{\GG} \arr \den{A}}
\end{eqnarray*}

Colloquially, $\J{\GD}{\GG}{e}{A}$ denotes a function from $\den{\GD} \x
\den{\GG}$ to $\den{A}$ that must be monotone in $\den{\GG}$ (but not in
$\den{\GD}$).

Our semantics requires the following lemma regarding fixed-points of monotone
functions:

\begin{lemma}[Fixed points in finite-height pointed posets]
  Any monotone map $f : P \to P$ on a poset $P$ of finite height with a least
  element $\unit$ has a least fixed point of the form $f^n(\unit)$.
\end{lemma}

\begin{proof}
  Consider the sequence $\unit, f(\unit), f^2(\unit), f^3(\unit), ...$. Note that
  $\unit \le f(\unit)$, so by monotonicity of $f$ and induction $f^i(\unit) \le
  f^{i+1}(\unit)$. Thus this sequence forms an ascending chain. Since $P$ has
  finite height, this chain cannot be infinite; thus there is an $n$ such that
  $f^n(\unit) = f^{n+1}(\unit)$, i.e. $f^n(\unit)$ is a fixed-point of $f$.

  Now consider any fixed-point $x$ of $f$. Since $\unit \le x$, by monotonicity of
  $f$, induction, and $x = f(x)$, we have $f^n(\unit) \le x$. Thus $f^n(\unit)$ is
  the least fixed point of $f$.
\end{proof}

We write $(\lfpin{L}{f})$ for the least fixed point of a monotone map $f$ on a
semilattice $L$ of finite height.

%% \paragraph{Lemma 1: Existence of fixed points in posets of finite height.}
%% Any monotone map $f : A \to A$ on a nonempty poset $A$ of finite height has
%% at least one fixed point.

%% \paragraph{Lemma 2: Finding least fixed points in pointed posets.} For any
%% poset $A$ with a least element $\unit$, for any monotone map $f : A \to A$,
%% if $f$ has a fixed point $x$ of finite height\footnote{The height of an
%% element $x$ in a poset $A$ is the height of the sub-poset $\{y \in A ~|~ y
%% \le x\}$.}, then $f$ has a least fixed point of the form $f^n(\unit)$ for
%% some $n \in \N$.


%% FIGURE: SUBSTITUTIONS
\begin{figure*}
  \[\begin{array}{rcll}
    \sub{e/v} v &=& e\\
    \sub{e/v} x &=& x\\
    \sub{e/v} (\fn\bind{u} e') &=& \fn\bind{u} \sub{e/v} e'\\
    \sub{e/v} (e_1 \;e_2) &=& (\sub{e/v} e_1)\;(\sub{e/v} e_2)\\
    \sub{e/v} (e_1, e_2) &=& (\sub{e/v} e_1, \sub{e/v} e_2)\\
    \sub{e/v} (\pi_i\;e') &=& \pi_i\;(\sub{e/v}e')\\
    \sub{e/v} (\ms{in}_i\;e') &=& \ms{in}_i \;(\sub{e/v} e')\\
    \sub{e/v} (\case{e'}{u}{e_1}{u}{e_2})
    &=& \case{\sub{e/v} e'}{u}{\sub{e/v} e_1}{u}{\sub{e/v} e_2}\\
    \sub{e/v} \unit &=& \unit\\
    \sub{e/v} (e_1 \vee e_2) &=& \sub{e/v} e_1 \vee \sub{e/v} e_2\\
    \sub{e/v} (\forin{x \in e_1} e_2)
    &=& \forin{x \in \sub{e/v} e_1} \sub{e/v} e_2\\
    \sub{e/v} (\fix{\m{x}} e') &=& \fix{\m{x}} \sub{e/v} e'\\
    \sub{e/v} (\fixle{\m{x}}{e_1}{e_2}) &=& \fixle{\m{x}}{\sub{e/v}{e_1}} e_2
  \end{array}\]
  \caption{Substitution}
  \label{fig:substitution}
\end{figure*}


\subsection{Metatheory}

We define substitution of Datafun terms in Figure \ref{fig:substitution}. We
could define separate forms of ordinary $\sub{e_1/x} e$ and monotone
$\sub{e_1/\m{x}} e$ substitution, but to avoid repetition we instead use the
metavariables $v,u$ to stand for any variable (ordinary or monotone) and define
$\sub{e_1/v} e$. Terms are taken up to $\alpha$-equivalence, and by convention,
whenever two variables are given distinct names $v,u$ it is assumed $v \ne u$.

We have proven the following theorems:

\begin{theorem}[Weakening and exchange]
  The rules \begin{mathpar}
    \infer[\rn{\rt{weak}}]{
      \J{\GD,\GD'}{\GG,\GG'}{e}{A}
    }{
      \J{\GD}{\GG}{e}{A}}
    \and
    \infer[\rn{\rt{xchg}}]{
      \J{\GD_1,\GD_2}{\GG_1,\GG_2}{e}{A}
    }{
      \J{\GD_2,\GD_1}{\GG_2,\GG_1}{e}{A}
    }
  \end{mathpar}
  are admissible.
\end{theorem}

\begin{theorem}[Substitution, ordinary]
  From
  \begin{itemize}
  \item $\J{\GD}{\cdot}{e_1}{A}$,
  \item and $\J{\GD,x \of A}{\GG}{e_2}{B}$,
  \end{itemize}
  it follows that
  \begin{itemize}
  \item $\J{\GD}{\GG}{\sub{e_1/x}e_2}{B}$,
  \item and $\den{\sub{e_1/x} e_2} \dg = \den{e_2} \;\pair{\delta}{\den{e_1}\dg}
    \;\gamma$.
  \end{itemize}
\end{theorem}

\begin{theorem}[Substitution, monotone]
  From
  \begin{itemize}
  \item $\J{\GD}{\GG}{e_1}{A}$,
  \item and $\J{\GD}{\GG,\m{x}\of A}{e_2}{B}$
  \end{itemize}
  it follows that
  \begin{itemize}
  \item $\J{\GD}{\GG}{\sub{e_1/\m{x}}e_2}{B}$,
  \item and $\den{\sub{e_1/\m{x}} e_2} \dg = \den{e_2} \;\delta
    \;\pair{\gamma}{\den{e_1} \dg}$.
  \end{itemize}
\end{theorem}

\subsection{Discussion}
It has been known for a very long time that database queries have a
monadic structure arising from the adjunction between \cSet and \cSL
--- indeed, the very name of the Kleisli~\cite{kleisli} database
system was chosen to reflect this fact! 

However, our decomposition of this adjunction into two smaller
adjunction, with an intermediate way-station in \cPoset is new. By
interpreting our types in \cPoset, we gain access to the comonad
$\Disc{|A|}$, which lets us distinguish between monotone and
non-monotone computations, which is the critical property letting us
interpret fixed points in a sensible way. 

% \todo{\paragraph{} More yik-yak here?}

%% \paragraph{}
%% \todo{TODO: future work reference? on operational semantics \& compatibility theorem?}


%% Section ?: Datafun vs Datalog
\section{Comparing Datalog and Datafun}
\label{sec:datalog-vs-datafun}

At this point, we have
demonstrated by example that Datafun programs are rather similar to
Datalog programs, and we have given the typing and denotational
semantics of Datafun. However, we still need to explain \emph{why} our
semantics lets us express Datalog-style programs.

To understand this, recall that Datalog is a bottom-up logic
programming language. A program consists of a primitive database of
facts, along with a set of rules the rules the programmer wrote. A
Datalog program executes by using the rules to derive new conclusions
from the database, and extending the database with them, until no
additional conclusions can be drawn. Then the query can be checked
simply by seeing if it occurs in the final database.

This is, essentially, a fixed point computation -- each stage of
execution of a Datalog program takes a database and returns an
extended database, until a fixed point is reached. The stratified
negation restriction essentially ensures that the database transformer
defined by a Datalog program is a monotone function on the set of
facts. This is why the type system of Datafun tracks the monotonicity
of functions --- since we permit both higher-order definitions and
taking fixed points, we need to ensure that the body of a fixed point
definition is monotone in order to guarantee that the recursion is
well-founded.

This ensures that the recursive definition is well-defined, but is not
sufficient by itself to guarantee termination. To manage this, Datalog
depends upon the other two restrictions described in the
introduction.By restricting terms occuring in predicates to consist of
either atoms or variables, Datalog ensures that quantifiers need only
be instantiated with the atoms used in a program. By requiring every
variable in the consequent of rules to also occur in the premise of a
rule, it ensures that every consequent will also only feature atoms
occuring in the original program.

Then, since there can only be finitely many atoms in a finite program,
this means that the set of possible arguments to a predicate is itself
finite. Then the lattice of sets of atomic predicates ordered by
inclusion will be finite, and so fixed point iteration is guaranteed
to terminate.

Instead of this (rather indirect) scheme, Datafun directly tracks the
finiteness of types, permitting recursion only if it is over a finite
type, or is bounded explicitly. These two approaches achieve the same
effect, albeit in different ways. Datalog's approach has the benefit
that no type discipline is needed to ensure finiteness. One advantage
of our choice is that we permit recursion over any semilattice, not
just the semilattice of sets. A much more serious advantage of our
approach is that it makes it much easier to write fixed-point
computations which actually \emph{compute} with the data they see (for
example, the CYK parser we wrote computed lengths of substrings). 


%% Section 7: Implementation
\section{Implementation}
We have built a proof-of-concept implementation of Datafun in Racket, available
at \texttt{[link omitted for double-blind review]}. In addition to core Datafun,
it supports pattern-matching, variant types, record types, dictionaries,
subtyping, antitone functions, and unbounded (potentially nonterminating) fixed
points. It performs no optimizations whatsoever.

\paragraph{Type inference}
As a practical matter, type-checking needs to distinguish between ordinary and
monotone $\lambda$, application, \ms{case}, \ms{let}, and \ms{if}. In our
implementation we solve this in two ways:
\begin{enumerate}
\item Bidirectional type inference \todo{CITE} determines whether $\fn$s and
  applications are ordinary or monotone.
\item For $\ms{if}$, $\ms{case}$, and $\ms{let}$, the programmer annotates which
  form is intended; for example, $(\ifthen{e}{e_1}{\unit})$ is written
  (\texttt{when e then e1}) to indicate the rule $\ms{if}^+$ applies.
\end{enumerate}

It remains an open question how much annotation is necessary. We speculate that
bidirectional inference could be replaced by a Damas-Milner \todo{CITE} style
algorithm, which infers a principal type for any term without any annotation at
all, \emph{if} we add polymorphism, tone-polymorphism, and subtyping---so that,
for example, $\fn\bind{f}\fn\bind{x} f\;x$ can be assigned the principal type
$\forall\bind{o\of\ms{tone}}\forall\bind{\alpha,\beta \of \ms{type}} (\alpha
\overset{o}\to \beta) \mto (\alpha \overset{o}\to \beta)$, where
$\overset{o}\to$ indicates a function of tone $o$; a tone may be empty (for an
ordinary function) or ${+}$ for a monotone function.

%% \todo{explain subtyping?}
%% \todo{explain antitonicity?}
%% \todo{explain ordering on dictionaries?}


%% Section 8: Tradeoffs, etc.
%% not sure how many sections to split this up into.
\section{Tradeoffs, limitations, and design decisions}

\paragraph{Finite and bounded fixed-points} \TODO Discuss
Datalog's constructor-free restriction, not being sure how to encode it in a
type theory, its disadvantages, and the disadvantages of our approaches ---
finite \& bounded fixed-points.

bounded fixed-points strictly more general than finitary-type fixed-points, but
at possible runtime cost, but finite types very restrictive in practice

\TODO In particular, could bounded fix-points have adverse performance
implications?

\paragraph{Termination} \TODO nontermination and declarativeness.

For example, one powerful optimization technique is \emph{loop reordering} (in
SQL terminology, \emph{join reordering}), that is, taking advantage of the
equation
\begin{eqnarray*}
  \forin{x \in e_1} \forin{y \in e_2} e
  &=& \forin{y \in e_2} \forin{x \in e_1} e
\end{eqnarray*}
when $x,y \notin \ms{FV}(e_1) \cup \ms{FV}(e_2)$. (\TODO Explain why join
reordering is powerful). But this equation does not always hold in the presence
of nontermination; for example, if $e_1 = \unit$ and $e_2$ diverges. \TODO
finish up

\paragraph{Type inference} Typechecking needs to distinguish between ordinary
and monotone $\lambda$, application, \ms{case}, \ms{let}, and \ms{if}. In our
implementation we solve this in two ways:
\begin{enumerate}
\item Bidirectional type inference \todo{CITE} determines whether $\fn$s and
  applications are ordinary or monotone.
\item For $\ms{if}$, $\ms{case}$, and $\ms{let}$, the programmer annotates which
  form is intended; for example, $(\ifthen{e}{e_1}{\unit})$ is written
  (\texttt{when e then e1}) to indicate the rule $\ms{if}^+$ applies.
\end{enumerate}

It remains an open question how much annotation is necessary. See
Section \ref{sec:futurework}, \emph{Related and future work}.



%% Section 9: Related & future work
\section{Related and future work}
\label{sec:futurework}

\paragraph{Aggregation}
Aggregation of values --- for example, taking the sum $\sum_{x \in A} f \;x$ of
a function $f$ across a set $A$ --- is a useful and ubiquitous database
operation. Datafun naturally supports \emph{semilattice} aggregation via
$\bigvee$, but many natural operations such as summation do not form
semilattices on their underlying type. There are several potential ways to add
support for aggregations to Datafun:
\begin{itemize}
\item Common aggregations can be provided as primitive functions, for example
  $\ms{size} : \Set{\eq{A}} \mto \N$ or $\ms{sum} : (\eq{A} \to \N) \uto
  \Set{\eq{A}} \mto \N$.

\item In the style of Machiavelli~\cite{machiavelli}, one could add a general
  operator $\ms{hom} : B \to (A \uto B \uto B) \uto \Set{A} \uto B$, which
  effectively linearizes a set in an unspecified order. The semantics of
  \ms{hom} are, alas, necessarily nondeterministic.

\item One could augment Datafun with a type of bags (multisets) $\Bag{A}$; bags
  naturally support a much broader class of aggregation --- commutative monoids
  --- than sets. See, for example, \citet{multilinear-bigdata} and
  \citet{reladj}.
\end{itemize}

\paragraph{Optimization} Because Datalog is so strongly constrained,
there has been a lot of very successful work on optimizing it, ranging
from compilation into binary decision diagrams~\cite{bdd} by
\citet{whaley-lam}, to the famous ``magic sets''~\cite{magicsets}
algorithm.

From our perspective, magic sets are a natural next step for
investigation into how to optimize Datafun. Intuitively, the magic
sets algorithm exploits the fact that Datalog (as a total logic
language) has both a top-down and bottom-up reading, and rewrites the
program so that it does bottom-up search, while using top-down
reasoning to strategically avoid adding useless facts to the
database. Transplanting this analysis to Datafun would essentially
give us optimized implementations of fixed points, but extending the
magic sets algorithm is likely to be very subtle, since Datafun has
higher-order functions and Datalog does not. As a result, our goal is
to first see if magic sets can be applied to first-order Datafun programs,
and then use defunctionalization~\cite{defunctionalization} to
extend it to full Datafun.

Very recently, \citet{flix} have introduced the Flix language, which
extends the semantics of Datalog to support defining relations valued
in arbitrary lattices (rather than just the powerset of atoms). Like
Datafun, this lets Flix support using monotone functions (on suitable
lattices) in program expressions. Unlike Datafun, Flix does not yet
have monotonicity checking for programmer-defined operators. However,
because Flix does not extend Datalog to higher order, efficient
Datalog implementation strategies (such as semi-naive evaluation)
continue to apply.


\paragraph{Databases} Datalog has sometimes been described
as ``relational algebra plus fixed points'', and there is a long line
of work on embedding database query languages into general-purpose
languages, including pioneering efforts such as
Machiavelli~\cite{machiavelli} and Kleisli~\cite{kleisli}, as well as
more recent systems such as Ferry~\cite{ferry} and LINQ in C\#~\cite{linq-wadler}.
%
The focus of this work has been on embedding query languages based
on relational algebra into general purpose languages, with an emphasis
on statically compiling higher-order queries into the first-order
queries supported by existing database systems (\citet{query-shredding} is a
representative example).

Our approach is a little bit different. Instead of embedding Datalog
into a general purpose language, Datafun is \emph{also} a ``little
language'', albeit one that happens to be a higher-order
functional language. We very deliberately did not try to embed Datafun
into an existing language, because that would have greatly complicated
the context-management operations needed to ensure monotonicity.

In fact, from a language designer's perspective, Datafun can be seen
as an argument in favor of extending functional languages to support
programming with user-defined, non-strong comonads.

\paragraph{Deletion} \citet{logical-algorithms} showed how
forward-chaining logic programming permits concise and elegant
expression of a wide variety of algorithms, including a natural cost
semantics. However, they noted that there were some algorithms (such
as union-find and greedy algorithms) which could be formulated in this
style, \emph{if} there were additionally support for deleting facts
from a database. Later, \citet{linear-logical-algorithms} went on to
show how deletion could be given a logical interpretation by
formulating in terms of linear logic programming.

This naturally raises the question of whether we could identify a
``linear Datafun'' corresponding to this style of programming, where
we might linear types to model features like deletion. There are many
nontrivial semantic issues (e.g., how to define monotonicity), but
it seems a promising question for future work.

\paragraph{Termination}

Datafun as presented is Turing-incomplete. This is advantageous for
optimization; for example, one powerful optimization technique is \emph{loop
  reordering} (in SQL terminology, \emph{join reordering}), that is, taking
advantage of the equation
\begin{eqnarray*}
  \forin{x \in e_1} \forin{y \in e_2} e
  &=& \forin{y \in e_2} \forin{x \in e_1} e
\end{eqnarray*}
when $x,y \notin \ms{FV}(e_1) \cup \ms{FV}(e_2)$. But this equation does not
always hold in the presence of nontermination; for example, if $e_1 = \unit$ and
$e_2$ diverges.

Nonetheless, without adding advanced facilities for termination
checking, there are many functions it is difficult to implement
without use of general recursion. So a natural direction for future
work is to study how to add support for general recursion to Datafun.
Because domains~\cite{domain-theory} can be understood as partial
orders with directed joins, there are likely many interesting
categorical structures connecting the category of domains to the
category of posets, some of which will hopefully lead to a principled
type-theoretic integration of partial functions into Datafun.

\paragraph{User-Defined Posets and Semilattices}
The two fundamental semilattice types Datafun provides are booleans and sets;
products and functions merely preserve semilattice structure where they find it.
One might contemplate allowing the programmer to define their own semilattice
structures using something like Haskell's \texttt{newtype}/\texttt{instance}. In
general, this is a difficult problem, because we may need to do serious
mathematical reasoning to prove that a comparison function implements a partial
ordering, or that a datatype can be equipped with a semilattice structure
obeying this partial ordering which is commutative, associative and idempotent.

One example of such a family of types are the \emph{lexicographic sum
  types}. Given two posets $P$ and $Q$, their disjoint union $P + Q$
is also a poset, with left values compared by the $P$-ordering, and
right values compared by the $Q$-ordering, and no ordering between
left and right values. However, this is not the only way that the
disjoint union could be equipped with an order structure.

For example, we could define the \emph{lexicographic} sum $P \lsum Q$,
which has the same elements as the sum, but extending the coproduct order
relation with the additional facts that $\ms{in}_1(p) \leq \ms{in}_2(q)$.
Indeed, we already have a special case of this: as we noted earlier, our boolean
type is not $1 + 1$, but it \emph{is} $1 \lsum 1$.

But as our Booleans already show, giving good syntax for their
eliminators is difficult, because we have to show that not just a term
is monotone, but that the different branches of a lexicographic case
expression are ordered with respect to \emph{each other}. For the case
of ordered Booleans, we were able to give a special eliminator which
guaranteed it, but in general it requires proof.

One natural direction for future work is to extend the syntax of
Datafun with support for these kinds of proofs, perhaps taking
inspiration from dependent type theory.



%% ---------- End matter ----------

%% \acks
%% Acknowledgments, if needed.

% We recommend abbrvnat bibliography style.
\bibliographystyle{abbrvnat}
\bibliography{datafun}

%% \appendix
%% \section{Appendix Title}

%% This is the text of the appendix, if you need one.


\end{document}
