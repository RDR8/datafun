\documentclass[preprint]{sigplanconf}

% The following \documentclass options may be useful:

% preprint      Remove this option only once the paper is in final form.
% 10pt          To set in 10-point type instead of 9-point.
% 11pt          To set in 11-point type instead of 9-point.
% numbers       To obtain numeric citation style instead of author/year.

\usepackage{datafun}


%% ---------- Setup ----------
\begin{document}

\special{papersize=8.5in,11in}
\setlength{\pdfpageheight}{\paperheight}
\setlength{\pdfpagewidth}{\paperwidth}

\conferenceinfo{ICFP '16}{18--24 September, 2016, Nara, Nara, Japan}
\copyrightyear{2016}
\copyrightdata{978-1-nnnn-nnnn-n/yy/mm}
\copyrightdoi{nnnnnnn.nnnnnnn}

% Uncomment the publication rights you want to use.
%\publicationrights{transferred}
%\publicationrights{licensed}     % this is the default
%\publicationrights{author-pays}


%% ---------- The title ----------
% These are ignored unless 'preprint' option specified.
\titlebanner{preprint}
\preprintfooter{Datafun: a Functional Datalog (PREPRINT)}

\title{Datafun: a Functional Datalog}
\subtitle{}

%% \title{Datafun}
%% \subtitle{A Functional Datalog}

%% \title{Datafun}
%% \subtitle{or, Datalog with datatypes}

%% \title{Datalog with Datatypes}
%% \subtitle{Toward a functional language of finite sets}

\authorinfo{Alfred Nony Mouskovi\v{c}}
           {}
           {a.mouse@example.org}

%% \authorinfo{Michael Arntzenius\and Neelakantan R. Krishnaswami}
%%            {University of Birmingham}
%%            {daekharel@gmail.com, N.Krishnaswami@cs.bham.ac.uk}

\maketitle


%% ---------- The abstract ----------
\begin{abstract}
This is the text of the abstract.
\end{abstract}

\category{CR-number}{subcategory}{third-level}

% general terms are not compulsory anymore,
% you may leave them out
\terms
term1, term2

\keywords
keyword1, keyword2


%% ---------- Paper body ----------
%% Section 1: Introduction
\section{Introduction}

The phrase ``declarative programming'' is as popular as it is
ambiguous, with seemingly hundreds of disparate senses in which it is
used. However, two of those usages stand out for popularity: both
\emph{functional} and \emph{logic} programming languages are generally
deemed declarative languages. Despite this common epithet, the logic
and functional programming traditions have largely evolved
independently of one another (with a few honorable exceptions such as
Mercury~\cite{mercury}, Curry~\cite{curry} and
Kanren~\cite{kanren}). This could be seen as an occasion for sorrow,
but we prefer to view it as an opportunity: as functional language
designers, we can look to logic languages to discover new ideas to
steal.

A Prolog program can be understood as a collection of logical axioms
formulated as Horn clauses (i.e., first-order formulas of the form
$\forall \vec{x}.\;P_1 \land \ldots \land P_n \to Q$, where $P_i$ and
$Q$ are atomic formulas).  Execution of a Prolog program can be
understood as running a proof search algorithm on these clauses to
figure out whether a particular formula is derivable or not.

In other words, functional and logic programming languages embody the
Curry-Howard correspondence in two different ways. In a functional
language, types are propositions, terms are proofs, and program
evaluation corresponds to proof normalization. On the other hand, for
logic programming languages, \emph{terms} are propositions, and
program evaluation corresponds to \emph{proof search}.

Due to the undecidability of general theorem proving, designers of
logic programming languages have to both be very careful about the
kinds of formulas they admit as programs, and to be careful about the
proof search algorithm they implement. Prolog offers a very expressive
language --- full Horn clauses --- and so it faces an undecidable
proof search problem. So to make its operational behaviour
predictable, its designers specified a particular proof search
strategy, namely depth-first goal-directed search. As a result,
programmers can reliably reason about the behaviour of a Prolog
program, at the cost of making certain logically natural programs go
into infinite loops. (Notoriously, transitive closure calculations are
much less elegant in Prolog than one might hope, since their most
natural specification is best computed with a breadth-first proof
search strategy.)

However, this view of Prolog suggests that there are other possible
design choices, such as restricting the logical language sufficiently
to make the proof search problem decidable. One of the oldest such
variants is Datalog~\cite{datalog}, which can be seen as a subset of
Prolog satisfying three restrictions on the clauses defining a program:
\begin{enumerate}
\item Terms occurring in predicates are forbidden from being
  composites: they can only be ground terms, or variables. This
  restriction ensures that deduction will never introduce new terms
  that do not occur in the source of the program.
\item Variables occurring in the head of a clause (i.e., the
  consequent of a Horn clause) must also occur (positively) in the
  body (i.e., the premises of a Horn clause).
\item No predicate can be negated unless it has already been fully
  defined.  This is sometimes called ``stratified negation''.
\end{enumerate}
These drastic restrictions make Datalog Turing-eincomplete -- all
queries are decidable. As functional programmers are well aware,
though, there is often power in restraint: for example, in a total
functional language, the compiler is free to switch between strict and
lazy evaluation as it deems fit. Similarly, in Datalog decidability
means that implementations are free to use forwards chaining, and so
can easily support queries (like reachability and transitive closure)
which are difficult to implement in ordinary Prolog.

Over the last decade or so, this freedom has been put to very good
use, with Datalog appearing at the heart of a a wild variety of
applications in both research and industry.  For example, Whaley and
Lam \cite{whaley-lam,whaley-phd} implemented pointer analysis
algorithms in Datalog, and found that they could reduce the size of
their analyses from thousands of lines of C code to \emph{tens} of
lines of Datalog code, while still retaining competitive
performance. Semmle has developed the .QL
language~\cite{semmlecode,ql-inference} based on Datalog for analysing
source code (which was used to analyze the code for NASA's Curiosity
Mars rover), and LogicBlox has developed the LogiQL~\cite{logicblox}
language for business analytics. The Boom project at Berkeley has
developed the Bloom language for distributed programming~\cite{bloom},
and the Datomic cloud database~\cite{datomic} uses Datalog (embedded
in Clojure) as its query language. Microsoft's SecPAL
language~\cite{secpal} uses Datalog as the foundation of its
decentralised authorization specification language.

In all of these cases, the use of Datalog permits giving
specifications and implementations which are dramatically shorter and
clearer than alternatives implemented in more conventional
languages. However, while all of these applications are built on a
foundation of Datalog, they all also extend it significantly. For
example, it is impossible even to implement arithmetic in Datalog,
since adding 2 and 3 produces 5, which is a new term not equal to
either 2 or 3! As a result, even though Datalog has a very clean
semantics, its metatheory needs to be re-established once again for
each application-specific extension to it.

As a result, it would be very desirable to undestand what makes
Datalog tick, so that we can embed into a more expressive language
\emph{without} sacrificing the properties that make it so powerful
within its domain.  In this way, extensions can become ``a simple
matter of programming'', without having to do a custom redesign of the
language for each application.

In this paper, we present Datafun, a typed functional language which
permits programming in the style of Datalog, while still supporting
the full power of higher-order functional programming.

\paragraph{Contributions}
\begin{itemize}
\item We describe Datafun, a typed language capturing the expressive
  power of Datalog and extending it to support higher-order functional
  programming. Datafun's key feature is to \emph{track monotonicity
    with types}. This permits us to use typing to analyze fixed point
  computations in a way ensuring their termination.

\item We present examples illustrating the expressive power of
  Datafun, including relational-algebra-style operations, transitive
  closure, CYK parsing, and dataflow analysis. Some of these examples
  are familiar from Datalog, but many of them go well beyond what can
  be expressed in it, illustrating the benefits of our analysis.

\item We identify the semantic structures underpinning Datalog, and
  use this to give a denotational semantics for Datafun in terms of a
  pair of adjunctions between \cSet{}, \cPoset{}, \cSL{}.

\item We have a prototype implementation of Datafun in Racket. \todo{(CITE)}
\end{itemize}

%% Contributions (as summarized by Michael):
% - Datafun, like Datalog but functional
% - examples, incl. both datalog examples & things datalog can’t do
% - key ingredient is monotonicity; ``found'' semantics by analyzing
%   datalog: two adjunctions, three categories
% - prototype implementation

%% Contributions (as written by Neel):

% - We describe Datafun, a type theory for a language capturing the expressive
%   power of Datalog and extends it to both relax the constructor term
%   restriction and to support higher-order functional programming.

% - We give a variety of examples that illustrate the expressive power of
%   Datafun, such as CYK parsing, dataflow analysis, and transitive closure on
%   graphs, etc. Many of these examples are traditional examples of Datalog,
%   but we are also able to support things like first-class relations (eg,
%   generic transitive closure) and higher-order functions (example using
%   monotonicity and HO?). (doing a fix-point code analysis / parsing something
%   & dispatching on result?)

% - We identify the semantic structures underpinning Datalog, and use this to
%   give a denotational semantics for Datafun in terms of a pair of adjunctions
%   between Set, Poset, and the category of semilattices with finitary joins.

% - We have a prototype implementation of Datafun in Racket.

% Local Variables:
% TeX-master: "datafun"
% End:


%% Section 2: Datafun, informally
%% FIGURE: CORE SYNTAX
\begin{figure}
  \[\begin{array}{ccl}
    %% types
    A, B     &\bnfeq& \bool \pipe \N \pipe \str \pipe \Set{A}
                      \pipe \Map{A}{B}
    \\
    \textsf{types} && A + B \pipe A \x B \pipe A \uto B \pipe A \mto B
    \vspace{0.5em}\\
    %% semilattice types
    L, M         &\bnfeq& \bool \pipe \N \pipe \Set{A} \pipe \Map{A}{L}
    \\
    \textsf{semilattice types} && L \x M \pipe A \uto L \pipe A \mto L
    \vspace{0.5em}\\
    %% equality types
    \eq{A}, \eq{B} &\bnfeq& \bool \pipe \N \pipe \str \pipe \Set{\eq{A}}
                            \pipe \Map{\eq{A}}{\eq{B}}
    \\
    \textsf{eqtypes} && \eq{A} + \eq{B} \pipe \eq{A} \x \eq{B}
    \vspace{0.5em}\\
    %% finite equality types
    \fineq{A},\fineq{B}
    &\bnfeq& \bool \pipe \Set{\fineq{A}}
       \pipe \Map{\fineq{A}}{\fineq{B}}
       \pipe \fineq{A} + \fineq{B} \pipe \fineq{A} \x \fineq{B}\\
    \textsf{finite eqtypes}
    \vspace{0.5em}\\
    %% contexts
    \GD &\bnfeq& \cdot \pipe \GD, x\of A\\
    \GG &\bnfeq& \cdot \pipe \GG{},\m{x}\of A\\
    \textsf{contexts}\\
    %% expressions
    e &\bnfeq& x \pipe \m{x} \pipe \fn\bind{x} e \pipe \fn\bind{\m{x}} e
    \pipe e\;e\\
    \textsf{terms}
    && (e,e) \pipe \pi_1\;e \pipe \pi_2\;e
    \pipe \ms{in}_1\;e \pipe \ms{in}_2\;e\\
    && \case{e}{x}{e}{x}{e}\\
    && \case{e}{\m{x}}{e}{\m{x}}{e}\\
    && \ms{true} \pipe \ms{false} \pipe \ifthen{e}{e}{e}\\
    && \singleset{e} \pipe \singlemap{e}{e}
       \pipe \unit \pipe e \vee e \pipe \letin{x}{e}{e}\\
    && \fix{\m{x}}{e} \pipe \fixle{\m{x}}{e}{e}
  \end{array}\]
  \caption{Syntax of core Datafun}
  \label{fig:syntax}
\end{figure}

\section{Datafun, informally}

We give the core syntax of Datafun in Figure \ref{fig:syntax}. Datafun is a
simply-typed $\lambda$-calculus extended in four major ways:

\begin{enumerate}
\item We add a type of finite sets, $\Set{A}$. We also include finite maps
  $\Map{A}{B}$, as they are useful for some examples.

  \todo{Describe why sets are useful?}

  %% We use finite sets to represent Datalog predicates; one might also think of
  %% them as tables or views in a database setting.

\item We add a type of \emph{monotone functions}, $A \mto B$. Consequently
  Datafun has two flavors of variable, \emph{ordinary} and \emph{monotone}. We
  write ordinary variables in $script$ and monotone variables in \m{bold}.

  In order for ``monotone'' to have meaning, our types are implicitly partially
  ordered:
  \begin{itemize}
  \item Booleans $\bool$ are ordered $\ms{false} < \ms{true}$.
  \item Natural numbers $\N$ have the usual order: $0 < 1 < 2 < ...$.
  \item We have no particular use-case for comparing strings $\str$ in
    this paper, so we order them discretely; $a \le b$ iff $a = b$. \todo{Better
      explanation?}
  \item Pairs and functions are ordered pointwise:
    \begin{itemize}
    %% \item $(a_1, b_1) \le (a_2, b_2)$ iff $a_1 \le a_2 \wedge b_1 \le b_2$
    \item $(a, x) \le (b, y)$ iff $a \le b \wedge x \le y$
    \item $f \le g$ iff $\forall \bind{x} f(x) \le g(x)$
    \end{itemize}
  \item Sum types are ordered disjointly: $\ms{in}_i\; a \le
    \ms{in}_i\; b$ iff $a \le b$, but $\ms{in}_1\; a$ and $\ms{in}_2\; b$ are
    never comparable.
  \item Sets are ordered by inclusion: $a \le b$ iff $a \subseteq b$.
  \item Maps are ordered both by inclusion and pointwise.
  \end{itemize}

\item We add a term $(\fix{\m{x}}{e})$ denoting the least fixed point of the
  monotone function $(\fn\bind{\m{x}} e)$. \todo{Explain why this is useful?}
  This is computed (modulo optimizations) by iteration, starting from the
  smallest value of the desired type and halting once a fixed point is found.
  This strategy constrains the types of \ms{fix} terms in several ways:
  \begin{itemize}
  \item The type must have a smallest value. We enforce this using semilattice
    types (see item \ref{item:semilattice-types}, below).

  \item The type must support equality tests, to determine when a fixed point
    has been reached. We call a type supporting equality tests an \emph{eqtype}.
    \todo{CITE}

  \item To ensure termination, the type must have finite height.\footnote{The
    height of a poset is the cardinality of its largest chain (totally-ordered
    subset).} We conservatively approximate this property by limiting \ms{fix}
    to finite types.
  \end{itemize}

  In summary, \ms{fix} may only be used at \emph{finite semilattice eqtypes}.

  \todo{TODO: connection to Datalog via finiteness of predicates}

  \todo{explain $\fixle{\m{x}}{e_\top}{e}$?}

  %% \todo{Explain:
  %%   \begin{itemize}
  %%   \item compute fixpoint (at least conceptually) by iteration
  %%   \item start from a least element
  %%   \item to ensure termination, need finite height of lattice
  %%   \item finiteness guarantees finite height!
  %%   \item (\emph{only then} mention analogy to Datalog)
  %% \end{itemize}}

\item\label{item:semilattice-types} Generalizing the empty set $\emptyset$ and
  union $\cup$, we identify a subset of types that have a \emph{least element}
  $\unit$ and a \emph{least upper bound} operator $\vee$. We call these
  \emph{semilattice types}\footnote{Technically, the partial orderings on these
    types form \emph{join-semilattices with a least element}. For brevity's
    sake, we call these structures simply ``semilattices.''}, and denote them by
  the metavariables $L,M$.

  Semilattice types serve two primary purposes:

  \textbf{First}, they provide a natural eliminator for sets. Given $e :
  \Set{A}$, we write $\letin{x}{e'}{e_x}$ for the least upper bound, over all
  elements $x \in e'$, of $e_x$, for $e_x$ of some semilattice type $L$. If
  $e_x$ is a set, for example, this provides the set type's monadic ``bind''
  operation.%% For example, $\forin{x \in \setlit{1,2,3}} \{10 \cdot x, x^2\}$
  %% denotes the set $\{1, 4, 9, 10, 20, 30\}$.

  \todo{example?}

  \todo{explain testing for set membership?}

  \textbf{Second}, as already mentioned, they guarantee the presence of a least
  element, needed to compute \ms{fix} terms.

  \todo{Explain how products of semilattice being semilattice + monotone
    fixed-points account for mutual recursion.}

\end{enumerate}


%% Section 3: examples
%% FIGURE: SYNTAX SUGAR
\begin{figure}
  \[\begin{array}{lccl}
  %% expressions
  \textsf{terms} &
  e &\bnfeq& ... \pipe e \isin e \pipe \setfor{e}{\mc{L}}
             \pipe \forin{\mc{L}}{e}\\
  &&& \mathcal{C}[\;e]^* \pipe \rawcase{e}{[{p} \cto {e}]^*}
  \vspace{0.5em}\\
  %% patterns
  %%
  %% TODO: maybe remove the pattern-matching stuff? since we don't explain how
  %% to translate it & we also use various other sugar we don't explain how to
  %% translate, why do we include only pattern-matching here?
  \textsf{patterns} &
  p &\bnfeq& \pwild \pipe x \pipe (p,p)
             \pipe \ms{true} \pipe \ms{false} \pipe \mathcal{C}[\;p]^*
  \vspace{0.5em}\\
  \textsf{constructors} & \mathcal{C} && \text{are abstract identifiers}
  \vspace{0.5em}\\
  %% loop clauses
  \textsf{loops} &
  \mc{L} &\bnfeq& \mc{L}, \mc{L} \pipe p \in e \pipe e
  \end{array}\]

  %% the desugaring syntax-expansion itself
  \begin{eqnarray*}
    %% e_1 \isin e_2            &\expandsto& \forin{x \in e_2} e_1 = x\\
    \setfor{e}{\mc{L}}       &\expandsto& \forin{\mc{L}}{\{e\}}\\
    \forin{\mc{L}_1,\mc{L}_2}{e}
    &\expandsto& \forin{\mc{L}_1}{\forin{\mc{L}_2}{e}}\\
    \forin{p\in e_1}{e_2} &\expandsto&
    \letin{x}{e_1}{\rawcase{x}{p \cto e_2;\,\pwild \cto \unit}}\\
    \forin{e_1}{e_2} &\expandsto& \ifthen{e_1}{e_2}{\unit}
    %% e_1 \x e_2 &\expandsto& \setfor{(a,b)}{a \in e_1, b \in e_2}\\
    %% e_1 \bullet e_2 &\expandsto& \setfor{(a,c)}{(a,b) \in e_1, (b,c) \in e_2}
    %% \ms{let}~x = e_1 ~\ms{in}~ e_2
    %% &\expandsto& (\fn\bind{x} e_2)\; e_1\\
    %% \ms{let}~[x_i = e_i]^* ~\ms{in}~ e
    %% &\expandsto& [\ms{let}~x_i = e_i~\ms{in}]^* e\\
    %% \rawcase{e}{[p \cto e]^*} &\expandsto& \text{(omitted, see \todo{CITE})}
  \end{eqnarray*}
  \caption{Syntax sugar}
  \label{fig:sugar}
\end{figure}


\section{Examples}

For purposes of these examples, we use a simple Haskell-like syntax for
top-level type and function definitions. We also permit ourselves infix
notation, \ms{let}-binding, $n$-ary tuples, $n$-ary sum types with named
constructors, pattern-matching\todo{(CITE)}, and additional syntax sugar given
in Figure \ref{fig:sugar} \todo{(TODO: mention \& cite monadic query syntax)}.
All of these conveniences are supported (with slightly different concrete
syntax) in our implementation.

For clarity, we set the names of top-level variables in \textsf{sans-serif};
ordinary variables in $script$ or \mi{italic} (for long variable names); and
monotone variables in \m{bold}.

\todo{TODO: Explain nonlinear variable usage means equality.}

\todo{\begin{itemize}
\item composition of relations?
\item \texttt{make}-style topological sort?
\item SQL-style examples?
\item translating relational algebra into datafun?
\end{itemize}}


\subsection{Filtering, mapping, and cross products}

Although Datafun as presented does not have polymorphism, we give our examples
their most general possible type schemes. \todo{why?}

\todo{these examples are perhaps a good place to point out subtleties of
  monotonicity typing}

\[\begin{array}{l}
\fname{map} ~:~ (A \uto B) \uto \Set{A} \mto \Set{B}\\
\fname{map}\;f\;\m{A} = \setfor{f\;x}{x \in \m{A}}\\
\\
\fname{filter} ~:~ (A \uto \bool) \mto \Set{A} \mto \Set{A}\\
\fname{filter}\;\m{f}\;\m{A} = \setfor{x}{x \in \m{A}, \m{f}\; x}\\
\\
(\times) ~:~ \Set{A} \mto \Set{B} \mto \Set{A \x B}\\
\m{A} \times \m{B} = \setfor{(a,b)}{a \in \m{A}, b \in \m{B}}
\end{array}\]

\TODO


%% FIGURE: PRIMITIVES
\begin{figure}
  %% TODO: remove unused primitives.
  \[\begin{array}{cll}
  \neg &\of& \bool \uto \bool\\
  =   &\of& \eq{A} \uto \eq{A} \uto \bool\\
  \le &\of& \eq{A} \uto \eq{A} \mto \bool\\
  \fname{keys}     &:& \Map{A}{B} \mto \Set{A}\\
  \fname{entries}  &:& \Map{A}{B} \uto \Set{A \x B}\\
  \fname{tabulate} &:& \Set{A} \mto (A \uto B) \mto \Map{A}{B}\\
  %% \fname{getWith}  &:& \Map{\eq{A}}{B} \mto \eq{A} \uto (B \mto L) \mto L\\
  \fname{get}      &:& \Map{\eq{A}}{L} \mto \eq{A} \uto L\\
  %% \fname{substrings} &\of& \ms{Str} \uto \Set{\ms{Str}}\\
  %% \fname{size}     &:& \Set{\eq{A}} \mto \N\\
  \fname{range}    &:& \N \uto \N \mto \Set{\N}\\
  \fname{length}   &:& \str \uto \N\\
  \fname{substring} &:& \str \uto \N \uto \N \uto \str
  \end{array}\]
  \caption{Primitive functions and their type schemes}
\end{figure}


\subsection{Testing for membership}

\todo{Use this as example of $\bigvee$ at non-set (here, boolean) type.}

\[\begin{array}{l}
(\isin) ~:~ \eq{A} \uto \Set{\eq{A}} \mto \bool\\
x \isin \m{A} = \forin{y \in \m{A}} x = y
\end{array}\]

\todo{TODO: discuss nonlinear pattern-matching}

An equivalent definition, making use of nonlinear pattern-matching, would be:
\[\begin{array}{l}
x \isin \m{A} = \forin{x \in \m{A}} \ms{true}
\end{array}\]

\todo{Nonlinear pattern matching makes implicit the equality test and attendant
  restriction to equality types, but is often extremely convenient.}


\subsection{Composition of relations}

\todo{mention eqtypes again here? or use this to show them off instead of
  $({\isin})$?} \todo{work phrase ``higher-order'' in here somewhere?}

\noindent As an example of a function over relations, consider:
\[\begin{array}{l}
(\bullet) : \Set{A \x \eq{B}} \mto \Set{\eq{B} \x C} \mto \Set{A \x C}\\
\m{R} \bullet \m{S} = \setfor{(a,c)}{(a,b) \in \m{R}, (b,c) \in \m{S}}
\end{array}\]

\todo{TODO: mention you can extend to all relational algebra? give example?}


\subsection{Transitive closure}

Consider the following Datalog program:
\begin{verbatim}
parent(earendil, elrond).
parent(elrond, arwen).
ancestor(X, Y) :- parent(X, Y).
ancestor(X, Z) :- ancestor(X, Y), ancestor(Y, Z).
\end{verbatim}

\todo{Discuss how this works in Datalog, but not in Prolog, b/c Prolog is
  defined by operational semantics of unification while Datalog is denotational,
  least-model semantics. It also works in Datafun!}

\todo{Neel suggests using distinction b/w backward \& forward chaining here,
  rather than operational/denotational. see Logical Algorithms paper by
  McAllister \& co for phrasing?}

In Datafun, we may write this as:

\[\begin{array}{l}
\mathbf{data}~\ms{person} =
\ctor{E\"arendil} ~|~ \ctor{Elrond} ~|~ \ctor{Arwen}\\
\fname{parent},~\ms{ancestor} ~:~ \Set{\ms{person} \x \ms{person}}\\
\ms{parent} =
\{(\ctor{E\"arendil}, \ctor{Elrond}), (\ctor{Elrond}, \ctor{Arwen})\}\\
\ms{ancestor} = \fix{\m{X}} \ms{parent} \vee (\m{X} \bullet \m{X})
%% \setfor{(a,c)}{(a,b) \in \m{X}, (b,c) \in \m{X}}
\\
\end{array}\]


\subsubsection{Transitive closure with an upper bound}

If you know the type over which you are taking your transitive closure is
finite, this suffices. But the more general way to take a fixed-point is to give
an upper bound which you know the desired answer will not exceed. For this we
write $(\fixle{\m{x}}{e_\top} e)$

\todo{mention using $\str$ as example}

\todo{explain what happens when $e \ge e_\top$ --- it gets clamped}

\[\begin{array}{l}
\ms{person} ~:~ \Set{\str}\\
\ms{person} = \{\texttt{"e\"arendil"}, \texttt{"elrond"}, \texttt{"arwen"}\}\\
\ms{parent}, \ms{ancestor} ~:~ \Set{\str \x \str}\\
\ms{parent} = \{(\texttt{"e\"arendil"}, \texttt{"elrond"}),
(\texttt{"elrond"}, \texttt{"arwen"})\}\\
\ms{ancestor} = \fixle{\m{X}}{(\ms{person} \x \ms{person})}
\ms{parent} \vee (\m{X} \bullet \m{X})
%% \ms{ancestor} = \fixle{\m{X}}{%
%%   \setfor{(a,b)}{a \in \ms{person}, b \in \ms{person}}
%% }
%% \\
%% \hspace{4.87em}\ms{parent} \vee
%% \setfor{(a,c)}{(a,b) \in \m{X}, (b,c) \in \m{X}}
\end{array}\]


\subsubsection{Generic transitive closure}

For any finite eqtype $\fineq{A}$, we may write:
\[\begin{array}{l}
\ms{trans} ~:~ \Set{\fineq{A} \x \fineq{A}} \mto \Set{\fineq{A} \x \fineq{A}}
\vspace{0.3em}\\
%% \ms{trans}\ E = \fix{X} E \vee \setfor{(a,c)}{(a,b) \in E, (b,c) \in X}
%% \ms{trans}\ \m{E} = \fix{\m{X}} \m{E} \vee %
%% \setfor{(a,c)}{(a,b) \in \m{X}, (b,c) \in \m{X}}
\ms{trans}\ \m{E} = \fix{\m{X}} \m{E} \vee (\m{X} \bullet \m{X})
\end{array}\]

\noindent Similarly, for any eqtype $\eq{A}$, we may write:
\[\begin{array}{l}
\ms{trans} ~:~
\Set{\eq{A}} \mto \Set{\eq{A} \x \eq{A}} \mto \Set{\eq{A} \x \eq{A}}
\vspace{0.3em}\\
%% \ms{trans}\ \m{V}\ \m{E} = %
%% \ms{fix}~ \m{S} \le \setfor{(a,b)}{a\in \m{V}, b \in \m{V}}\\
%% \hspace{5.35em}\ms{is}~ \m{E} \vee %
%% \setfor{(a,c)}{(a,b) \in \m{S}, (b,c) \in \m{S}}\\
\ms{trans}\ \m{V}\ \m{E} = %
\fixle{\m{S}}{(\m{V} \x \m{V})} \m{E} \vee (\m{S} \bullet \m{S})
\end{array}\]

\TODO This already shows how we go beyond Datalog's .


\subsection{CYK parsing}

\[\begin{array}{l}
\mathbf{type}~\ms{sym} = \str\\
\mathbf{data}~\ms{rule} = \ctor{String}~\str ~|~ \ctor{Concat}~\ms{sym}~\ms{sym}\\
\mathbf{type}~\ms{grammar} = \Set{\ms{sym} \x \ms{rule}}\\
\mathbf{type}~\ms{fact} = \ms{sym} \x \N \x \N\\
\fname{parse} ~:~ \str \uto \ms{grammar} \mto \Set{\ms{sym}}\\
\fname{iter} ~:~ \str \uto \ms{grammar} \mto \Set{\ms{fact}}
\mto \Set{\ms{fact}}\\
%% parse
\fname{parse} \;\mi{text} \;\m{G} =\\
\hspace{1em} \ms{let}~ n = \ms{length} \;\mi{text}\\
\hspace{2.375em}\m{bound} =
  \{(a,i,j) ~|~ (a,\pwild) \in \m{G},\\
\hspace{10.5em}i \in \ms{range}\;0\;n, \\
\hspace{10.5em}j\in\fname{range}\;i\;n\}\\
\hspace{2.375em} \m{chart} = \fixle{\m{C}}{\m{bound}}
  \ms{iter} \;\mi{text} \;\m{G} \;\m{C}\\
\hspace{1em}\ms{in}~\setfor{a}{(a, 0, n) \in \m{chart}}\\
%% iter with \forin
\\
\fname{iter} \;\mi{text} \;\m{G} \;\m{chart} =\\
\hspace{1em}\phantom{\vee~}
(\bigvee((a, \ctor{Concat} \;b \;c) \in \m{G},\\
\hspace{1.25em}\phantom{\vee~ \bigvee(}
(b,i,j) \in \m{chart}, (c,j,k) \in \m{chart})\\
\hspace{1.25em}\phantom{\vee~}\, \setlit{(a,i,k)})\\
\hspace{1em}\vee~ (\bigvee((a, \ctor{String} \;s) \in \m{G},\\
\hspace{1.25em}\phantom{\vee~\bigvee(}
i \in \ms{range} \;0 \;(n - \ms{length} \; s),\\
\hspace{1.25em}\phantom{\vee~\bigvee(}
s = \ms{substring} \;\mi{text} \;i \;(i+\ms{length}\;s))\\
\hspace{1.25em}\phantom{\vee~}\, \setlit{(a,i,i+ \ms{length}\;s)})
\\
%% iter with case. I like this version best.
\\
\fname{iter} \;\mi{text} \;\m{G} \;\m{chart} =\\
\hspace{1em}\forin{(a,r) \in \m{G}}\\
\hspace{1.875em}\ms{case}~ r\\
\hspace{2.4em}\ms{of}~
%% \hspace{3.05em}\pipe
\ctor{Concat} \;b \;c \cto \{(a,i,k) ~|~ (b,i,j) \in \m{chart},\\
\hspace{14.42em}(c,j,k) \in \m{chart}\}\\
\hspace{3.05em}\pipe \ctor{String} \;s \cto
\{\,(a, i, i+\ms{length}\;s)\\
\hspace{9em}|~ i \in \ms{range} \;0 \;(n-\ms{length}\;s),\\
\hspace{9em}\phantom{|~}
s = \ms{substring} \;\mi{text} \;i \;(i+\ms{length}\;s)\}
\\
%% iter
\\
\fname{iter} \;\mi{text} \;\m{G} \;\m{chart} =\\
\hspace{1em}\phantom{\vee~}
\{(a,i,k) ~|~ (a, \ctor{Concat}~b~c) \in \m{G},\\
\hspace{6.25em} (b,i,j) \in \m{chart}, (c,j,k) \in \m{chart}\}\\
\hspace{1em}\vee~ \{(a,i,i+\fname{length}\;s)\\
\hspace{2.1em}|~ (a, \ctor{String}~s) \in \m{G},\\
\hspace{2.2em}\phantom{|~} i \in \fname{range}\;0\;(n-\fname{length}\;s),\\
\hspace{2.2em}\phantom{|~}
s = \fname{substring} \;\mi{text} \;i \;(i+\ms{length}\;s)\}
\end{array}\]

\TODO Since Datalog provides no way to represent a \emph{grammar} as a piece of
data (because it's compound, not an atom), there is simply no way in Datalog
proper to express a generic CYK parser. This demonstrates the benefits of moving
to a functional language.

\TODO Moreover, Datalog's strategy for proving termination involves a
constructor restriction to ensure all relations are finite. Primitives such as
\ms{range} and \ms{substring} violate this restriction (as relations, they are
infinite); it is not immediately obvious that Datalog programs extended with
these primitives remain terminating. Our use of bounded fixed-points to
guarantee termination is robust under such extensions; as long as all primitive
functions are total, Datafun programs always terminate.


\subsection{Dataflow analysis}

\[\begin{array}{l}
\textbf{type}~\ms{var} = \str\\
\textbf{type}~\ms{label} = \N\\
\textbf{data}~\ms{oper} = \ctor{Eq} \pipe \ctor{Le}
\pipe \ctor{Add} \pipe \ctor{Sub} \pipe\ctor{Mul}\pipe\ctor{Div}\\
\textbf{data}~\ms{atom} = \ctor{Var}\;\ms{var} \pipe \ctor{Num}\;\N\\
\textbf{data}~\ms{expr} = \ctor{Atom}\;\ms{atom}
\pipe \ctor{Apply}\;\ms{oper}\;\ms{atom}\;\ms{atom}\\
\textbf{data}~\ms{stmt} =
\ctor{Assign} \;\ms{var} \;\ms{expr}
\pipe \ctor{If} \;\ms{expr} \;\ms{label}\;\ms{label}\\
\textbf{type}~\ms{code} = \Map{\ms{label}}{\ms{stmt}}
\vspace{0.5em}\\
%% omitted functions
\ms{vars} ~:~ \ms{code} \uto \Set{\ms{var}} \quad\textsf{-{}- omitted}\\
\ms{uses} ~:~ \ms{stmt} \uto \Set{\ms{var}} \quad\textsf{-{}- omitted}\\
\ms{defines} ~:~ \ms{stmt} \uto \Set{\ms{var}} \quad\textsf{-{}- omitted}
\vspace{0.5em}\\
%% control flow
\textbf{type}~\ms{flow} = \Set{\ms{label} \x \ms{label}}\\
\fname{flow} ~:~ \ms{code} \uto \ms{flow}\\
\fname{flow}\;c = \forin{(i,s) \in \ms{entries}\;c}\\
\hspace{4em}\ms{case}~ s ~\ms{of}~
\ctor{If} \;\pwild \;j \;k \cto \setlit{(i,j),(i,k)}\\
\hspace{7.45em}\pipe\pwild \cto \setfor{(i,i+1)}{i+1 \isin \ms{keys}\;c}
\vspace{0.5em}\\
%% live code analysis
\ms{live} ~:~ \ms{code} \uto \ms{flow} \uto \Set{\ms{label} \x \ms{var}}\\
\ms{live} \;\mi{code} \;\mi{flow} =\\
\hspace{2em} \fixle{\m{Live}}{ %
  \ms{keys}\;\mi{code} \x \ms{vars}\;\mi{code}}\\
\hspace{2em}\forin{(i,\mi{stmt}) \in \ms{entries} \;\mi{code}}\\
\hspace{2.875em} (\phantom{\vee~}\setfor{(i,v)}{v \in \ms{uses}\;\mi{stmt}}\\
\hspace{3.2em} \vee~ \{(i,v) ~|~ (i,j) \in \mi{flow},\\
\hspace{7.4em}(j,v) \in \m{Live},\\
\hspace{7.4em}\neg (v \isin \ms{defines}\; \mi{stmt})\})
\vspace{0.5em}\\
%% reaching definitions analysis
\textsf{\todo{explain which label means what}}\\
\ms{reachingDefinitions} ~:~ \ms{code} \uto \ms{flow}
\uto \Set{\ms{label} \x \ms{label} \x \ms{var}}\\
\ms{reachingDefinitions} \;\mi{code} \;\mi{flow} =\\
\hspace{2em}\fixle{\m{RD}}{%
  \ms{keys}\;\mi{code} \x \ms{keys}\;\mi{code} \x \ms{vars}\;\mi{code}}\\
\hspace{2em}\forin{(i,\mi{stmt}) \in \ms{entries}\;\mi{code}}\\
\hspace{2.875em} (
\phantom{\vee~}\setfor{(i,i,v)}{v \in \ms{defines}\;\mi{stmt}}\\
\hspace{3.2em} \vee~ \{(i,l,v) ~|~ (j,i) \in \mi{flow},\\
\hspace{8.14em}(j,l,v) \in \m{RD},\\
\hspace{8.14em}\neg(v \isin \ms{defines}\;\mi{stmt})\})
\end{array}\]

\TODO


%% Section 4: Typing rules
%% FIGURE: Typing rules
\begin{figure*}
  %% \boxed{\ensuremath{\mathsz{10pt}{\J{\Delta}{\Gamma}{e}{A}}}}
  \begin{mathpar}
    \infer[\rn{\rt{var}}]{\J{\GD}{\GG}{x}{A}}{x\of A \in \GD}
    \and
    \infer[\rn{\rt{var}^+}]{\J{\GD}{\GG}{\m{x}}{A}}{\m{x}\of A \in \GG}
    %% function rules
    \and
    \infer[\rn{\fn}]{\J{\GD}{\GG}{\fn\bind{x} e}{A \uto B}}{
      \J{\GD,x\of A}{\GG}{e}{B}}
    \and
    \infer[\rn{\rt{app}}]{\J{\GD}{\GG}{e_1\;e_2}{B}}{
      \J{\GD}{\GG}{e_1}{A \uto B} &
      \J{\GD}{\cdot}{e_2}{A}}
    \and
    \infer[\rn{\monofn}]{\J{\GD}{\GG}{\fn\bind{x} e}{A \mto B}}{
      \J{\GD}{\GG,x \of A}{e}{B}}
    \and
    \infer[\rn{\rt{app}^+}]{\J{\GD}{\GG}{e_1\;e_2}{B}}{
      \J{\GD}{\GG}{e_1}{A \mto B} &
      \J{\GD}{\GG}{e_2}{A}}
    %% product & sum rules
    \and
    \infer[\rn{\rt{pair}}]{\J{\GD}{\GG}{(e_1,e_2)}{A_1 \x A_2}}{
      \J{\GD}{\GG}{e_i}{A_i}}
    \and
    \infer[\rn{\pi}]{\J{\GD}{\GG}{\pi_i\;e}{A_i}}{\J{\GD}{\GG}{e}{A_1 \x A_2}}
    \and
    \infer[\rn{\rt{in}}]{\J{\GD}{\GG}{\ms{in}_i\;e}{A_1 + A_2}}{
      \J{\GD}{\GG}{e}{A_i}
    }
    \and
    \infer[\rn{\rt{case}}]{\J{\GD}{\GG}{\case{e}{x}{e_1}{x}{e_2}}{C}}{
      \J{\GD}{\cdot}{e}{A_1 + A_2} &
      \J{\GD,x\of A_i}{\GG}{e_i}{C}}
    \and
    \infer[\rn{\rt{case}^+}]{
      \J{\GD}{\GG}{\case{e}{\m{x}}{e_1}{\m{x}}{e_2}}{C}
    }{
      \J{\GD}{\GG}{e}{A_1 + A_2} &
      \J{\GD}{\GG,\m{x}\of A_i}{e_i}{C}
    }
    %% boolean rules
    %% \and
    %% \infer[\rn{=}]{\J{\GD}{\GG}{e_1 = e_2}{\bool}}{\J{\GD}{\GG}{e_i}{\eq{A}}}
    \and
    \infer[\rn{\rt{true}}]{\J{\GD}{\GG}{\ms{true}}{\bool}}{}
    \and
    \infer[\rn{\rt{false}}]{\J{\GD}{\GG}{\ms{false}}{\bool}}{}
    \and
    \infer[\rn{\rt{if}}]{\J{\GD}{\GG}{\ifthen{e}{e_1}{e_2}}{A}}{
      \J{\GD}{\cdot}{e}{\bool} &
      \J{\GD}{\GG}{e_i}{A}}
    \and
    \infer[\rn{\rt{if}^+}]{\J{\GD}{\GG}{\ifthen{e}{e_1}{\unit}}{L}}{
      \J{\GD}{\GG}{e}{\bool} &
      \J{\GD}{\GG}{e_1}{L}}
    %% set & semilattice rules
    \and
    \infer[\unit]{\J{\GD}{\GG}{\unit}{L}}{}
    \and
    \infer[\rn{\vee}]{\J{\GD}{\GG}{e_1 \vee e_2}{L}}{\J{\GD}{\GG}{e_i}{L}}
    \and
    \infer[\rn{\{\}}]{\J{\GD}{\GG}{\{e\}}{\Set{A}}}{\J{\GD}{\cdot}{e}{A}}
    \and
    \infer[\rn{\bigvee}]{\J{\GD}{\GG}{\letin{x}{e_1}{e_2}}{L}}{
      \J{\GD}{\GG}{e_1}{\Set{A}} &
      \J{\GD,x\of A}{\GG}{e_2}{L}}
    \and
    \infer[\rn{\{:\}}]{\J{\GD}{\GG}{\singlemap{e_1}{e_2}}{\Map{A}{B}}}{
      \J{\GD}{\cdot}{e_1}{A} &
      \J{\GD}{\GG}{e_2}{B}}
    \and
    \infer[\rn{\rt{fix}}]{\J{\GD}{\GG}{\fix{\m{x}}{e}}{\fineq{L}}}{
      \J{\GD}{\GG,x\of L}{e}{\fineq{L}}}
    \and
    \infer[\rn{\rt{fix}_{\le}}]{
      \J{\GD}{\GG}{\fixle{\m{x}}{e_1}{e_2}}{\eq{L}}
    }{
      \J{\GD}{\GG}{e_1}{\eq{L}} &
      \J{\GD}{\GG,x \of \eq{L}}{e_2}{\eq{L}}}
  \end{mathpar}

  \caption{Typing rules for core Datafun}
  \label{fig:typing-rules}
\end{figure*}


\section{Typing rules}

Datafun's typing judgment $\J{\GD}{\GG}{e}{A}$ is defined by the inference rules
given in Figure \ref{fig:typing-rules}. We gloss $\J{\GD}{\GG}{e}{A}$ as
follows: ``expression $e$ has type $A$ using variables from $\GD \cup \GG$, and
moreover the value of $e$ is \emph{monotone} with respect to the variables in
$\GG$''.

The context $\GD$ types ordinary variables; $\GG$, monotone variables. Both
admit the usual structural rules of exchange, weakening, and contraction.
Variables from either context may be used freely (rules \rt{var}, $\rt{var}^+$).
\todo{explain?}

\subsection{Functions and application}
Two function types require two function introduction rules: the ordinary
$\lambda$ and the monotone $\lambda^+$. These simply introduce variables into
their respective contexts. Monotone function application $\rt{app}^+$ is
perfectly standard, but ordinary function application \rt{app} has a pecularity:
the argument $e_2$ gets an \emph{empty} monotone context.

To understand why, recall our gloss: the application $e_1\;e_2$ must be monotone
in $\GG$. But $e_1$ is an ordinary, and in general \emph{non-monotone}, function
$A \uto B$: there is no guarantee that it respect any order on its argument. We
work around this scoff-law behavior on $e_1$'s part by ensuring its argument
$e_2$ is \emph{constant} with respect to $\GG$---which we accomplish by simply
prohibiting $e_2$ from using any of $\GG$'s variables.

This technique of \emph{wiping clean} the monotone context to guarantee
constancy\footnote{Wherever we write ``constant'' in this section, substitute
  ``constant with respect to the monotone context''. The ordinary context is
  never ``wiped clean'', and behaves entirely as it would in a simply-typed
  $\lambda$-calculus.} of a subterm recurs in several other rules. \todo{Mention
  comonads and modal logic?}

\subsection{Products and sums}
The pairing and projection rules, \rt{pair} and $\pi$, are completely standard,
as is the \rt{in} rule for sum introduction. Sum elimination, however, splits
into two rules, \rt{case} and $\rt{case}^+$. $\rt{case}^+$ requires its branches
to be monotone in the variable $\m{x}$ it introduces, and consequently its
subject $e$ is permitted access to the monotone context $\GG$. \rt{case},
however, analyses its subject $e$ as a constant --- wiping clean its monotone
context --- and thus is allowed to introduce the variable $x$ into the ordinary
context $\GD$. \todo{rewrite for clarity}

\subsection{Booleans}
While \rt{true} and \rt{false} are straightforward, there are two rules for
boolean elimination, \rt{if} and $\rt{if}^+$. This is because in Datafun, $1$
plus $1$ does not equal $\bool$: booleans are \emph{not} a sum of
units.\footnote{For simplicity, we have omitted the unit type 1 from our
  presentation of Datafun, but it is easy enough to imagine including it.} At
the type $1 + 1$, $\ms{in}_1 \triv$ and $\ms{in}_2 \triv$ are incomparable. But
in Datafun, $\ms{true} > \ms{false}$. Therefore, to eliminate a boolean in a
monotone fashion, one must ensure one's \emph{then}-branch is always greater
than one's \emph{else}-branch.

Thus Datafun has two \ms{if} rules. First, \rt{if}, where the boolean $e$ being
analysed is constant (has an empty monotone context), and so the branches $e_1$,
$e_2$ may be arbitrary expressions.

Second, $\rt{if}^+$, where the subject $e$ has full access to $\GG$, but the
\ms{if}-expression must have \emph{semilattice type}, and the \emph{else}-branch
is constrained to be $\unit$ --- the least value, thus smaller than $e_1$.

This is a conservative approach: there are many semantically monotone, but
untypeable, \ms{if}-terms. However, it is complete for semilattice types, for in
that case $(\ifthen{e}{e_1}{e_2})$ may be rewritten $(e_2 \vee
\ifthen{e}{e_1}{\unit})$; as long as $e_1 \ge e_2$ and so $e_2 \vee e_1 = e_1$,
this will not change the meaning of the expression.\footnote{If Datafun were
  Turing-complete, this would no longer be the case; consider the case where
  $e_2$ diverges.}

Thus the only meaningful restriction here is to semilattice types. In practice,
we have yet to find a case where this is problematic.

\todo{Mention that $e_1 \ge e_2$ is in general statically undecidable (rice's
  theorem), and that a typesystem to check it would amount to dependent typing?}

\subsection{Semilattices and sets}
The semilattice $\unit$ and $\vee$ operations are typed by their
correspondingly-named rules. Since $\vee$ is by definition monotone, its
arguments have full access to the monotone context $\GG$.

Sets are simply the free semilattice on an underlying set, so aside from the
semilattice rules,

\subsection{Maps}
Finite maps are an abstract type, and most of their interface is exposed through
primitive functions (see Figure \ref{fig:primitives}). The only typing rule is
$\{:\}$, the introduction rule for singleton maps. By analogy with the
introduction rule for singleton sets $\{\}$,

\subsection{Fixed points}

\newpage


%% Section 5: Semantics
\begin{figure}
   \tikzset{
   no line/.style={draw=none,
     commutative diagrams/every label/.append style={/tikz/auto=false}}}
\begin{center}
{\large
   \begin{tikzcd}
        \mathbf{\mathbf{Set}}     \arrow[bend left=35]{r}[name=F]{\mathsf{Disc}}
                                  \arrow[rr, bend left=60, "\mathsf{FS}"]
      & \mathbf{\mathbf{Poset}}   \arrow[bend left=35]{l}[name=U]{|\mathit{-}|}
                                  \arrow[to path={(F) -- (U)\tikztonodes}, no line]{}{\dashv}
                                  \arrow[bend left=35]{r}[name=H]{\mathsf{Sups}}
      & \mathbf{\mathbf{SemiLat}} \arrow[bend left=35]{l}[name=K]{U}
                                  \arrow[to path={(H) -- (K)\tikztonodes}, no line]{}{\dashv}
   \end{tikzcd}
}
\end{center}
  \caption{Semantic categories of Datafun}
  \label{fig:sem-cats}
\end{figure}

\begin{figure}
  \begin{center}
    \begin{tabular}{cl}
      \multicolumn{2}{c}{\textbf{Set notation}}\\
      $\U{P}$ & Underlying set of the poset $P$\\
      $\stringset$ & Set of strings\\
      $A \boxtimes B$ & Cartesian product of sets $A$, $B$\\
      $A \boxplus B$ & Disjoint union of sets $A$, $B$\\
      $A \Arr B$ & Functions from set $A$ to set $B$
      \vspace{0.5em}\\
      \multicolumn{2}{c}{\textbf{Poset and semilattice notation}}\\
      \one & One-element poset $\{\triv\}$\\
      \two & Two-element poset $\{\sff,\stt\}$, with $\sff < \stt$\\
      $\N_\le$ & The naturals $\N$, as a (totally ordered) poset\\
      $P + Q$ & Disjointly-ordered poset on disjoint union of $P,Q$\\
      $P \x Q$ & Pointwise poset on pairs of $P$s and $Q$s\\
      $P \arr Q$ & Pointwise poset on monotone maps $\cPoset(P, Q)$\\
      %% $L \lol M$ & Pointwise poset on $\cSL(L,M)$\\
      $\Disc{A}$ & Discrete poset on underlying set $A$\\
      $\Sups{P}$ & Free semilattice on a poset $P$\\
      $\FS{A}$ & Free semilattice on a set $A$; same as $\Sups{(\Disc{A})}$\\
      $\mathsf{U}{L}$ & Underlying poset of a seminlattice $L$
      %% \\ $\FM{A}{P}$ & Poset of finite maps from the set $A$ to the poset $P$
    \end{tabular}
  \end{center}

  \caption{Semantic notation}
  \label{fig:sem-notation}
\end{figure}

%% \todo{``Finite maps'' ambiguous terminology? Gibbons uses it differently, for
%%   example.}


\section{Semantics}
\label{sec:semantics}

We give a denotational semantics for Datafun in terms of three categories
(\cSet{}, \cPoset{}, and \cSL{}) and two adjunctions between them (see Figure
\ref{fig:sem-cats}). We use nonstandard notation to avoid confusion between sets
and posets (see Figure \ref{fig:sem-notation}). We take less care to distinguish
posets and semilattices, since while a set can be partially ordered in many
ways, a poset either \emph{is} or \emph{is not} a semilattice.

\subsection{The category \cSL{}}

\cSL{} is the category of join-semilattices with least elements, which we call
simply ``semilattices''.

Directly, a semilattice is a poset $L$, with a least element $\unit$, in which
any two elements $a,b$ have a least-upper-bound $a \vee b$. From $\unit$ and
$\vee$ it follows that any finite subset $X \subseteq_{\ms{fin}} \U{L}$ has a
least upper bound, written $\bigvee X$.

A morphism $f \in \cSL(L, M)$ is a function from $\U{L}$ to $\U{M}$ satisfying:
\begin{eqnarray*}
  f(a \vee_A b) &=& f(a) \vee_B f(b)\\
  f(\unit_A) &=& \unit_B
\end{eqnarray*}

\cSL{} is a subcategory of \ms{Poset}; every \cSL{}-morphism $f$ is monotone,
since $a \le b \iff a \vee b = b$, and so from $a \le b$ we know $f(a) \vee f(b)
= f(a \vee b) = f(b)$, thus $f(a) \le f(b)$. Since it is a subcategory, we will
typically not explicitly write the forgetful functor $\mathsf{U}(L)$ which sends
semilattices to posets by forgetting the lattice structure.


\subsection{Denotation of Datafun types}
\begin{figure}
  \[\begin{array}{rcl}
  \den{A} &\in& \cPoset_0\\
  \den{\bool} &=& \two\\
  \den{\N} &=& \N_\le\\
  \den{\str} &=& \Disc{\mathbb{S}}\\
  \den{A \x B} &=& \den{A} \x \den{B}\\
  \den{A + B} &=& \den{A} + \den{B}\\
  \den{A \mto B} &=& \den{A} \arr \den{B}\\
  \den{A \uto B} &=& \Disc{\U{\den{A}}} \arr \den{B}\\
  \den{\Set{A}} &=& \FS{\U{\den{A}}}%% \\
  %% \den{\Map{A}{B}} &=& \FM{\U{\den{A}}}{\den{B}}
  %% \end{array}\]\[\begin{array}{rcl}
  \\\\
  \den{\GD}, \den{\GG} &\in& \cPoset_0\\
  \den{\cdot} &=& \one\\
  \den{\GD, x\of A} &=& \den{\GD} \x \den{A}\\
  \den{\GG{}, \m{x}\of A} &=& \den{\GG} \x \den{A}
  \end{array}\]
  \caption{Denotations of Datafun types and contexts}
  \label{fig:sem-types}
\end{figure}

Datafun types and contexts denote posets as shown in Figure \ref{fig:sem-types}.
Datafun's semilattice types denote posets which are also semilattices---and
moreover, semilattices where the least-upper-bound operator $\bigvee$ is
\emph{computable}. \todo{should this have a proof?}


\subsection{Denotation of Datafun terms}
\begin{figure}
  \TODO
  \caption{Denotations of Datafun typing derivations}
  \label{fig:sem-terms}
\end{figure}

In Figure \ref{fig:sem-terms} we give a denotation for typing derivations with
the following signature:
\begin{eqnarray*}
  \den{\J{\GD}{\GG}{e}{A}} &\in&
  \cSet(\U{\den{\GD}}, \cPoset(\den{\GG}, \den{A}))
  %% \\ &\in&
  %% \U{\den{\GD}} \Arr \U{\den{\GG} \arr \den{A}}
\end{eqnarray*}

Colloquially, $\J{\GD}{\GG}{e}{A}$ denotes a function from $\den{\GD} \x
\den{\GG}$ to $\den{A}$ that must be monotone in $\den{\GG}$ (but not in
$\den{\GD}$).


%% Section 6: Implementation
\section{Implementation}
We have built a proof-of-concept implementation of Datafun in Racket
\todo{CITE}. In addition to core Datafun, it supports pattern-matching, variant
types, record types, subtyping, antitone functions, and unbounded (potentially
nonterminating) fixed points. It performs no optimizations whatsoever.

\todo{LINK}


%% Section 7: Tradeoffs, etc.
%% not sure how many sections to split this up into.
\section{Tradeoffs, limitations, and design decisions}

\paragraph{Finite and bounded fixed-points} \TODO Discuss
Datalog's constructor restriction (name?), not being sure how to encode it in a
type theory, its disadvantages, and the disadvantages of our approaches ---
finite \& bounded fixed-points.

bounded fixed-points strictly more general than finitary-type fixed-points, but
at possible runtime cost, but finite types very restrictive in practice

\TODO In particular, could bounded fix-points have adverse performance
implications?

\paragraph{Termination} \TODO nontermination and declarativeness.

For example, one powerful optimization technique is \emph{loop reordering} (in
SQL terminology, \emph{join reordering}), that is, taking advantage of the
equation
\begin{eqnarray*}
  \forin{x \in e_1} \forin{y \in e_2} e
  &=& \forin{y \in e_2} \forin{x \in e_1} e
\end{eqnarray*}
when $x,y \notin \ms{FV}(e_1) \cup \ms{FV}(e_2)$. (\TODO Explain why join
reordering is powerful). But this equation does not always hold in the presence
of nontermination; for example, if $e_1 = \unit$ and $e_2$ diverges. \TODO
finish up

\paragraph{Type inference} Typechecking needs to distinguish between ordinary
and monotone $\lambda$, application, \ms{case}, \ms{let}, and \ms{if}. In our
implementation we solve this in two ways:
\begin{enumerate}
\item Bidirectional type inference \todo{CITE} determines whether $\fn$s and
  applications are ordinary or monotone.
\item For $\ms{if}$, $\ms{case}$, and $\ms{let}$, the programmer annotates which
  form is intended; for example, $(\ifthen{e}{e_1}{\unit})$ is written
  (\texttt{when e then e1}) to indicate the rule $\ms{if}^+$ applies.
\end{enumerate}

It remains an open question how much annotation is necessary. See
Section \ref{sec:futurework}, \emph{Related and future work}.

\paragraph{User-defined usls}
The two fundamental usl types Datafun provides are booleans and sets; products
and functions merely preserve usl structure where they find it. One might
contemplate allowing the programmer to define their own usl structures using
something like Haskell's \texttt{newtype}/\texttt{instance}. Unfortunately, this
\TODO
\begin{itemize}
\item User-defined functions give the compiler no guarantee that they are
  commutative, associative, and idempotent.
\item Without further information it would be impossible to
\end{itemize}

Unfortunately, there is no simple characterization of the space of all possible
computable usls.

\paragraph{Lexical types}
The category \cPoset{} has a much richer structure than \cSet{} (which may
indeed be seen as a subcategory of \cPoset{}).

\todo{\begin{itemize}
\item Pro: allow expressing many things more generally, for example, map lookup
\item Con: have no good general monotone elimination rule
\item Con: complicate the type theory
\end{itemize}}


%% Section 8: Related & future work
\section{Comparing Datalog and Datafun}
\label{sec:datalog-vs-datafun}

At this point, we have
demonstrated by example that Datafun programs are rather similar to
Datalog programs, and we have given the typing and denotational
semantics of Datafun. However, we still need to explain \emph{why} our
semantics lets us express Datalog-style programs.

To understand this, recall that Datalog is a bottom-up logic
programming language. A program consists of a primitive database of
facts, along with a set of rules the rules the programmer wrote. A
Datalog program executes by using the rules to derive new conclusions
from the database, and extending the database with them, until no
additional conclusions can be drawn. Then the query can be checked
simply by seeing if it occurs in the final database.

This is, essentially, a fixed point computation -- each stage of
execution of a Datalog program takes a database and returns an
extended database, until a fixed point is reached. The stratified
negation restriction essentially ensures that the database transformer
defined by a Datalog program is a monotone function on the set of
facts. This is why the type system of Datafun tracks the monotonicity
of functions --- since we permit both higher-order definitions and
taking fixed points, we need to ensure that the body of a fixed point
definition is monotone in order to guarantee that the recursion is
well-founded.

This ensures that the recursive definition is well-defined, but is not
sufficient by itself to guarantee termination. To manage this, Datalog
depends upon the other two restrictions described in the
introduction.By restricting terms occuring in predicates to consist of
either atoms or variables, Datalog ensures that quantifiers need only
be instantiated with the atoms used in a program. By requiring every
variable in the consequent of rules to also occur in the premise of a
rule, it ensures that every consequent will also only feature atoms
occuring in the original program.

Then, since there can only be finitely many atoms in a finite program,
this means that the set of possible arguments to a predicate is itself
finite. Then the lattice of sets of atomic predicates ordered by
inclusion will be finite, and so fixed point iteration is guaranteed
to terminate.

Instead of this (rather indirect) scheme, Datafun directly tracks the
finiteness of types, permitting recursion only if it is over a finite
type, or is bounded explicitly. These two approaches achieve the same
effect, albeit in different ways. Datalog's approach has the benefit
that no type discipline is needed to ensure finiteness. One advantage
of our choice is that we permit recursion over any semilattice, not
just the semilattice of sets. A much more serious advantage of our
approach is that it makes it much easier to write fixed-point
computations which actually \emph{compute} with the data they see.



\section{Other Related and Future Work}
\label{sec:futurework}

\paragraph{Deletion} \citet{logical-algorithms} showed how
forwards-chaining logic programming permits concise and elegant
expression of a wide variety of algorithms, including a natural cost
semantics. However, they noted that there were some algorithms (such
as union-find and greedy algorithms) which could be formulated in this
style, \emph{if} there were additionally support for deleting facts
from a database. Later, \citet{linear-logical-algorithms} went on to
show how deletion could be given a logical interpretation by
formulating in terms of linear logic programming.

This naturally raises the question of whether we could identify a
``linear Datafun'' corresponding to this style of programming, where
we might linear types to model features like deletion. There are many
nontrivial semantic issues (e.g., how to define monotonicity), but
it seems a promising question for future work.



\paragraph{Optimization}
\begin{itemize}
\item \TODO the datalog literature
\end{itemize}

\paragraph{Polymorphism}
\begin{itemize}
\item \TODO quantification over different classes of type variable (ordinary,
  equality, lattice); amounts to a typeclass system, so this is not new work.
\item \TODO tone polymorphism and why you need it for principal types; e.g.
  what is the type of $\fn\bind{f}\fn\bind{x} f\;x$?
\end{itemize}

\paragraph{Type inference} blah

\TODO Ref Dunfield \& Krishnaswami, higher-order bidirectional type inference.

\todo{REWRITE} We speculate that bidirectional inference could be replaced by a
Damas-Milner \todo{CITE} style algorithm, which infers a principal type for any
term without any annotation at all, \emph{if} we add polymorphism,
tone-polymorphism, and subtyping---so that, for example, $\fn\bind{f}\fn\bind{x}
f\;x$ can be assigned the principal type
$\forall\bind{o\of\ms{tone}}\forall\bind{\alpha,\beta \of \ms{type}} (\alpha
\overset{o}\to \beta) \mto (\alpha \overset{o}\to \beta)$, where
$\overset{o}\to$ indicates a function of tone $o$; a tone may be empty (for an
ordinary function) or ${+}$ for a monotone function.



%% ---------- End matter ----------
\acks

Acknowledgments, if needed.

% We recommend abbrvnat bibliography style.
\bibliographystyle{abbrvnat}
\bibliography{datafun}

%% \appendix
%% \section{Appendix Title}

%% This is the text of the appendix, if you need one.


\end{document}
