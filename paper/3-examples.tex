%% FIGURE: SYNTAX SUGAR
\begin{figure}
  \[\begin{array}{lccl}
  %% expressions
  \textsf{terms} &
  e &\bnfeq& ... \pipe e \isin e \pipe \setfor{e}{\mc{L}}
             \pipe \forin{\mc{L}}{e}\\
  &&& \mathcal{C}[\;e]^* \pipe \rawcase{e}{[{p} \cto {e}]^*}
  \vspace{0.5em}\\
  %% patterns
  %%
  %% TODO: maybe remove the pattern-matching stuff? since we don't explain how
  %% to translate it & we also use various other sugar we don't explain how to
  %% translate, why do we include only pattern-matching here?
  \textsf{patterns} &
  p &\bnfeq& \pwild \pipe x \pipe (p,p)
             \pipe \ms{true} \pipe \ms{false} \pipe \mathcal{C}[\;p]^*
  \vspace{0.5em}\\
  \textsf{constructors} & \mathcal{C} && \text{are abstract identifiers}
  \vspace{0.5em}\\
  %% loop clauses
  \textsf{loops} &
  \mc{L} &\bnfeq& \mc{L}, \mc{L} \pipe p \in e \pipe e
  \end{array}\]

  %% the desugaring syntax-expansion itself
  \begin{eqnarray*}
    %% e_1 \isin e_2            &\expandsto& \forin{x \in e_2} e_1 = x\\
    \setfor{e}{\mc{L}}       &\expandsto& \forin{\mc{L}}{\{e\}}\\
    \forin{\mc{L}_1,\mc{L}_2}{e}
    &\expandsto& \forin{\mc{L}_1}{\forin{\mc{L}_2}{e}}\\
    \forin{p\in e_1}{e_2} &\expandsto&
    \letin{x}{e_1}{\rawcase{x}{p \cto e_2;\,\pwild \cto \unit}}\\
    \forin{e_1}{e_2} &\expandsto& \ifthen{e_1}{e_2}{\unit}
    %% e_1 \x e_2 &\expandsto& \setfor{(a,b)}{a \in e_1, b \in e_2}\\
    %% e_1 \bullet e_2 &\expandsto& \setfor{(a,c)}{(a,b) \in e_1, (b,c) \in e_2}
    %% \ms{let}~x = e_1 ~\ms{in}~ e_2
    %% &\expandsto& (\fn\bind{x} e_2)\; e_1\\
    %% \ms{let}~[x_i = e_i]^* ~\ms{in}~ e
    %% &\expandsto& [\ms{let}~x_i = e_i~\ms{in}]^* e\\
    %% \rawcase{e}{[p \cto e]^*} &\expandsto& \text{(omitted, see \todo{CITE})}
  \end{eqnarray*}
  \caption{Syntax sugar}
  \label{fig:sugar}
\end{figure}


\section{Examples}

For purposes of these examples, we use a simple Haskell-like syntax for
top-level type and function definitions. We also permit ourselves infix
notation, \ms{let}-binding, $n$-ary tuples, $n$-ary sum types with named
constructors, pattern-matching\todo{(CITE)}, and additional syntax sugar given
in Figure \ref{fig:sugar} \todo{(TODO: mention \& cite monadic query syntax)}.
All of these conveniences are supported (with slightly different concrete
syntax) in our implementation.

For clarity, we set the names of top-level variables in \textsf{sans-serif};
ordinary variables in $script$ or \mi{italic} (for long variable names); and
monotone variables in \m{bold}.

\todo{TODO: Explain nonlinear variable usage means equality.}

\todo{\begin{itemize}
\item composition of relations?
\item \texttt{make}-style topological sort?
\item SQL-style examples?
\item translating relational algebra into datafun?
\end{itemize}}


\subsection{Filtering, mapping, and cross products}

Although Datafun as presented does not have polymorphism, we give our examples
their most general possible type schemes. \todo{why?}

\todo{these examples are perhaps a good place to point out subtleties of
  monotonicity typing}

\[\begin{array}{l}
\fname{map} ~:~ (A \uto B) \uto \Set{A} \mto \Set{B}\\
\fname{map}\;f\;\m{A} = \setfor{f\;x}{x \in \m{A}}\\
\\
\fname{filter} ~:~ (A \uto \bool) \mto \Set{A} \mto \Set{A}\\
\fname{filter}\;\m{f}\;\m{A} = \setfor{x}{x \in \m{A}, \m{f}\; x}\\
\\
(\times) ~:~ \Set{A} \mto \Set{B} \mto \Set{A \x B}\\
\m{A} \times \m{B} = \setfor{(a,b)}{a \in \m{A}, b \in \m{B}}
\end{array}\]

\TODO


%% FIGURE: PRIMITIVES
\begin{figure}
  %% TODO: remove unused primitives.
  \[\begin{array}{cll}
  \neg &\of& \bool \uto \bool\\
  =   &\of& \eq{A} \uto \eq{A} \uto \bool\\
  \le &\of& \eq{A} \uto \eq{A} \mto \bool\\
  %% \fname{keys}     &:& \Map{A}{B} \mto \Set{A}\\
  %% \fname{entries}  &:& \Map{A}{B} \uto \Set{A \x B}\\
  %% \fname{tabulate} &:& \Set{A} \mto (A \uto B) \mto \Map{A}{B}\\
  %% \fname{getWith}  &:& \Map{\eq{A}}{B} \mto \eq{A} \uto (B \mto L) \mto L\\
  %% \fname{get}      &:& \Map{\eq{A}}{L} \mto \eq{A} \uto L\\
  %% \fname{substrings} &\of& \ms{Str} \uto \Set{\ms{Str}}\\
  %% \fname{size}     &:& \Set{\eq{A}} \mto \N\\
  \fname{range}    &:& \N \uto \N \mto \Set{\N}\\
  \fname{length}   &:& \str \uto \N\\
  \fname{substring} &:& \str \uto \N \uto \N \uto \str
  \end{array}\]
  \caption{Primitive functions and their type schemes}
  \label{fig:primitives}
\end{figure}


\subsection{Testing for membership}

\todo{Use this as example of $\bigvee$ at non-set (here, boolean) type.}

\[\begin{array}{l}
(\isin) ~:~ \eq{A} \uto \Set{\eq{A}} \mto \bool\\
x \isin \m{A} = \forin{y \in \m{A}} x = y
\end{array}\]

\todo{TODO: discuss nonlinear pattern-matching}

An equivalent definition, making use of nonlinear pattern-matching, would be:
\[\begin{array}{l}
x \isin \m{A} = \forin{x \in \m{A}} \ms{true}
\end{array}\]

\todo{Nonlinear pattern matching makes implicit the equality test and attendant
  restriction to equality types, but is often extremely convenient.}


\subsection{Composition of relations}

\todo{mention eqtypes again here? or use this to show them off instead of
  $({\isin})$?} \todo{work phrase ``higher-order'' in here somewhere?}

\noindent As an example of a function over relations, consider:
\[\begin{array}{l}
(\bullet) : \Set{A \x \eq{B}} \mto \Set{\eq{B} \x C} \mto \Set{A \x C}\\
\m{R} \bullet \m{S} = \setfor{(a,c)}{(a,b) \in \m{R}, (b,c) \in \m{S}}
\end{array}\]

\todo{TODO: mention you can extend to all relational algebra? give example?}


\subsection{Transitive closure}

Consider the following Datalog program:
\begin{verbatim}
parent(earendil, elrond).
parent(elrond, arwen).
ancestor(X, Y) :- parent(X, Y).
ancestor(X, Z) :- ancestor(X, Y), ancestor(Y, Z).
\end{verbatim}

\todo{Discuss how this works in Datalog, but not in Prolog, b/c Prolog is
  defined by operational semantics of unification while Datalog is denotational,
  least-model semantics. It also works in Datafun!}

\todo{Neel suggests using distinction b/w backward \& forward chaining here,
  rather than operational/denotational. see Logical Algorithms paper by
  McAllister \& co for phrasing?}

In Datafun, we may write this as:

\[\begin{array}{l}
\mathbf{data}~\ms{person} =
\ctor{E\"arendil} ~|~ \ctor{Elrond} ~|~ \ctor{Arwen}\\
\fname{parent},~\ms{ancestor} ~:~ \Set{\ms{person} \x \ms{person}}\\
\ms{parent} =
\{(\ctor{E\"arendil}, \ctor{Elrond}), (\ctor{Elrond}, \ctor{Arwen})\}\\
\ms{ancestor} = \fix{\m{X}} \ms{parent} \vee (\m{X} \bullet \m{X})
%% \setfor{(a,c)}{(a,b) \in \m{X}, (b,c) \in \m{X}}
\\
\end{array}\]


\subsubsection{Transitive closure with an upper bound}

If you know the type over which you are taking your transitive closure is
finite, this suffices. But the more general way to take a fixed-point is to give
an upper bound which you know the desired answer will not exceed. For this we
write $(\fixle{\m{x}}{e_\top} e)$

\todo{mention using $\str$ as example}

\todo{explain what happens when $e \ge e_\top$ --- it gets clamped}

\[\begin{array}{l}
\ms{person} ~:~ \Set{\str}\\
\ms{person} = \{\texttt{"e\"arendil"}, \texttt{"elrond"}, \texttt{"arwen"}\}\\
\ms{parent}, \ms{ancestor} ~:~ \Set{\str \x \str}\\
\ms{parent} = \{(\texttt{"e\"arendil"}, \texttt{"elrond"}),
(\texttt{"elrond"}, \texttt{"arwen"})\}\\
\ms{ancestor} = \fixle{\m{X}}{(\ms{person} \x \ms{person})}
\ms{parent} \vee (\m{X} \bullet \m{X})
%% \ms{ancestor} = \fixle{\m{X}}{%
%%   \setfor{(a,b)}{a \in \ms{person}, b \in \ms{person}}
%% }
%% \\
%% \hspace{4.87em}\ms{parent} \vee
%% \setfor{(a,c)}{(a,b) \in \m{X}, (b,c) \in \m{X}}
\end{array}\]


\subsubsection{Generic transitive closure}

For any finite eqtype $\fineq{A}$, we may write:
\[\begin{array}{l}
\ms{trans} ~:~ \Set{\fineq{A} \x \fineq{A}} \mto \Set{\fineq{A} \x \fineq{A}}
\vspace{0.3em}\\
%% \ms{trans}\ E = \fix{X} E \vee \setfor{(a,c)}{(a,b) \in E, (b,c) \in X}
%% \ms{trans}\ \m{E} = \fix{\m{X}} \m{E} \vee %
%% \setfor{(a,c)}{(a,b) \in \m{X}, (b,c) \in \m{X}}
\ms{trans}\ \m{E} = \fix{\m{X}} \m{E} \vee (\m{X} \bullet \m{X})
\end{array}\]

\noindent Similarly, for any eqtype $\eq{A}$, we may write:
\[\begin{array}{l}
\ms{trans} ~:~
\Set{\eq{A}} \mto \Set{\eq{A} \x \eq{A}} \mto \Set{\eq{A} \x \eq{A}}
\vspace{0.3em}\\
%% \ms{trans}\ \m{V}\ \m{E} = %
%% \ms{fix}~ \m{S} \le \setfor{(a,b)}{a\in \m{V}, b \in \m{V}}\\
%% \hspace{5.35em}\ms{is}~ \m{E} \vee %
%% \setfor{(a,c)}{(a,b) \in \m{S}, (b,c) \in \m{S}}\\
\ms{trans}\ \m{V}\ \m{E} = %
\fixle{\m{S}}{(\m{V} \x \m{V})} \m{E} \vee (\m{S} \bullet \m{S})
\end{array}\]

\TODO This already shows how we go beyond Datalog's .


\subsection{CYK parsing}
Parsing can be understood logically, with a parse tree representing a
proof that a certain string belongs to a language described by a
context-free grammar. As a result, it is possible to formulate parsing
in terms of proof search~\cite{deductive-parsing}. One of the
simplest algorithms for parsing context free grammars is the
Cocke-Younger-Kasami (CYK) algorithm for parsing with grammars in
Chomsky normal form.\footnote{In Chomsky normal form, each production
  is of the form $A \to B \cdot C$ or $A \to \vec{a}$, with $A,B,C$
  ranging over nonterminals, and $\vec{a}$ over nonempty strings of
  terminals.}  Given a grammar $G$, we begin by introducing a family
of predicates (sometimes called \emph{facts} or \emph{items}) $A(i,j)$,
with one $A$ for each nonterminal, and $i$ and $j$ representing
indices into a string. Given a word $w$, we write $w[i,n]$ for the
$n$-element substring of $w$ beginning at position $i$. Then, we can
specify the CYK algorithm with the following two inference rules:
\begin{mathpar}
  \inferrule*{B(i, j) \\ C(j, k) \\ (A \to B\; C) \in G}
             {A(i, k)}
  \and
  \inferrule*{ (A \to \vec{a}) \in G \\ w[i,n] = \vec{a} }
             {A(i,i+n)}
\end{mathpar}
Then, the predicate $A(i,j)$ means that $A$ is derivable from the
substring of $w$ running from $i$ to $j$, and so the whole word $w$ is
derivable from the start symbol $S$ if $S(0, \mathit{length}\;w)$ is
derivable.

In Datafun, this rule-based description of the algorithm can be
transliterated almost directly into code. We begin by introducing a
few basic types.
\[\begin{array}{l}
\mathbf{type}~\ms{sym} = \str\\
\mathbf{data}~\ms{rule} = \ctor{String}~\str ~|~ \ctor{Concat}~\ms{sym}~\ms{sym}\\
\mathbf{type}~\ms{grammar} = \Set{\ms{sym} \x \ms{rule}}\\
\mathbf{type}~\ms{fact} = \ms{sym} \x \N \x \N\\
\end{array}\]
The $\ms{sym}$ type is a type synonym representing nonterminal names
with strings.  The $\ms{rule}$ type is the type of the
right-hand-sides of productions Chomsky normal form -- either a
string, or a pair of nonterminals. A $\ms{grammar}$ is just a set of
productions -- a set of pairs of nonterminals paired with their rules.
The type $\ms{fact}$ is the type representing the atomic facts derived
by the CYK inference system -- they are triples of the rulename, the
start position, and the end position.

With these types in hand, we can write the CYK algorithm as a fixed
point computation. In fact, it is convenient to break it into two
pieces, by first defining the function whose fixed point we take. So
we can write down the $\fname{iter}$ function, which represents one step of
the fixed point iteration.
\[\begin{array}{l}
\fname{iter} ~:~ \str \uto \ms{grammar} \mto \Set{\ms{fact}} \mto \Set{\ms{fact}}\\
\fname{iter} \;\mi{text} \;\m{G} \;\m{chart} =\\
\hspace{1em}\phantom{\vee~}
\{(a,i,k) ~|~ (a, \ctor{Concat}~b~c) \in \m{G},\\
\hspace{6.25em} (b,i,j) \in \m{chart}, (c,j,k) \in \m{chart}\}\\
\hspace{1em}\vee~ \{(a,i,i+\fname{length}\;s)\\
\hspace{2.1em}|~ (a, \ctor{String}~s) \in \m{G},\\
\hspace{2.2em}\phantom{|~} i \in \fname{range}\;0\;(n-\fname{length}\;s),\\
\hspace{2.2em}\phantom{|~}
s = \fname{substring} \;\mi{text} \;i \;(i+\ms{length}\;s)\}
\end{array}\]
This function works by taking a string $\mi{text}$ and a grammar $\m{G}$, and
then taking a set of facts $\m{chart}$, and taking a union. The first clause
is a set comprehension, saying that we return $(a, i, k)$ if $(b, i, j)$ and $(c, j, k)$
are in $\m{chart}$ -- this corresponds to applications of the first rule. The
second clause corresponds to the second rule above, saying that $(a, i, i + length(s))$
is a generated fact if $s$ is a substring of $\mi{text}$ at position $i$.

We can then use $\fname{iter}$ to implement the $\fname{parse}$ function.
%% parse
\[\begin{array}{l}
\fname{parse} ~:~ \str \uto \ms{grammar} \mto \Set{\ms{sym}}\\
\fname{parse} \;\mi{text} \;\m{G} =\\
\hspace{1em} \ms{let}~ n = \ms{length} \;\mi{text}\\
\hspace{2.375em}\m{bound} =
  \{(a,i,j) ~|~ (a,\pwild) \in \m{G},\\
\hspace{10.5em}i \in \ms{range}\;0\;n, \\
\hspace{10.5em}j\in\fname{range}\;i\;n\}\\
\hspace{2.375em} \m{chart} = \fixle{\m{C}}{\m{bound}}
  \ms{iter} \;\mi{text} \;\m{G} \;\m{C}\\
\hspace{1em}\ms{in}~\setfor{a}{(a, 0, n) \in \m{chart}}\\
%% %% iter with \forin
%% \\
%% \fname{iter} \;\mi{text} \;\m{G} \;\m{chart} =\\
%% \hspace{1em}\phantom{\vee~}
%% (\bigvee((a, \ctor{Concat} \;b \;c) \in \m{G},\\
%% \hspace{1.25em}\phantom{\vee~ \bigvee(}
%% (b,i,j) \in \m{chart}, (c,j,k) \in \m{chart})\\
%% \hspace{1.25em}\phantom{\vee~}\, \setlit{(a,i,k)})\\
%% \hspace{1em}\vee~ (\bigvee((a, \ctor{String} \;s) \in \m{G},\\
%% \hspace{1.25em}\phantom{\vee~\bigvee(}
%% i \in \ms{range} \;0 \;(n - \ms{length} \; s),\\
%% \hspace{1.25em}\phantom{\vee~\bigvee(}
%% s = \ms{substring} \;\mi{text} \;i \;(i+\ms{length}\;s))\\
%% \hspace{1.25em}\phantom{\vee~}\, \setlit{(a,i,i+ \ms{length}\;s)})
%% \\
%% %% iter with case. I like this version best.
%% \\
%% \fname{iter} \;\mi{text} \;\m{G} \;\m{chart} =\\
%% \hspace{1em}\forin{(a,r) \in \m{G}}\\
%% \hspace{1.875em}\ms{case}~ r\\
%% \hspace{2.4em}\ms{of}~
%% %% \hspace{3.05em}\pipe
%% \ctor{Concat} \;b \;c \cto \{(a,i,k) ~|~ (b,i,j) \in \m{chart},\\
%% \hspace{14.42em}(c,j,k) \in \m{chart}\}\\
%% \hspace{3.05em}\pipe \ctor{String} \;s \cto
%% \{\,(a, i, i+\ms{length}\;s)\\
%% \hspace{9em}|~ i \in \ms{range} \;0 \;(n-\ms{length}\;s),\\
%% \hspace{9em}\phantom{|~}
%% s = \ms{substring} \;\mi{text} \;i \;(i+\ms{length}\;s)\}
%% \\
%% iter. Neel prefers this. People know set-comprehension.
\end{array}\]
This function just takes the fixed point of $\fname{iter}$ --
almost. Because facts are triples $\ms{sym} \x \N \x \N$, sets of
facts may in general grow unboundedly.  To ensure termination, we
construct a set $\m{bound}$ to bound the sets of facts we consider in
our fixed point computation, by bounding the symbols to names found in
the grammar $\m{G}$, and the indices to positions of the string. Since
all of these are finite, we know that the computation of $\m{chart}$
as a bounded fixed point will terminate. Then, having computed the
fixed point, we can check chart to see if $(a, 0, \ms{length}\;\mi{text})$
is derivable.

There are three things worth noting about this program. First, it is
not expressible in Datalog. Because Datalog provides no way to
represent a \emph{grammar} as a piece of data (it's compound, not an
atom), there is simply no way in Datalog to express a \emph{generic}
parser taking a grammar as an input. This demonstrates one of the key
benefits of moving to a functional language like Datafun.

Moreover, Datalog's strategy for proving termination involves a
constructor restriction to ensure all relations are finite. Primitives such as
\ms{range} and \ms{substring} violate this restriction (as relations, they are
infinite); it is not immediately obvious that Datalog programs extended with
these primitives remain terminating. Our use of bounded fixed-points to
guarantee termination is robust under such extensions; as long as all primitive
functions are total, Datafun programs always terminate.

Finally, having computed a set via a fixed point, we can test whether
or not an element is in that set \emph{or not} -- the ability to test
for negative information after the fixed point computation completes
corresponds to a use of stratified negation in Datalog.


\subsection{Dataflow analysis}
In this section, we show how some simple dataflow analyses can be expressed in
Datafun. We begin with the types in these programs.
\[\begin{array}{l}
\textbf{type}~\ms{var} = \str\\
\textbf{type}~\ms{label} = \N\\
\textbf{data}~\ms{oper} = \ctor{Eq} \pipe \ctor{Le}
\pipe \ctor{Add} \pipe \ctor{Sub} \pipe\ctor{Mul}\pipe\ctor{Div}\\
\textbf{data}~\ms{atom} = \ctor{Var}\;\ms{var} \pipe \ctor{Num}\;\N\\
\textbf{data}~\ms{expr} = \ctor{Atom}\;\ms{atom}
\pipe \ctor{Apply}\;\ms{oper}\;\ms{atom}\;\ms{atom}\\
\textbf{data}~\ms{stmt} =
\ctor{Assign} \;\ms{var} \;\ms{expr}
\pipe \ctor{If} \;\ms{expr} \;\ms{label}\;\ms{label} \\
\textbf{type}~\ms{program} = \Set{\ms{label} \x \ms{stmt}}

\end{array}\]
The basic idea is that we represent a program as a kind of control
flow graph. Each node of this graph has a $\ms{label}$, which is a
natural number, and contains a statement of type $\ms{stmt}$, which is
either an assignment of an expression (of type $\ms{expr}$) to a
variable (of type $\ms{var}$), or a conditional jump.  A program is
then just the set of nodes -- i.e., a set of label, statement pairs --
with the invariant that the relation is functional (i.e., if $(l, s)$
and $(l,s')$ are both in a program, then $s = s'$).

In what follows, we use a few trivial functions whose definitions are omitted
for space reasons.
\[\begin{array}{l}
\vspace{0.5em}\\
%% omitted functions
\ms{labels} ~:~ \ms{program} \uto \Set{\ms{label}}\quad\textsf{-{}- omitted}\\
\ms{vars} ~:~ \ms{program} \uto \Set{\ms{var}} \quad\textsf{-{}- omitted}\\
\ms{uses} ~:~ \ms{stmt} \uto \Set{\ms{var}} \quad\textsf{-{}- omitted}\\
\ms{defines} ~:~ \ms{stmt} \uto \Set{\ms{var}} \quad\textsf{-{}- omitted}
\end{array}
\]
The $\ms{labels}$ function returns the set of labels in a program. The
$\ms{vars}$ function returns the set of variables used in a program (both in
expressions and as targets for assignments). The $\ms{uses}$ function
returns the set of variables used by the expressions in a statement. The
$\ms{defines}$ function returns the set of variables defined by a statement
(i.e., at most one variable -- the target of the assignment).

Given a program, we define the 1-step control flow graph with the $\ms{flow}$
function.
\[\begin{array}{l}
%% control flow
%% TODO: use long variable name for argument.
\textbf{type}~\ms{flow} = \Set{\ms{label} \x \ms{label}}\\
\fname{flow} ~:~ \ms{program} \uto \ms{flow}\\
\fname{flow}\;c = \forin{(i,s) \in c}\\
\hspace{4em}\ms{case}~ s ~\ms{of}~
\ctor{If} \;\pwild \;j \;k \cto \setlit{(i,j),(i,k)}\\
\hspace{7.45em}\pipe\pwild \cto \setfor{(i,i+1)}{i+1 \isin \ms{labels}\;c}
\vspace{0.5em}\\
\end{array}
\]
It says that if $(i, s)$ is a node of the program, then if $s$ is a
conditional jump $\ctor{If} \;\pwild \;j \;k$, then control can flow from $i$
to $j$, and from $i$ to $k$ -- i.e., we add both $(i, j)$ and $(i, k)$ to the
set of edges. Otherwise, it's an assignment, and there is just a control flows
to the next statement (i.e., we add $(i, i+1)$ to the set of edges).

Now, we can define liveness analysis, one of the classic ``backwards'' dataflow
analyses. The type of $\ms{live}$ say that given a program and its flow graph,
it returns a set of label/variable pairs, which determine a relation saying
for each label which variables are live.
%% live code analysis
\[\begin{array}{l}
\ms{live} ~:~ \ms{program} \uto \ms{flow} \uto \Set{\ms{label} \x \ms{var}}\\
\ms{live} \;\mi{code} \;\mi{flow} =\\
\hspace{2em} \fixle{\m{Live}}{ %
  \ms{labels}\;\mi{code} \x \ms{vars}\;\mi{code}}\\
\hspace{2em}\forin{(i,\mi{stmt}) \in \mi{code}}\\
\hspace{2.875em} (\phantom{\vee~}\setfor{(i,v)}{v \in \ms{uses}\;\mi{stmt}}\\
\hspace{3.2em} \vee~ \{(i,v) ~|~ (i,j) \in \mi{flow},\\
\hspace{7.4em}(j,v) \in \m{Live},\\
\hspace{7.4em}\neg (v \isin \ms{defines}\; \mi{stmt})\})
\vspace{0.5em}\\
\end{array}\]
For a statement $\mi{stmt}$ at label $i$, we say that the variable
$v$ is live at $i$ if $v$ is used by $\mi{stmt}$. The variable $v$
is also live at $i$ if control flows from $i$ to $j$, and and $v$
is live at $j$, assuming that $\mi{stmt}$ isn't a definition site for $v$.

When computing this analysis, we again need to use a bounded fixed
point, which we do by taking the Cartesian product of the labels and
variables occuring in the program.


Next, we give one of the classic forwards dataflow analyses,
reaching definitions. This analysis is used to figure out whether
an assignment (a ``definition'') can influence the value of later
expressions or not.
%% reaching definitions analysis
\[\begin{array}{l}
\ms{reachingDefinitions} ~:~ \ms{program} \uto \ms{flow}
\uto \Set{\ms{label} \x (\ms{label} \x \ms{var})}\\
\ms{reachingDefinitions} \;\mi{code} \;\mi{flow} =\\
\hspace{2em}\fixle{\m{RD}}{%
  \ms{labels}\;(\ms{labels}\;\mi{code} \x \ms{vars}\;\mi{code}) \x \mi{code} }\\
\hspace{2em}\forin{(i,\mi{stmt}) \in \mi{code}}\\
\hspace{2.875em} (
\phantom{\vee~}\setfor{((i,v), i)}{v \in \ms{defines}\;\mi{stmt}}\\
\hspace{3.2em} \vee~ \{((l,v), i) ~|~ (j,i) \in \mi{flow},\\
\hspace{8.95em}((l,v), j) \in \m{RD},\\
\hspace{8.95em}\neg(v \isin \ms{defines}\;\mi{stmt})\})
\end{array}\]
We define a function $\fname{reachingDefinitions}$ which takes a
program and a set of flows as arguments, and returns a relation of
type $\Set{(\ms{label} \x \ms{var}) \x \ms{label}}$. An entry $((l,v),
i)$ in this relation means the definition of $v$ at $l$ reaches program
point $i$.

This is then computed as a fixed point of two clauses. First, if there
is a definition $v$ at program point $i$, then $i$ is reached by that
definition. Second, if $(l,v)$ reaches $j$, and $j$ flows to $i$, then
$(l,v)$ reaches $i$ as long as $v$ is not re-defined at $i$.

As \citet{whaley-lam} observed, Datalog makes it very easy to express
dataflow analyses, and it is similarly easy in Datafun.
