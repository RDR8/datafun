\begin{figure}
   \tikzset{
   no line/.style={draw=none,
     commutative diagrams/every label/.append style={/tikz/auto=false}}}
\begin{center}
{\large
   \begin{tikzcd}
        \mathbf{\mathbf{Set}}     \arrow[bend left=35]{r}[name=F]{\mathsf{Disc}}
                                  \arrow[rr, bend left=60, "\mathsf{FS}"]
      & \mathbf{\mathbf{Poset}}   \arrow[bend left=35]{l}[name=U]{|\mathit{-}|}
                                  \arrow[to path={(F) -- (U)\tikztonodes}, no line]{}{\bot}
                                  \arrow[bend left=35]{r}[name=H]{\mathsf{Sups}}
      & \mathbf{\mathbf{SemiLat}} \arrow[bend left=35]{l}[name=K]{\ms{U}}
                                  \arrow[to path={(H) -- (K)\tikztonodes}, no line]{}{\bot}
   \end{tikzcd}
}
\end{center}
  \caption{Semantic categories of Datafun}
  \label{fig:sem-cats}
\end{figure}

\begin{figure}
  \begin{center}
    \begin{tabular}{cl}
      \multicolumn{2}{c}{\textbf{Set notation}}\\
      $\U{P}$ & Underlying set of the poset $P$\\
      $\stringset$ & Set of strings\\
      $A \boxtimes B$ & Cartesian product of sets $A$, $B$\\
      $A \boxplus B$ & Disjoint union of sets $A$, $B$\\
      $A \Arr B$ & Functions from set $A$ to set $B$
      \vspace{0.5em}\\
      \multicolumn{2}{c}{\textbf{Poset and semilattice notation}}\\
      \one & One-element poset $\{\triv\}$\\
      \two & Two-element poset $\{\sff,\stt\}$, with $\sff < \stt$\\
      $\N_\le$ & The naturals $\N$, as a (totally ordered) poset\\
      $P + Q$ & Disjointly-ordered poset on disjoint union of $P,Q$\\
      $P \x Q$ & Pointwise poset on pairs of $P$s and $Q$s\\
      $P \arr Q$ & Pointwise poset on monotone maps $\cPoset(P, Q)$\\
      %% $L \lol M$ & Pointwise poset on $\cSL(L,M)$\\
      $\Disc{A}$ & Discrete poset on underlying set $A$\\
      $\Sups{P}$ & Free semilattice on a poset $P$\\
      $\FS{A}$ & Free semilattice on a set $A$; same as $\Sups{(\Disc{A})}$\\
      $\mathsf{U}{L}$ & Underlying poset of a seminlattice $L$
      %% \\ $\FM{A}{P}$ & Poset of finite maps from the set $A$ to the poset $P$
    \end{tabular}
  \end{center}

  \caption{Semantic notation}
  \label{fig:sem-notation}
\end{figure}

%% \todo{``Finite maps'' ambiguous terminology? Gibbons uses it differently, for
%%   example.}


\section{Semantics}
\label{sec:semantics}

We give a denotational semantics for Datafun in terms of three categories
(\cSet{}, \cPoset{}, and \cSL{}) and two adjunctions between them (see Figure
\ref{fig:sem-cats}). We use nonstandard notation to avoid confusion between sets
and posets (see Figure \ref{fig:sem-notation}). We take less care to distinguish
posets and semilattices, since while a set can be partially ordered in many
ways, a poset either \emph{is} or \emph{is not} a semilattice.

\subsection{The category \cSL{}}

\cSL{} is the category of join-semilattices with least elements, which we call
simply ``semilattices''.

Directly, a semilattice is a poset $L$, with a least element $\unit$, in which
any two elements $a,b$ have a least-upper-bound $a \vee b$. From $\unit$ and
$\vee$ it follows that any finite subset $X \subseteq_{\ms{fin}} \U{L}$ has a
least upper bound, written $\bigvee X$.

A morphism $f \in \cSL(L, M)$ is a function from $\U{L}$ to $\U{M}$ satisfying:
\begin{eqnarray*}
  f(a \vee_A b) &=& f(a) \vee_B f(b)\\
  f(\unit_A) &=& \unit_B
\end{eqnarray*}

\cSL{} is a subcategory of \ms{Poset}; every \cSL{}-morphism $f$ is monotone,
since $a \le b \iff a \vee b = b$, and so from $a \le b$ we know $f(a) \vee f(b)
= f(a \vee b) = f(b)$, thus $f(a) \le f(b)$. Since it is a subcategory, we will
typically not explicitly write the forgetful functor $\mathsf{U}(L)$ which sends
semilattices to posets by forgetting the lattice structure.


\subsection{Denotation of Datafun types}
\begin{figure}
  \[\begin{array}{rcl}
  \den{A} &\in& \cPoset_0\\
  \den{\bool} &=& \two\\
  \den{\N} &=& \N_\le\\
  \den{\str} &=& \Disc{\mathbb{S}}\\
  \den{A \x B} &=& \den{A} \x \den{B}\\
  \den{A + B} &=& \den{A} + \den{B}\\
  \den{A \mto B} &=& \den{A} \arr \den{B}\\
  \den{A \uto B} &=& \Disc{\U{\den{A}}} \arr \den{B}\\
  \den{\Set{A}} &=& \FS{\U{\den{A}}}%% \\
  %% \den{\Map{A}{B}} &=& \FM{\U{\den{A}}}{\den{B}}
  %% \end{array}\]\[\begin{array}{rcl}
  \\\\
  \den{\GD}, \den{\GG} &\in& \cPoset_0\\
  \den{\cdot} &=& \one\\
  \den{\GD, x\of A} &=& \den{\GD} \x \den{A}\\
  \den{\GG{}, \m{x}\of A} &=& \den{\GG} \x \den{A}
  \end{array}\]
  \caption{Denotations of Datafun types and contexts}
  \label{fig:sem-types}
\end{figure}

Datafun types and contexts denote posets as shown in Figure \ref{fig:sem-types}.
Datafun's semilattice types denote posets which are also semilattices---and
moreover, semilattices where the least-upper-bound operator $\bigvee$ is
\emph{computable}. \todo{should this have a proof?}


\subsection{Denotation of Datafun terms}
\begin{figure}
  \TODO
  \caption{Denotations of Datafun typing derivations}
  \label{fig:sem-terms}
\end{figure}

In Figure \ref{fig:sem-terms} we give a denotation for typing derivations with
the following signature:
\begin{eqnarray*}
  \den{\J{\GD}{\GG}{e}{A}} &\in&
  \cSet(\U{\den{\GD}}, \cPoset(\den{\GG}, \den{A}))
  %% \\ &\in&
  %% \U{\den{\GD}} \Arr \U{\den{\GG} \arr \den{A}}
\end{eqnarray*}

Colloquially, $\J{\GD}{\GG}{e}{A}$ denotes a function from $\den{\GD} \x
\den{\GG}$ to $\den{A}$ that must be monotone in $\den{\GG}$ (but not in
$\den{\GD}$).
