\section{Other Related and Future Work}
\label{sec:futurework}

\paragraph{Deletion} \citet{logical-algorithms} showed how
forward-chaining logic programming permits concise and elegant
expression of a wide variety of algorithms, including a natural cost
semantics. However, they noted that there were some algorithms (such
as union-find and greedy algorithms) which could be formulated in this
style, \emph{if} there were additionally support for deleting facts
from a database. Later, \citet{linear-logical-algorithms} went on to
show how deletion could be given a logical interpretation by
formulating in terms of linear logic programming.

This naturally raises the question of whether we could identify a
``linear Datafun'' corresponding to this style of programming, where
we might linear types to model features like deletion. There are many
nontrivial semantic issues (e.g., how to define monotonicity), but
it seems a promising question for future work.

\paragraph{Optimization}
\begin{itemize}
\item \TODO the datalog literature
\end{itemize}

\paragraph{Polymorphism}
\begin{itemize}
\item \TODO quantification over different classes of type variable (ordinary,
  equality, lattice); amounts to a typeclass system, so this is not new work.
\item \TODO tone polymorphism and why you need it for principal types; e.g.
  what is the type of $\fn\bind{f}\fn\bind{x} f\;x$?
\end{itemize}

\paragraph{Type inference} blah

\TODO Ref Dunfield \& Krishnaswami, higher-order bidirectional type inference.

\todo{REWRITE} We speculate that bidirectional inference could be replaced by a
Damas-Milner \todo{CITE} style algorithm, which infers a principal type for any
term without any annotation at all, \emph{if} we add polymorphism,
tone-polymorphism, and subtyping---so that, for example, $\fn\bind{f}\fn\bind{x}
f\;x$ can be assigned the principal type
$\forall\bind{o\of\ms{tone}}\forall\bind{\alpha,\beta \of \ms{type}} (\alpha
\overset{o}\to \beta) \mto (\alpha \overset{o}\to \beta)$, where
$\overset{o}\to$ indicates a function of tone $o$; a tone may be empty (for an
ordinary function) or ${+}$ for a monotone function.
