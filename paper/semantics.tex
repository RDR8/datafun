\begin{figure}
   \tikzset{
   no line/.style={draw=none,
     commutative diagrams/every label/.append style={/tikz/auto=false}}}
\begin{center}
{\large
   \begin{tikzcd}
        \mathbf{\mathbf{Set}}     \arrow[bend left=35]{r}[name=F]{\mathsf{Disc}}
                                  \arrow[rr, bend left=60, "\mathsf{FS}"]
      & \mathbf{\mathbf{Poset}}   \arrow[bend left=35]{l}[name=U]{\U{-}}
                                  \arrow[to path={(F) -- (U)\tikztonodes}, no line]{}{\bot}
                                  \arrow[bend left=35]{r}[name=H]{\slF}
      & \mathbf{\mathbf{SemiLat}} \arrow[bend left=35]{l}[name=K]{\slU}
                                  \arrow[to path={(H) -- (K)\tikztonodes}, no line]{}{\bot}
   \end{tikzcd}
}
\end{center}
  \caption{Semantic categories of Datafun}
  \label{fig:sem-cats}
\end{figure}

\begin{figure}
  \begin{center}
    \begin{tabular}{cl}
      %% \multicolumn{2}{c}{\textbf{Set notation}}\\
      $\U{P}$ & Underlying set of the poset $P$\\
      $\stringset$ & Set of strings\\
      %% $A \boxtimes B$ & Cartesian product of sets $A$, $B$\\
      %% $A \boxplus B$ & Disjoint union of sets $A$, $B$\\
      %% $A \Arr B$ & Functions from set $A$ to set $B$
      %% \vspace{0.5em}\\
      %% \multicolumn{2}{c}{\textbf{Poset and semilattice notation}}\\
      \one & One-element poset $\{\triv\}$\\
      \two & Two-element poset $\{\sff,\stt\}$, with $\sff < \stt$\\
      $\N_\le$ & The naturals $\N$, as a (totally ordered) poset\\
      $P + Q$ & Disjointly-ordered poset on disjoint union of $P,Q$\\
      $P \x Q$ & Pointwise poset on pairs of $P$s and $Q$s\\
      $P \arr Q$ & Pointwise poset on monotone maps $\cPoset(P, Q)$\\
      %% $L \lol M$ & Pointwise poset on $\cSL(L,M)$\\
      $\slF\;P$ & Free semilattice on a poset $P$\\
      $\slU\;{L}$ & Underlying poset of a semilattice $L$\\
      $\Disc{A}$ & Discrete poset on underlying set $A$\\
      $\FS{A}$ & Free semilattice on a set $A$; same as $\slF\;(\Disc{A})$\\
      $\below{x}{P}$ &
      The sub-poset of $P$ below $x$: $\{y \in P ~|~ y \le x\}$
      %% \\ $\FM{A}{P}$ & Poset of finite maps from the set $A$ to the poset $P$
    \end{tabular}
  \end{center}

  \caption{Semantic notation}
  \label{fig:sem-notation}
\end{figure}

%% \todo{``Finite maps'' ambiguous terminology? Gibbons uses it differently, for
%%   example.}


\section{Semantics and metatheory}
\label{sec:semantics}

%% \todo{Somewhere in this we should talk about the comonad $\Disc{\U{\cdot}}$, the
%%   monad $\slU\;(\FS{\U{\cdot}})$, and adjoint logic.}

%% \todo{Should we talk about how $\Disc{\U{A + B}} = \Disc{\U{A} \boxplus \U{B}} =
%%   \Disc{\U{A}} + \Disc{\U{B}}$ and what this has to do with \ms{case}? Honestly
%%   I don't remember that exactly myself.}

We give a denotational semantics for Datafun in terms of three categories
(\cSet{}, \cPoset{}, and \cSL{}) and two adjunctions between them (see Figure
\ref{fig:sem-cats}). We present the notation we use in Figure
\ref{fig:sem-notation}; we take care to distinguish between sets and posets, and
since posets are more central to our semantics, most of our notation concerns
them. We take less care to distinguish posets and semilattices, since while a
set can be partially ordered in many ways, a poset either \emph{is} or \emph{is
  not} a semilattice.

\subsection{The category \cSL{}}

\cSL{} is the category of join-semilattices with least elements, which we call
simply ``semilattices''.

Directly, a semilattice is a poset $L$, with a least element $\unit$, in which
any two elements $a,b$ have a least-upper-bound $a \vee b$. From $\unit$ and
$\vee$ it follows that any finite subset $X \subseteq_{\ms{fin}} \U{L}$ has a
least upper bound, written $\bigvee X$.

A morphism $f \in \cSL(L, M)$ is a function from $\U{L}$ to $\U{M}$ satisfying:
\begin{eqnarray*}
  f(a \vee_A b) &=& f(a) \vee_B f(b)\\
  f(\unit_A) &=& \unit_B
\end{eqnarray*}

\cSL{} is a subcategory of \ms{Poset}; every \cSL{}-morphism $f$ is monotone,
since $a \le b \iff a \vee b = b$, and so from $a \le b$ we know $f(a) \vee f(b)
= f(a \vee b) = f(b)$, thus $f(a) \le f(b)$. Since it is a subcategory, we will
typically not explicitly write the forgetful functor $\mathsf{U}(L)$ which sends
semilattices to posets by forgetting the lattice structure.


\subsection{Denotation of Datafun types}
\begin{figure}
  \[\begin{array}{rcl}
  \den{A} &\in& \cPoset_0\\
  \den{\bool} &=& \two\\
  \den{\N} &=& \N_\le\\
  \den{\str} &=& \Disc{\mathbb{S}}\\
  \den{A \x B} &=& \den{A} \x \den{B}\\
  \den{A + B} &=& \den{A} + \den{B}\\
  \den{A \mto B} &=& \den{A} \arr \den{B}\\
  \den{A \uto B} &=& \Disc{\U{\den{A}}} \arr \den{B}\\
  \den{\Set{A}} &=& \FS{\U{\den{A}}}%% \\
  %% \den{\Map{A}{B}} &=& \FM{\U{\den{A}}}{\den{B}}
  %% \end{array}\]\[\begin{array}{rcl}
  \\\\
  \den{\GD}, \den{\GG} &\in& \cPoset_0\\
  \den{\cdot} &=& \one\\
  \den{\GD, x\of A} &=& \den{\GD} \x \den{A}\\
  \den{\GG{}, \m{x}\of A} &=& \den{\GG} \x \den{A}
  \end{array}\]
  \caption{Denotations of Datafun types and contexts}
  \label{fig:sem-types}
\end{figure}

Datafun types and contexts denote posets as shown in Figure \ref{fig:sem-types}.
To complete our semantics, we will need a few simple lemmas about the
denotations of Datafun types. First, we need to know that our semilattice types
are semilattices, and that our finite types are finite:

\begin{lemma}
  The denotation $\den{L}$ of a semilattice type $L$ is a semilattice.
\end{lemma}

\begin{lemma}
  The poset $\den{\fineq{A}}$ denoted by a finite eqtype $\fineq{A}$ is finite.
\end{lemma}

Second, to show that bounded fixed-points $(\fixle{\m{x}}{e_\top}{e})$
terminate, we need any possible $e_\top$ to pick out a finite-height sub-poset:

\begin{lemma}
  For any semilattice equality type $\eq{L}$, for any $x \in \den{\eq{L}}$, the
  height of $\below{x}{\den{\eq{L}}}$ is finite.
\end{lemma}

\paragraph{}
All of these are trivial to prove by induction over types.


%% TERM DENOTATION FIGURE
\newcommand{\fux}[2]{\Den{\vcenter{\infer{#1}{#2}}}}
%% \newcommand{\fuxn}[3]{\Den{\vcenter{\infer[\rn{#1}]{#2}{#3}}}}
\newcommand{\dg}{\;\delta\;\gamma}

\begin{figure*}
  \[\begin{array}{rcll}
  \textbf{Derivation}\;\phantom{\dg} && \textbf{Denotation}
  \vspace{.8em}\\
  \den{\J{\GD}{\GG}{e}{A}}\;\phantom{\dg} &\in&
  \cSet(\U{\den{\GD}},\,\cPoset(\den{\GG}, \den{A}))
  %% \U{\den{\GD}} \Arr \U{\den{\GG} \arr \den{A}}
  \vspace{.8em}\\
  \fux{\J{x_1\of A_1, ..., x_n\of A_N}{\GG}{x_i}{A_i}}{
    \phantom{.}}\dg
  &=& \pi_i\;\delta
  \vspace{.8em}\\
  \fux{\J{\GD}{\m{x}_1\of L_1, ..., \m{x}_n\of L_n}{\m{x}_i}{L_i}}{
    \phantom{.}}\dg
  &=& \pi_i\;\gamma
  \vspace{.8em}\\

  %% function rules
  \fux{\J{\GD}{\GG}{\fn\bind{x} e}{A \uto B}}{
    \J{\GD,x\of A}{\GG}{e}{B}}\dg
  &=& x \mapsto \den{e}\;\tuple{\delta,x}\;\gamma
  \vspace{.8em}\\
  \fux{\J{\GD}{\GG}{\fn\bind{\m{x}} e}{A \mto B}}{
    \J{\GD}{\GG,\m{x}\of A}{e}{B}}\dg
  &=& x \mapsto \den{e} \;\delta \;\tuple{\gamma,x}
  \vspace{.8em}\\
  \fux{\J{\GD}{\GG}{e_1\;e_2}{B}}{
    \J{\GD}{\GG}{e_1}{A \uto B} &
    \J{\GD}{\cdot}{e_2}{A}} \dg
  &=& \den{e_1}\dg\;(\den{e_2}\;\delta\;\triv)
  \vspace{.8em}\\
  \fux{\J{\GD}{\GG}{e_1\;e_2}{B}}{
    \J{\GD}{\GG}{e_1}{A \mto B} &
    \J{\GD}{\GG}{e_2}{B}} \dg
  &=& \den{e_1}\dg\;(\den{e_2}\dg)
  \vspace{.8em}\\

  %% product types
  \fux{\J{\GD}{\GG}{(e_1, e_2)}{A_1 \x A_2}}{
    \J{\GD}{\GG}{e_i}{A_i}}\dg
  &=& \pair{\den{e_1}\dg}{\den{e_2}\dg}
  \vspace{.8em}\\
  \fux{\J{\GD}{\GG}{\pi_i\;e}{A_i}}{
    \J{\GD}{\GG}{e}{A_1 + A_2}}\dg
  &=& \pi_i\;(\den{e}\dg)
  \vspace{.8em}\\

  %% sum type rules
  \fux{\J{\GD}{\GG}{\ms{in}_i\;e}{A_1 + A_2}}{
    \J{\GD}{\GG}{e}{A_i}}\dg
  &=& \ms{in}_i\;(\den{e}\dg)
  \vspace{.8em}\\
  \fux{\J{\GD}{\GG}{\case{e}{x}{e_1}{x}{e_2}}{B}}{
    \J{\GD}{\cdot}{e}{A_1 + A_2} &
    \J{\GD,x\of A_i}{\GG}{e_i}{B}}
  \dg
  &=&
  \begin{cases}
    \den{e_1}\;\pair{\delta}{x}\;\gamma
    &\text{if }\den{e}\;\delta\;\triv = \ms{in}_1\;x\\
    \den{e_2}\;\pair{\delta}{x}\;\gamma
    &\text{if }\den{e}\;\delta\;\triv = \ms{in}_2\;x\\
  \end{cases}
  \vspace{.8em}\\
  \fux{\J{\GD}{\GG}{\case{e}{\m{x}}{e_1}{\m{x}}{e_2}}{B}}{
    \J{\GD}{\GG}{e}{A_1 + A_2} &
    \J{\GD}{\GG,\m{x}\of A_i}{e_i}{B}
  }\dg
  &=&
  \begin{cases}
    \den{e_1}\;\delta\;\pair{\gamma}{x}
    &\text{if }\den{e} \dg = \ms{in}_1\;x\\
    \den{e_2}\;\delta\;\pair{\gamma}{x}
    &\text{if }\den{e} \dg = \ms{in}_2\;x\\
  \end{cases}
  \vspace{.8em}\\

  %% boolean rules
  \fux{\J{\GD}{\GG}{\ms{true}}{\bool}}{\phantom{.}}\dg
  &=& \stt
  \vspace{.8em}\\
  \fux{\J{\GD}{\GG}{\ms{false}}{\bool}}{\phantom{.}}\dg
  &=& \sff
  \vspace{.8em}\\
  \fux{\J{\GD}{\GG}{\ifthen{e}{e_1}{e_2}}{A}}{
    \J{\GD}{\cdot}{e}{\bool} &
    \J{\GD}{\GG}{e_i}{A}} \dg
  &=&
  \begin{cases}
    \den{e_1}\dg & \text{if}~ \den{e}\;\delta\;\triv = \stt\\
    \den{e_2}\dg & \text{if}~ \den{e}\;\delta\;\triv = \sff
  \end{cases}
  \vspace{.8em}\\

  \fux{\J{\GD}{\GG}{\ifthen{e}{e_1}{\unit}}{L}}{
    \J{\GD}{\GG}{e}{\bool} &
    \J{\GD}{\GG}{e_1}{L}} \dg
  &=&
  \begin{cases}
    \den{e_1}\dg & \text{if}~ \den{e}\dg = \stt\\
    \unit_{\den{L}} & \text{if}~ \den{e}\dg = \sff
  \end{cases}
  \vspace{.8em}\\

  %% semilattice rules
  \fux{\J{\GD}{\GG}{\unit}{L}}{\phantom{.}}\dg
  &=& \unit_{\den{L}}
  \vspace{.8em}\\
  \fux{\J{\GD}{\GG}{e_1 \vee e_2}{L}}{
    \J{\GD}{\GG}{e_i}{L}}\dg
  &=& \den{e_1}\dg \vee_{\den{L}} \den{e_2}\dg
  \vspace{.8em}\\

  %% set intro/elim rules
  \fux{\J{\GD}{\GG}{\singleset{e}}{\Set{A}}}{\J{\GD}{\cdot}{e}{A}}\dg
  &=& \{\den{e}\;\delta\;\triv\}
  \vspace{.8em}\\
  \fux{\J{\GD}{\GG}{\letin{x}{e_1}{e_2}}{L}}{
    \J{\GD}{\GG}{e_1}{\Set{A}} &
    \J{\GD,x\of A}{\GG}{e_2}{L}}\dg
  &=& \displaystyle\bigvee \left\{
  \den{e_2}\;\tuple{\delta,x}\;\gamma
  ~|~ {x \in \den{e_1}\dg}\right\}
  \vspace{.8em}\\

  %% fix rules
  \fux{\J{\GD}{\GG}{\fix{\m{x}}{e}}{\fineq{L}}}{
    \J{\GD}{\GG,\m{x}\of \fineq{L}}{e}{\fineq{L}}
  }\dg
  &=&
  \lfpin{\den{\fineq{L}}}{(x \mapsto \den{e}\;\delta\;\pair{\gamma}{x})}
  \vspace{0.8em}\\
  \fux{\J{\GD}{\GG}{\fixle{\m{x}}{e_1}{e_2}}{\eq{L}}}{
    \J{\GD}{\GG}{e_1}{\eq{L}} &
    \J{\GD}{\GG,\m{x}\of \eq{L}}{e_2}{\eq{L}}}\dg
  &=&
  \lfpin{\below{\den{e_1}\dg}{\den{\eq{L}}}}{
    \left(x \mapsto
    %% \ms{clamp}(\den{e_2}\;\delta\;\pair{\gamma}{x}, \den{e_1} \dg)
    \begin{cases}
      \den{e_2}\;\delta\;\pair{\gamma}{x}
      %% & \text{if}~\den{e_2}\;\delta\;\pair{\gamma}{x} \le \den{e_1}\dg\\
      & \text{if it's} \le \den{e_1}\dg\\
      \den{e_1} \dg & \text{otherwise}
    \end{cases}
    \right)}
  \end{array}\]

  \caption{Denotations of Datafun typing derivations}
  \label{fig:sem-terms}
\end{figure*}


\subsection{Denotation of Datafun terms}

In Figure \ref{fig:sem-terms} we give a denotation for typing derivations with
the following signature:
\begin{eqnarray*}
  \den{\J{\GD}{\GG}{e}{A}} &\in&
  \cSet(\U{\den{\GD}}, \cPoset(\den{\GG}, \den{A}))
  %% \\ &\in&
  %% \U{\den{\GD}} \Arr \U{\den{\GG} \arr \den{A}}
\end{eqnarray*}

Colloquially, $\J{\GD}{\GG}{e}{A}$ denotes a function from $\den{\GD} \x
\den{\GG}$ to $\den{A}$ that must be monotone in $\den{\GG}$ (but not in
$\den{\GD}$).

Our semantics requires the following lemma regarding fixed-points of monotone
functions:

\begin{lemma}[Fixed points in finite-height pointed posets]
  Any monotone map $f : P \to P$ on a poset $P$ of finite height with a least
  element $\unit$ has a least fixed point of the form $f^n(\unit)$.
\end{lemma}

\begin{proof}
  Consider the sequence $\unit, f(\unit), f^2(\unit), f^3(\unit), ...$. Note that
  $\unit \le f(\unit)$, so by monotonicity of $f$ and induction $f^i(\unit) \le
  f^{i+1}(\unit)$. Thus this sequence forms an ascending chain. Since $P$ has
  finite height, this chain cannot be infinite; thus there is an $n$ such that
  $f^n(\unit) = f^{n+1}(\unit)$, i.e. $f^n(\unit)$ is a fixed-point of $f$.

  Now consider any fixed-point $x$ of $f$. Since $\unit \le x$, by monotonicity of
  $f$, induction, and $x = f(x)$, we have $f^n(\unit) \le x$. Thus $f^n(\unit)$ is
  the least fixed point of $f$.
\end{proof}

We write $(\lfpin{L}{f})$ for the least fixed point of a monotone map $f$ on a
semilattice $L$ of finite height.

%% \paragraph{Lemma 1: Existence of fixed points in posets of finite height.}
%% Any monotone map $f : A \to A$ on a nonempty poset $A$ of finite height has
%% at least one fixed point.

%% \paragraph{Lemma 2: Finding least fixed points in pointed posets.} For any
%% poset $A$ with a least element $\unit$, for any monotone map $f : A \to A$,
%% if $f$ has a fixed point $x$ of finite height\footnote{The height of an
%% element $x$ in a poset $A$ is the height of the sub-poset $\{y \in A ~|~ y
%% \le x\}$.}, then $f$ has a least fixed point of the form $f^n(\unit)$ for
%% some $n \in \N$.


%% FIGURE: SUBSTITUTIONS
\begin{figure*}
  \[\begin{array}{rcll}
    \sub{e/v} v &=& e\\
    \sub{e/v} x &=& x\\
    \sub{e/v} (\fn\bind{u} e') &=& \fn\bind{u} \sub{e/v} e'\\
    \sub{e/v} (e_1 \;e_2) &=& (\sub{e/v} e_1)\;(\sub{e/v} e_2)\\
    \sub{e/v} (e_1, e_2) &=& (\sub{e/v} e_1, \sub{e/v} e_2)\\
    \sub{e/v} (\pi_i\;e') &=& \pi_i\;(\sub{e/v}e')\\
    \sub{e/v} (\ms{in}_i\;e') &=& \ms{in}_i \;(\sub{e/v} e')\\
    \sub{e/v} (\case{e'}{u}{e_1}{u}{e_2})
    &=& \case{\sub{e/v} e'}{u}{\sub{e/v} e_1}{u}{\sub{e/v} e_2}\\
    \sub{e/v} \unit &=& \unit\\
    \sub{e/v} (e_1 \vee e_2) &=& \sub{e/v} e_1 \vee \sub{e/v} e_2\\
    \sub{e/v} (\forin{x \in e_1} e_2)
    &=& \forin{x \in \sub{e/v} e_1} \sub{e/v} e_2\\
    \sub{e/v} (\fix{\m{x}} e') &=& \fix{\m{x}} \sub{e/v} e'\\
    \sub{e/v} (\fixle{\m{x}}{e_1}{e_2}) &=& \fixle{\m{x}}{\sub{e/v}{e_1}} e_2
  \end{array}\]
  \caption{Substitution}
  \label{fig:substitution}
\end{figure*}


\subsection{Metatheory of substitution}

We define substitution of Datafun terms in Figure \ref{fig:substitution}. We
could define separate forms of ordinary $\sub{e_1/x} e$ and monotone
$\sub{e_1/\m{x}} e$ substitution, but to avoid repetition we instead use the
metavariables $v,u$ to stand for any variable (ordinary or monotone) and define
$\sub{e_1/v} e$. Terms are taken up to $\alpha$-equivalence, and by convention,
whenever two variables are given distinct names $v,u$ it is assumed $v \ne u$.

We have proven the following theorems:

\begin{theorem}[Weakening and exchange]
  The rules \begin{mathpar}
    \infer[\rn{\rt{weak}}]{
      \J{\GD,\GD'}{\GG,\GG'}{e}{A}
    }{
      \J{\GD}{\GG}{e}{A}}
    \and
    \infer[\rn{\rt{xchg}}]{
      \J{\GD_1,\GD_2}{\GG_1,\GG_2}{e}{A}
    }{
      \J{\GD_2,\GD_1}{\GG_2,\GG_1}{e}{A}
    }
  \end{mathpar}
  are admissible.
\end{theorem}

\begin{theorem}[Substitution, ordinary]
  From
  \begin{itemize}
  \item $\J{\GD}{\cdot}{e_1}{A}$,
  \item and $\J{\GD,x \of A}{\GG}{e_2}{B}$,
  \end{itemize}
  it follows that
  \begin{itemize}
  \item $\J{\GD}{\GG}{\sub{e_1/x}e_2}{B}$,
  \item and $\den{\sub{e_1/x} e_2} \dg = \den{e_2} \;\pair{\delta}{\den{e_1}\dg}
    \;\gamma$.
  \end{itemize}
\end{theorem}

\begin{theorem}[Substitution, monotone]
  From
  \begin{itemize}
  \item $\J{\GD}{\GG}{e_1}{A}$,
  \item and $\J{\GD}{\GG,\m{x}\of A}{e_2}{B}$
  \end{itemize}
  it follows that
  \begin{itemize}
  \item $\J{\GD}{\GG}{\sub{e_1/\m{x}}e_2}{B}$,
  \item and $\den{\sub{e_1/\m{x}} e_2} \dg = \den{e_2} \;\delta
    \;\pair{\gamma}{\den{e_1} \dg}$.
  \end{itemize}
\end{theorem}

% \todo{\paragraph{} More yik-yak here?}

%% \paragraph{}
%% \todo{TODO: future work reference? on operational semantics \& compatibility theorem?}
