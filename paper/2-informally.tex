%% FIGURE: CORE SYNTAX
\begin{figure}
  \[\begin{array}{ccl}
    %% types
    A, B     &\bnfeq& \bool \pipe \N \pipe \str \pipe \Set{A}
                      \pipe \Map{A}{B}
    \\
    \textsf{types} && A + B \pipe A \x B \pipe A \uto B \pipe A \mto B
    \vspace{0.5em}\\
    %% semilattice types
    L, M         &\bnfeq& \bool \pipe \N \pipe \Set{A} \pipe \Map{A}{L}
    \\
    \textsf{semilattice types} && L \x M \pipe A \uto L \pipe A \mto L
    \vspace{0.5em}\\
    %% equality types
    \eq{A}, \eq{B} &\bnfeq& \bool \pipe \N \pipe \str \pipe \Set{\eq{A}}
                            \pipe \Map{\eq{A}}{\eq{B}}
    \\
    \textsf{eqtypes} && \eq{A} + \eq{B} \pipe \eq{A} \x \eq{B}
    \vspace{0.5em}\\
    %% finite equality types
    \fineq{A},\fineq{B}
    &\bnfeq& \bool \pipe \Set{\fineq{A}}
       \pipe \Map{\fineq{A}}{\fineq{B}}
       \pipe \fineq{A} + \fineq{B} \pipe \fineq{A} \x \fineq{B}\\
    \textsf{finite eqtypes}
    \vspace{0.5em}\\
    %% contexts
    \GD &\bnfeq& \cdot \pipe \GD, x\of A\\
    \GG &\bnfeq& \cdot \pipe \GG{},\m{x}\of A\\
    \textsf{contexts}\\
    %% expressions
    e &\bnfeq& x \pipe \m{x} \pipe \fn\bind{x} e \pipe \fn\bind{\m{x}} e
    \pipe e\;e\\
    \textsf{terms}
    && (e,e) \pipe \pi_1\;e \pipe \pi_2\;e
    \pipe \ms{in}_1\;e \pipe \ms{in}_2\;e\\
    && \case{e}{x}{e}{x}{e}\\
    && \case{e}{\m{x}}{e}{\m{x}}{e}\\
    && \ms{true} \pipe \ms{false} \pipe \ifthen{e}{e}{e}\\
    && \singleset{e} \pipe \singlemap{e}{e}
       \pipe \unit \pipe e \vee e \pipe \letin{x}{e}{e}\\
    && \fix{\m{x}}{e} \pipe \fixle{\m{x}}{e}{e}
  \end{array}\]
  \caption{Syntax of core Datafun}
  \label{fig:syntax}
\end{figure}

\section{Datafun, informally}

We give the core syntax of Datafun in Figure \ref{fig:syntax}. Datafun is a
simply-typed $\lambda$-calculus extended in four major ways:

\begin{enumerate}
\item We add a type of finite sets, $\Set{A}$. We also include finite maps
  $\Map{A}{B}$, as they are useful for some examples.

  \todo{Describe why sets are useful?}

  %% We use finite sets to represent Datalog predicates; one might also think of
  %% them as tables or views in a database setting.

\item We add a type of \emph{monotone functions}, $A \mto B$. Consequently
  Datafun has two flavors of variable, \emph{ordinary} and \emph{monotone}. We
  write ordinary variables in $script$ and monotone variables in \m{bold}.

  In order for ``monotone'' to have meaning, our types are implicitly partially
  ordered:
  \begin{itemize}
  \item Booleans $\bool$ are ordered $\ms{false} < \ms{true}$.
  \item Natural numbers $\N$ have the usual order: $0 < 1 < 2 < ...$.
  \item We have no particular use-case for comparing strings $\str$ in
    this paper, so we order them discretely; $a \le b$ iff $a = b$. \todo{Better
      explanation?}
  \item Pairs and functions are ordered pointwise:
    \begin{itemize}
    %% \item $(a_1, b_1) \le (a_2, b_2)$ iff $a_1 \le a_2 \wedge b_1 \le b_2$
    \item $(a, x) \le (b, y)$ iff $a \le b \wedge x \le y$
    \item $f \le g$ iff $\forall \bind{x} f(x) \le g(x)$
    \end{itemize}
  \item Sum types are ordered disjointly: $\ms{in}_i\; a \le
    \ms{in}_i\; b$ iff $a \le b$, but $\ms{in}_1\; a$ and $\ms{in}_2\; b$ are
    never comparable.
  \item Sets are ordered by inclusion: $a \le b$ iff $a \subseteq b$.
  \item Maps are ordered both by inclusion and pointwise.
  \end{itemize}

\item We add a term $(\fix{\m{x}}{e})$ denoting the least fixed point of the
  monotone function $(\fn\bind{\m{x}} e)$. \todo{Explain why this is useful?}
  This is computed (modulo optimizations) by iteration, starting from the
  smallest value of the desired type and halting once a fixed point is found.
  This strategy constrains the types of \ms{fix} terms in several ways:
  \begin{itemize}
  \item The type must have a smallest value. We enforce this using semilattice
    types (see item \ref{item:semilattice-types}, below).

  \item The type must support equality tests, to determine when a fixed point
    has been reached. We call a type supporting equality tests an \emph{eqtype}.
    \todo{CITE}

  \item To ensure termination, the type must have finite height.\footnote{The
    height of a poset is the cardinality of its largest chain (totally-ordered
    subset).} We conservatively approximate this property by limiting \ms{fix}
    to finite types.
  \end{itemize}

  In summary, \ms{fix} may only be used at \emph{finite semilattice eqtypes}.

  \todo{TODO: connection to Datalog via finiteness of predicates}

  \todo{explain $\fixle{\m{x}}{e_\top}{e}$?}

  %% \todo{Explain:
  %%   \begin{itemize}
  %%   \item compute fixpoint (at least conceptually) by iteration
  %%   \item start from a least element
  %%   \item to ensure termination, need finite height of lattice
  %%   \item finiteness guarantees finite height!
  %%   \item (\emph{only then} mention analogy to Datalog)
  %% \end{itemize}}

\item\label{item:semilattice-types} Generalizing the empty set $\emptyset$ and
  union $\cup$, we identify a subset of types that have a \emph{least element}
  $\unit$ and a \emph{least upper bound} operator $\vee$. We call these
  \emph{semilattice types}\footnote{Technically, the partial orderings on these
    types form \emph{join-semilattices with a least element}. For brevity's
    sake, we call these structures simply ``semilattices.''}, and denote them by
  the metavariables $L,M$.

  Semilattice types serve two primary purposes:

  \textbf{First}, they provide a natural eliminator for sets. Given $e :
  \Set{A}$, we write $\letin{x}{e'}{e_x}$ for the least upper bound, over all
  elements $x \in e'$, of $e_x$, for $e_x$ of some semilattice type $L$. If
  $e_x$ is a set, for example, this provides the set type's monadic ``bind''
  operation.%% For example, $\forin{x \in \setlit{1,2,3}} \{10 \cdot x, x^2\}$
  %% denotes the set $\{1, 4, 9, 10, 20, 30\}$.

  \todo{example?}

  \todo{explain testing for set membership?}

  \textbf{Second}, as already mentioned, they guarantee the presence of a least
  element, needed to compute \ms{fix} terms.

  \todo{Explain how products of semilattice being semilattice + monotone
    fixed-points account for mutual recursion.}

\end{enumerate}
