%% FIGURE: CORE SYNTAX
\begin{figure}
  \[\begin{array}{ccl}
    %% types
    A, B     &\bnfeq& \bool \pipe \N \pipe \str \pipe \Set{A}
                      \pipe \Map{A}{B}
    \\
    \textsf{types} && A + B \pipe A \x B \pipe A \uto B \pipe A \mto B
    \vspace{0.5em}\\
    %% semilattice types
    L, M         &\bnfeq& \bool \pipe \N \pipe \Set{A} \pipe \Map{A}{L}
    \\
    \textsf{semilattice types} && L \x M \pipe A \uto L \pipe A \mto L
    \vspace{0.5em}\\
    %% equality types
    \eq{A}, \eq{B} &\bnfeq& \bool \pipe \N \pipe \str \pipe \Set{\eq{A}}
                            \pipe \Map{\eq{A}}{\eq{B}}
    \\
    \textsf{eqtypes} && \eq{A} + \eq{B} \pipe \eq{A} \x \eq{B}
    \vspace{0.5em}\\
    %% finite equality types
    \fineq{A},\fineq{B}
    &\bnfeq& \bool \pipe \Set{\fineq{A}}
       \pipe \Map{\fineq{A}}{\fineq{B}}
       \pipe \fineq{A} + \fineq{B} \pipe \fineq{A} \x \fineq{B}\\
    \textsf{finite eqtypes}
    \vspace{0.5em}\\
    %% contexts
    \GD &\bnfeq& \cdot \pipe \GD, x\of A\\
    \GG &\bnfeq& \cdot \pipe \GG{},\m{x}\of A\\
    \textsf{contexts}\\
    %% expressions
    e &\bnfeq& x \pipe \m{x} \pipe \fn\bind{x} e \pipe \fn\bind{\m{x}} e
    \pipe e\;e\\
    \textsf{terms}
    && (e,e) \pipe \pi_1\;e \pipe \pi_2\;e
    \pipe \ms{in}_1\;e \pipe \ms{in}_2\;e\\
    && \case{e}{x}{e}{x}{e}\\
    && \case{e}{\m{x}}{e}{\m{x}}{e}\\
    && \ms{true} \pipe \ms{false} \pipe \ifthen{e}{e}{e}\\
    && \singleset{e} \pipe \singlemap{e}{e}
       \pipe \unit \pipe e \vee e \pipe \letin{x}{e}{e}\\
    && \fix{\m{x}}{e} \pipe \fixle{\m{x}}{e}{e}
  \end{array}\]
  \caption{Syntax of core Datafun}
  \label{fig:syntax}
\end{figure}

\section{Datafun, informally}

We give the core syntax of Datafun in Figure \ref{fig:syntax}. Datafun is a
simply-typed $\lambda$-calculus extended in four major ways:

\begin{enumerate}
\item We add a type of finite sets, $\Set{A}$. We also include finite maps
  $\Map{A}{B}$, as they are useful for some examples.

  \todo{Describe why sets are useful?}

  %% We use finite sets to represent Datalog predicates; one might also think of
  %% them as tables or views in a database setting.

\item We add a type of \emph{monotone functions}, $A \mto B$. Consequently
  Datafun has two flavors of variable, \emph{ordinary} and \emph{monotone}.
  Ordinary functions introduce ordinary variables; monotone functions introduce
  monotone variables. This distinction is left latent in the formal syntax,
  \todo{TODO: why?}, but for clarity our examples put monotone variables in
  \textbf{bold}.

  In order for ``monotone'' to have meaning, our types are implicitly partially
  ordered. \todo{Discuss orderings}. Sets are ordered by inclusion: $a \le b :
  \Set{A}$ iff $a \subseteq b$. Maps are ordered both by inclusion and
  pointwise.

\item We add a term $(\fix{\m{x}}{e})$ denoting the least fixed point of the
  monotone function $(\fn\bind{\m{x}} e)$.

  \todo{Explain:
    \begin{itemize}
    \item compute fixpoint (at least conceptually) by iteration
    \item start from a least element
    \item to ensure termination, need finite height of lattice
    \item finiteness guarantees finite height!
    \item (\emph{only then} mention analogy to Datalog)
  \end{itemize}}

  To ensure termination, the type at which this fixed-point is taken must be
  \emph{finite} (this corresponds directly to Datalog's requirement that
  predicates be finite). To ensure uniqueness of the least fixed point, the type
  must have a \emph{least element}, which we guarantee by only allowing fixed
  points at \emph{semilattice types} (see point \ref{item:semilattice-types},
  just below).

\item\label{item:semilattice-types} Generalizing the empty set $\emptyset$ and
  union $\cup$, we identify a subset of types that have a \emph{least element}
  $\unit$ and a \emph{least upper bound} operator $\vee$. We call these
  \emph{semilattice types}\footnote{Technically, the partial orderings on these
    types form \emph{join-semilattices with a least element}. For brevity's
    sake, we call these structures simply ``semilattices.''}, and denote them by
  the metavariables $L,M$.

  Semilattice types are the ``glue'' of Datafun; they serve several small
  purposes that together simplify and unify the language.

  First, they provide a natural eliminator for sets: given $e : \Set{A}$ we may
  write $\letin{x}{e}{e'}$ for the least upper bound, over all elements $x \in
  e$, of $e'$, as long as $e' : L$ has a semilattice type. If $e' : \Set{B}$ is
  itself a set, for example, this serves as the set type's monadic ``bind''
  operation. For example, $\forin{x \in \setlit{1,2,3}} \{10 \cdot x, x^2\}$
  denotes the set $\{1, 4, 9, 10, 20, 30\}$.

  \todo{explain testing for set membership?}

  Second, we restrict fixed-points $(\fix{\m{x}}{e})$ to semilattice types,
  which guarantees that we take

  \todo{Explain how products of semilattice being semilattice + monotone
    fixed-points account for mutual recursion.}

  \todo{Rewrite this to be clearer.}

\end{enumerate}
