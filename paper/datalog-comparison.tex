\section{Comparing Datalog and Datafun}
\label{sec:datalog-vs-datafun}

At this point, we have
demonstrated by example that Datafun programs are rather similar to
Datalog programs, and we have given the typing and denotational
semantics of Datafun. However, we still need to explain \emph{why} our
semantics lets us express Datalog-style programs.

To understand this, recall that Datalog is a bottom-up logic
programming language. A program consists of a primitive database of
facts, along with a set of rules the rules the programmer wrote. A
Datalog program executes by using the rules to derive new conclusions
from the database, and extending the database with them, until no
additional conclusions can be drawn. Then the query can be checked
simply by seeing if it occurs in the final database.

This is, essentially, a fixed point computation -- each stage of
execution of a Datalog program takes a database and returns an
extended database, until a fixed point is reached. The stratified
negation restriction essentially ensures that the database transformer
defined by a Datalog program is a monotone function on the set of
facts. This is why the type system of Datafun tracks the monotonicity
of functions --- since we permit both higher-order definitions and
taking fixed points, we need to ensure that the body of a fixed point
definition is monotone in order to guarantee that the recursion is
well-founded.

This ensures that the recursive definition is well-defined, but is not
sufficient by itself to guarantee termination. To manage this, Datalog depends
upon the other two restrictions described in the introduction. By restricting
terms occuring in predicates to consist of either atoms or variables, Datalog
ensures that quantifiers need only be instantiated with the atoms used in a
program. By requiring every variable in the consequent of rules to also occur in
the premise of a rule, it ensures that every consequent will also only feature
atoms occuring in the original program.

Then, since there can only be finitely many atoms in a finite program,
this means that the set of possible arguments to a predicate is itself
finite. Then the lattice of sets of atomic predicates ordered by
inclusion will be finite, and so fixed point iteration is guaranteed
to terminate.

Instead of this (rather indirect) scheme, Datafun directly tracks the
finiteness of types, permitting recursion only if it is over a finite
type, or is bounded explicitly. These two approaches achieve the same
effect, albeit in different ways. Datalog's approach has the benefit
that no type discipline is needed to ensure finiteness. One advantage
of our choice is that we permit recursion over any semilattice, not
just the semilattice of sets. A much more serious advantage of our
approach is that it makes it much easier to write fixed-point
computations which actually \emph{compute} with the data they see (for
example, the CYK parser we wrote computed lengths of substrings). 
